%XXX add a chapter about limit switch wiring somewhere XXX
%XXX add a chapter where to put the reference switch XXX
%XXX add about stepper motor coils XXX
%XXX add chapter about the units XXX
%XXX add chapter steppers in general XXX
%XXX implement automatic checking for lost steps XXX
%XXX implement G-code limit checks in software and maybe tool size to jogging
%XXX implement homing reset value
%XXX backslash compensation

\chapter{Setting Up and Configuring}


EazyCNC should work out of the box without any setup or configuration,
so you can play around and test it right away.

However before you can actually use it with TOAD4 or TOAD5 and machine something
there are a few things you need to set up and configure.

This chapter proceeds in the preferred order of setting things up,
however I suggest you read it all through once before setting up
the system.

The two most important thing to set up is to configure each axis to match the motor
characteristics and axis gearing.

\section{Saving Your Setup}

EazyCNC stores all setup and configuration information as well
as the current machine state in file named 'EazyCNC-Mach-Config.ecnc' 
in a directory named 'EazyCNC' in your 'home' directory.

Your 'home' directory location depends on your operating system as
follows:


\begin{tabular}{ll}
\hline
Mac OS X & \path{/Users/username} \\
Windows 10 & \path{C:\Users\username} \\
Linux & \path{/Users/username} \\
\hline
\end{tabular}

where \emph{username} is the name you use when you log into your computer.


Note, EazyCNC does not ever automatically save the settings,
this is to protect you from accidentally altering your
carefully crafted machine setup and configuration. To
save your settings or any changes you've made you need to
click the 'SAVE' button. 

If the file or folder does not exist EazyCNC will create them
with reasonable default settings when you click the 'SAVE' 
button.

You do not have to 'worry' about the configuration file but it is good to know
about it as you may want to make backup copies of it or maintain
several different ones for different system configurations.

The file is in a plain text format so it is ok to view and even
edit it manually, though that is not recommend unless you
know what you are doing. 

Many programs like Mac OS X Textedit may mess things up by 
adding formatting and changing the file name extension with
.txt or .rtf, so learn to avoid those if you venture there.

\subsection{Setting up permission in Linux}
\label{subsec:setting-up-permissions-in-linux}

If you are not running EazyCNC under Linux you can skip this section.

Starting with EazyCNC version 2.0.41 if you install it from the supplied .deb
file you do not have to worry about udev rules and can also skip this section.

Most Linux distros take a very serious view of security. This means that by default
you are not even allowed to use your own devices! To complicate matters there is
no easy way to give yourself the necessary permissions, instead you have to do
the following.

To allow EazyCNC to talk to the TOAD4 or TOAD5 you need to create a file named:
 
\begin{verbatim}
99-TOADX.rules 
\end{verbatim}

 and put it in the directory 

\begin{verbatim}
 /etc/udev/rules.d 
\end{verbatim}

In the  file '99-TOAD4.rules' you need to put the following line (all on one line):

\begin{verbatim}
SUBSYSTEMS=="usb", ATTRS{idProduct}=="000a", ATTRS{idVendor}=="0408", MODE="0666", GROUP="plugdev"
\end{verbatim}

Basically you can do that with any text editor. Unfortunately you do not have the permission 
to write it to the directory '/etc/udev/rules.d'. 
So I recommend creating the file in your home directory and then use the 
Terminal and type in the following command to copy the file over:

\begin{verbatim}
sudo cp ~/99-TOAD4.rules /etc/udev/rules.d
\end{verbatim}

This command will ask for your password to allow you temporarily write to that directory.

Alternatively following one-liner should accomplish setting up the rules for you:

\begin{verbatim}
sudo echo 'SUBSYSTEMS=="usb", ATTRS{idProduct}=="000a", ATTRS{idVendor}=="0408", MODE="0666", GROUP="plugdev"' >> /etc/udev/99-TOAD4.rules
\end{verbatim}

Note that above absolutely must be on one line, so if you copy paste it from here,
make a practice paste to a text file to see that you got it all on one line.



\section{Updating Motor Controller Firmware } 
\label{sec:updating-motor-controller-firmware}

You can check which firmware version your TOAD4 or TOAD5 board currently has
by activating the MACH button. The version will be displayed on the status
display.

At the time of updating this manual for EazyCNC 2.0.41 the latest firmware was 2.0.10-4 for TOAD4 
and 2.0.10-5 for TOAD5. The last number of the version number indicates which hardware
the firmware is for.

Never update the firmware using a ICSP programmer or PICKit!

You can upate the firmware in The Motor Controller screen, see .
\screencaps{mach-setup-view-Controller-TOAD4+}{0.5}{Mach Setup when Controller button is selected} 

To update the firmware do as follows.

Connect your TOAD hardware to a USB port and power up the controller. 

For safety it is better if none of the machinery that the 
controller is controlling are powered up at the same time!

Ensure that EazyCNC is NOT in MACH mode.

Select which firmware version you want to upload from the Firmware -popup.

Click the UPDATE -button.

At this point a dialog box will appear that shows the progress of the firmware
update. 

\emph{Do not touch anything in the computer or hardware during the update.}

If the firmware update is interrupted you may have to force the TOAD board
into the firmware update mode, check the corresponding TOAD hardware manual.

\subsection{Firmware -popup menu}

EazyCNC comes with all the firmwares that have been release at the time of
te EazyCNC release and those firmwares are listed in this popup menu.

If the firmware you want to upload is not listed you can
either update your EazyCNC, which should never be taken lightly and without
good judgement, or you can obtain the firmware file (.hex) separately.

If you want to use a separate firmware file select the the 'From file...' 
option from the popup menu.


\subsubsection{UPDATE -button}

When you click this button the firmware will be updated.

If you selected 'From file...' in the 'Firmware' -popup then a file dialog
will be present at this point and you need to select the correct firmware
file.

During the update a dialog box detailing the update progress is shown. 
This dialog box will disapper after the update is complete. A success/fail
message will be show briefly in the Status Display, if you miss that you
can always press and hold down the little button marked '?' next to it.

If update was succesful the firmware version should show up in the Status Display
when you click the MACH button.


\subsection{Testing the Motor Controller Connection } 

To test that the motor controller is connected power it up and click the MACH button in
the EazyCNC user interface.

If everything is as working properly you should see a message next to the button marked 
with '?' that says something like:

\begin{verbatim}
Controller: Connected -- firmware version x.y.z-q
\end{verbatim}

\section{Enabling Debug Logs}
\label{sec:enable-debug-log}

EazyCNC can collect diagnostic information into log files stored inside your
EazyCNC directory to help in trouble shooting. 

By default the collection of that information is not enabled as it
can slow things down which in itself can cause problems.



Your EazyCNC directory location depends on your operating system as follows:


\begin{tabular}{ll}
\hline
Mac OS X & \path{/Users/username} \\
Windows 10 & \path{C:\Users\username} \\
Linux & \path{/Users/username} \\
\hline
\end{tabular}

where \emph{username} is the name you use when you log into your computer.

Log files are periodically deleted from the file system if they are older than seven days.

\subsection{Enable Java Console to file logging -checkbox}

EazyCNC outputs diagnostic information to a window name Java Console.
You can bring up (or hide) the Java Console by pressing the F12 key.

If this checkbox is ticked then anything that is output to the Java Console
is also written to a text file in the directory named 'java-console' inside your EazyCNC 
directory. 

These are all text files and you can open them with a text editor should you wish to have
a peek.

\subsection{Enable TOAD4 communication logging -checkbox}

If this checkbox is ticked then all communication is between EazyCNC and TOAD4/TOAD5
is written to a binary file in the directory named 'debuglogs' inside your EazyCNC 
directory. 

These are binary files and you need some special knowledge to be able to view them.


\section{Setting up the User Interface}
\label{sec:mach-setup-units}

Now that the communication works and before we set up the 
motors and axis we want to select the length units you are comfortable
with. 

To do that go the  'User Interface' setup screen 
\screencaps{mach-setup-view-Screen}{0.5}{The User Interface setup screen} 

EazyCNC supports working in millimeters or inches. All displays and entry fields
will always show values in the selected units and accept values in these
units. You can change the units at any time and it will not confuse EazyCNC,
so you can use millimeters to set up things and then switch to inches; however
even if EazyCNC will not get confused, you may, so it is best to select
one system of units and stick with it.

Note that regardless of the units selected here the G-code file can contain
coordinates that are expressed in millimeters (G21 mode) or inches (G20 mode),
and this is perfectly fine, as long as the correct G20/G21 mode is specified
in the G-code file.


\subsection{Units -popup menu}

There are two options.

'mm' -- with this setting all the entry fields and DROs displays are
in millimeters.

'inch' -- with this setting all the entry fields and DROs are
in inches.

\subsection{DRO-format -entry field}

With this entry field you can control how numbers in the entry fields and DROs are
displayed.

The main usage is to control the number of decimals you want displayed, to do 
that just enter '0.' followed by as many '0' characters as you want decimals.

For example with inches it is probably preferable to use three decimals to see
the 'thous' so enter '0.000' in this field, for working with millimeters
'0.00' is probably best.

\subsection{Update rate -entry field}

This entry field controls how many times per second (frames per second, fps) the
toolpath display is updated. Smaller values than 10 may produce jerky updates and
higher values than 30 are unnecessary and may bog down the computer.

\subsection{Language -popup menu}

This popup allows you to select the language used in the user interface. 


\subsection{Screen Size -popup menu}

This popup allows you to select between 'Standard' and 'Compact' screen layouts.

Depending on the operating system and the various docks and toolbars in them
the 'Compact' layout should fit a screen as small as 1024 x 600 at a pinch.

Note that you need to both Save the configuration and re-start the program. If
you are running in a small screen the Save-button may not be visible, in that
case use the Alt-Cmd-S for Mac OS X or Alt-Ctrl-S for Windows/Linux to save the
the configuration before re-starting EazyCNC.

\subsection{Machined Path -settings}
\label{subsec:machined-path-color-settings}

The settings in this panel control how the tool path for the already cut path is 
displayed in the Toolpaht display panel in the main screen.

A different color can be specified depending on weather the spindle was on or off
when the path was cut.

The width of the path in the Toolpath display panel, in \emph{pixels}, can be specified 
in the 'Width:' -entry field. If this is set to zero then the actual tool width from the
tool table for the tool selected by the G-code program is used in displayin the path.

\subsection{Planned Path -settings}
\label{subsec:planned-path-color-settings}


The settings in this panel control how the tool path for the planned i.e. yet to be cut
path is displayed in the toolpath display panel in the main screen.

A different color can be specified depending on weather the spindle will be on or off
when the path will be cut.

The width of the path in the Toolpath display panel, in \emph{pixels}, can be specified 
in the 'Width:' -entry field.

\subsection{Tool Display -settings}
\label{subsec:tool-color-settings}

The color of the tool in the Toolpath display can be spefied as well as the size of 
it by entering the desired tool width , in \emph{mm or inch}, into the 'Width:' -entry field.

If the tool width is set to zero then the actual tool width from the
tool table for the tool selected by the G-code program is used in displayin the path.

\subsection{Axis Display -settings}
\label{subsec:axis-color-settings}

The size and colors used to display the  axis grids in the Toolpath display can be spefied here.

The main purpose of the grids is to allow you check your toolpath against the current
working coordinates to ensure that the toolpath and coordinates are set up as you
want.


\subsection{Spacing -entry field}

This entry field determines the grid line spacing.

\subsection{Width -entry field}

This entry field is used to set the line width of the grid lines.

\subsection{Grid -popup menu}

The size of the grid can be set manually or automatically. In the automatic mode
the grid size is determined by the toolpath and is automatically calculated to
just encompass the complete toolpath. 

\subsection{Min/Max -entry fields}

These entry fields control from which coordinate (min) to which coordinate (max) the grid 
extends for each axis.

\subsection{Posision -entry fields}

These entry fields control at which position each plane crosses third coordinate axis.


\subsection{Color -button}

These buttons are used to set color for each grid.

\section{Setting up Inputs and Outputs} 
\label{sec:configuring-io}

For simplicity most of the I/O on the TOAD board has a fixed purpose.

The polarity of some of the inputs can be configured. 

To configure the polarity of the home REFIN inputs go to the Axis Setup screen.

\subsection{Probe Input -panel}

The polarity of the PROBE input can be configured in here.

\subsubsection{Polarity -popup}

Select here weather the PROBE signal is zero or one when the probe is activated 
(i.e. when the probe deflected when it touches the work piece).

\subsubsection{Probe -LED}

This 'LED' displays in real-time if the probe is activated or not.

To test that your probe is functional and correctly configured, first ensure that you 
are in the MACH mode. Then just connect the probe
to the PROBE INPUT and trip the probe. The PROBE 'LED' on the screen should lit green
when the probe is tripped, if it doesn't the just change the polarity.
Naturally if the PROBE 'LED' does not change state when you trip the probe then you
either have a hardware problem or you are not in the MACH mode.

\subsection{Spindle Speed Calibration -panel}

The spindle in a milling machine is  driven by Variable Frequency Drive
(VFD) that synthesizes three phase AC voltage frequency of which is 
controlled by a voltage.

The SPEED output of the TOAD board is meant to be connected to that control
voltage input in the VFD.

Whenever the spindle has been turned ON with M3 or M4 C-code TOAD will
output a voltage from the SPEED output proportional to the S-word value
and the voltage that is feeding the SPEED output circuitry.

Because the Digital to Analog Conversion (DAC) of a TOAD board is not
linear starting from zero voltage, it needs to be calibrated.

To calibrate the conversion you need to measure the spindle rotation
rpm speed at two different DAC setting.

I recommend using 10\% and 90\% DAC settings.

To obtain the RPM you of course need a tachometer to measure the actual spindle
speed and you need to set the DAC to those two values using the Test screen,
see section \ref{subsec:dac-field}.

So go ahead turn on the spindle  and set the DAC to the two values and note down what the tachometer indicates.

Then all you have to do is to enter those rpm and DAC values into the 
corresponding fields in this panel.

That is it!

If you are curious this is how the math works but you do not need to care.

The output voltage is relative to the voltage fed into the +10V IN input to the 
TOAD board. The intention is that you get that voltage 
from the VFD from which typically provides this voltage just for this purpose.

The output voltage is calculated as

\begin{equation}
Uout = (\frac{Sword - LoRPM}{HiRPM - LoRPM}*(HiDAC-LoDAC)+LoDAC)/100*Uref
\end{equation}

where 

Sword = \textrm{The S-word value from the G-code program}

Uout = \textrm{'SPEED' output voltage in TOAD}

Uref = \textrm{'+10V IN' input voltage in TOAD4}

LoRPM = \textrm{Value entered into the 'Low ' entry field}

HiRPM = \textrm{Value entered into the 'High ' entry field}

LoDAC = \textrm{Value entered into the 'Low @DAC' entry field}

HiDAC = \textrm{Value entered into the 'High @DAC' entry field}


If above should result in a value outside the range 0..100\% of Uref then
it is clamped to that range.

\subsection{Spindle Speed Limits -panel}

In this panel you should enter the minimum and maximum RPM 
values want to allow for the spindle.

An S-word in G-code that is outside of that range will cause an error message
to be dispalayed and to protect the spindle EazyCNC will not execute a G-code file which
violetes these limits.



\section{Configuring Motors and Axes} \label{axis-configuration}

To configure the motors and axes go to the 'Axis Setup' screen, 
see \screencaps{axis-setup-view-axis-X}{0.5}{Axis Setup screen}

On the top of the 'Axis Setup' panel you see six   or four  [TOAD4] buttons. By clicking at 
those buttons you control which axis parameters are shown
on the panel.

There are three group of parameters for each motor and axis plus a panel for 
testing your motor and mechanics.

Also here you can configure the axis DRO to automatically zero when you
start a machining run.



\subsection{Motor Config -panel}

There are five parameters for each motor.

\subsubsection{Axis On -checkbox}

Axis On -parameter controls weather the G-code commands control
that axis/motor or not. If the motor is not controlled by the
G-code (Axis On checkbox is not 'ticked') then that motor/axis is
available for manual jogging during machining or it can be
controlled with EazyCNC plugin extensions. 

Typically you want to use G-code to control the motors so
make sure the Axis On checkbox is ticked, but if an axis
is not used (say you only have a three axis set up) then 
un-tick the box so that you do not need to set up the
motor properly.

\subsubsection{Current -popup menu}

TOAD4 supports two different drive currents for each motor,
named High and Low, in addition to which the current can be totally
off. The actual motor current depends on the current measurement
resistors mounted to the TOAD4 board and the jumper settings
on the TOAD4 board, see TOAD4 Hardware Manual for details.

TOAD5 does not support current control, the motor drivers
are either enabled via the ENABLE signal implying 100\%
current, as determined by the driver module used, or not
enabled implying zero current.

The Current -popup menu controls how the three different currents
are used when driving the motors.

There are four different options.

Low -- with this setting the motor current is always set to Low. 
You might want to use this if the High current is too much for 
your motor.

High -- with this setting the motor current is always set to High. This 
provides the most 'stiff' setup but means that motors will have full
full current applied and 'run' hot.

Auto Off -- with this setting when the motor/axis is moving or the G-code 
program is being executed the current will be set to High, but once 
the movement or machining stops the motor current will be turned off 
completely within two seconds.

Depending on the mechanics and usage this may not be ideal as the 
motors may move under external forces if there is no current and thus
the axis may lose its accurate position.

Auto Hold -- with this setting when the motor/axis is moving or a G-code 
program is being executed the current will be set to High, but once the movement or
machining stops the motor current will be set to Low within two seconds.

This is often the most desirable motor current setting as full current
and force is used during machining but the current and heat is reduced
when the motors are not being used.

\subsubsection{Forward -popup menu}

The Forward popup menu controls weather the direction output on
the TOAD4 board is 1 or 0 when the motor is driven forward.

Forward means that the coordinates of the axis are increasing.

There are two options.

Output = 0 -- with this setting the (internal to TOAD4 board)
DIR signal is set to logic zero to when the axis/motor
is driven forward.

Output = 1 -- with this setting the (internal to TOAD4 board)
DIR signal is set to one zero to when the axis/motor
is driven forward.

You need not to care about zeros or ones, just make sure this
setting is right! When you press the axis jog buttons (+X,+Y,+Z or +4) 
the motor should be running in the direction that you have designated
as the increasing coordinate for that axis.

If the motor runs in the wrong direction just change the setting
in this popup.

\subsubsection{Home -popup menu}

TOAD4 supports one home/reference position switch input for each axis. 

The Home popup menu controls weather the REF input on
the TOAD4 board is 1 or 0 when home/reference is switch is
active. 

There are three different options.

None -- with this option the REF input is ignored and when
you press the HOME button no movement happens, only the
DRO for that axis is reset.

Input = 0 -- with this setting EazyCNC expects that the
REF input is a logical zero (closed) when
the reference switch is active.

Input = 1 -- with this setting EazyCNC expects that the
REF input is a logical one (open) when
the reference switch is active.

Again you should not care if the signal is active or non-active,
zero or one, just make sure it works for you. If, when you press 
the 'HOME' button, the axis does not begin
to move towards the reference switch the setting of this
input is wrong. Note that you should first ensure that
DIR signal is correctly configures, see previous section.

When you press the 'HOME' button for an axis EazyCNC will drive that 
axis until it finds the home/reference position at which point
that axis DRO is automatically reset. 

The way this works when you press the 'HOME' button 
is that if the REF input is active the axis is driven to
the positive axis direction until the signal becomes non-active. If the
signal is non-active to begin with then the axis is driven 
in the negative direction until the REF signal becomes active and
then to the positive direction until it becomes non-active again.

This ensures that even though there is some backlash in the mechanism
or hysteresis in the switch the mechanism position will always be
correct.

You do not need to use a reference switch but by having one for
each axis allows the system to know its absolute physical position 
which in turn makes it possible for EazyCNC
to guard the movements against the physical limits of your system
preventing crashes.

Using reference switches it is also possible to continue machining
after a sudden loss of power because the absolute axis positions
can be re-covered by homing the axes.

\emph{\color{red} Note that the home or reference switch is no substitute for limit
switches that should be installed at each end of the movements
and wired to act on the emergency stop system}. 

The optimal placement for a reference switch is around the
middle of the axis travel, but this requires that the switch
is so configured that the REF signal is always on or off 
depending on which side of the switch the 'axis' is; it 
should not be possible to drive the axis 'beyond' the switch.

\subsubsection{Home polarity -popup menu}

TOAD supports one home/reference position switch input for each axis. 

The Home popup menu controls weather the REF input on
the TOAD board is 1 or 0 when home/reference is switch is
active. 

Note that if you are using two motors to control a single axis,
ie have the 'Slave Motor' defined for an axis then this polarity
setup applies to both.

\subsubsection{Home offset -entry field}

This entry field controls to which absolute machine coordinates
the axis will be re-set when you press the HOME button.

The absolute machine coordinates are more or less irrelevant
except for specifying the axis movement limits.

Limits of course only make sense and should only be used
if/when an axis is equipped with a Home/Ref switch.

Typically you set the limits so that the Min limit is zero
and the Max limit is the total allowable movement for that
axis.

For example, say your total X-axis movement is 500 mm. So you
set Min=0, Max=500 in the Axis Limits panel. Further suppose
your limit switch is set to activate in the exact midle of that 
range so you set the Home Offset=250.



\subsubsection{Steps/unit -entry}

This entry fields tells EazyCNC how many steps it takes
to move the axis a unit (mm or inch or degrees) amount. We call
this value the step ratio.

Note that this is not an integral value and you should enter it
with as many significant digits as required  to achieve
the desired accuracy. As rule of thumb use at least six 
significant digits in calculations and entry to achieve 0.01 mm 
accuracy over 1000 mm axis movement range. Note that you can
enter more digits that what the entry field will display.

To calculate this value you need to know following:

\begin{itemize}
\item $mode$, a factor dependent on step mode
\item $steps$, the number of steps per revolution for the motor
\item $pitch$, the axis movement per motor revolution (including possible gearing)
\end{itemize}

Then you calculate the step ratio as follows:

\begin{equation}
\label{step-ratio} 
step_{ratio} =  \frac{mode *steps}{pitch}
\end{equation}

The step mode depends on the jumpers for each motor on the TOAD4 board. 
With TOAD5 you need to look this up from the documentation of the
driver modules you use.


The step mode factor tells how many STEP pulses it takes to make
the stepper motor to take one full step.

So for full step mode the mode factor is 1, for half step it is 2 and so on.

Typically you want to run the motors with as small step size as possible
for  smoother ride, but sometimes speed requirements dictate that a 
coarser step size is needed because the max STEP output frequency is fixed.

The actual maximum motor rpm (rotation per minute) goes down in direct 
relation to the step mode factor.

The maximum STEP output frequency is 100 kHz, to keep the jitter of
that signal acceptable I recommend to keep the step rate below 10\% of
that frequency.

Supposing a 200  full steps / revolution motor, this means that the 
 maximum rotation speed of the motor 
is 100.000 kHz * 10\% / 200 steps/rev = 50 rev/ sec = 3000 rpm.
With quad steps this is reduced to 750 rpm or 12.5 rev/sec, which
is still a high speed for a stepper motor, most likely not achievable
in practice.



This may feel a bit complicated so an example maybe useful. 

Most stepper motors have 200 steps or step angle of 1.8\degree so we have:

\begin{equation*}
steps = 200 (steps/rev)
\end{equation*}

For this example we assume that we want to run X-axis motor at 'Half Step' 
so we set jumpers on the hardware to that. 

A half step this implies that two STEP pulses are needed for a full step, so we have:


\begin{equation*}
mode = 2
\end{equation*}

To make this more interesting and life-like let's suppose we use a lead screw
to move the X-axis and the screw has a pitch of 3 mm/revolution and that
we use a toothed belt to drive it with a 16 tooth
pulley on the motor axis and 45 tooth pulley on the lead screw, so we have:

\begin{equation*}
pitch = 3 * \frac{16}{45} =  1.066666 (mm/rev)
\end{equation*}

Putting it all together we have

\begin{equation*}
step_{ratio} =  \frac{2 * 200}{ 1.0666666} = 375.000 (steps/mm)
\end{equation*}

Now would be good time to check how fast we can move the X-axis.

As stated the maximum theoretical step rate is about 100 kHz, for jitter
and other reasons the maximum recommended pulse rate is about
$\frac{1}{10}$ of that, say 10000 pulses/sec. You need to divide
this by the $mode$ factor we looked up above so in our example
the maximum step rate is 10000 steps/sec and so our max speed
is

\begin{equation*}
\frac{10000}{375} \approx 26 mm/sec
\end{equation*}



\begin{framed}
Here is a Top Tip!

Even though EazyCNC guides you if you try to enter too big (or small) value
to an entry field and even tells you what the limiting parameter is
and  further tells you what is the maximum value you can use, sometimes this can
turn into a bit of a chore.

So before you enter the step ratio, set the maximum jog acceleration and velocity
for all axes (Mach Setup / Axis Setup) and movement (Mach Setup / Movement) 
to a very small values, say 1 mm/sec or 0.01 inch/sec.
This will allow you to enter almost any step ratio, which is necessary as
the step ratio is a function of your gearing and thus won't budge.

Once you have entered the correct step ratios try to set the accelerations
and velocities to a very large values, say 1000 mm/sec or 10 inch/sec and EazyCNC will tell the
maximum possible values, so enter and use those.

\end{framed}

\subsubsection{Scaler -entry field}

This entry field is used to specify a scaling factor for each axis which is used when 
the tool path planner code in the software plans the tool moves.

Normally you set this to 1.0 in which case an F-word in your G-code specifies
the feedrate in mm (or inch) per minute for XYZ axes and degrees per minute for 
the rotating axes ABC.

However, sometimes you may want to change that behaviour.

I'll give you two examples.

Example 1

You have 'knee-mill' or a plasma cutter in which the Z-axis cannot be run at the
same speed as X and Y axis. Instead of using the F-word to 
slow down movement when ever you move the Z-axis you can use
thew scaler. 

If the Z-axis can only be moved at half the speed of XY axis you
set the scaler for Z axis to  1 / 0.5 => 2. This will in effect
cause the Z-feed rate to be scaled down by a factor of 2.

Example 2

Suppose you are controlling a mill where the A axis is
rotating a cylindrical workpiece of 100 mm in diameter and you
are working on the perimeter. To maintain the correct cut rate 
even when you are using the rotation you need to use the scaler.

In this example case a scale factor of 1 for A axis and F word of
1 would give you a feed rate of 1 degrees per minute which is in
effect 100 mm * pi * 1 / 360 = .872 mm/minute which is obviously wrong.


To correct that you would need to set the scale factor for A axis
to 100 * pi / 360 = 0.872 .



\subsubsection{Master Motor -popup menu }

EazyCNC allows you to select for each axis which physical motor is used to drive the axis.

You select that with this popup menu.

On the TOAD4 PCB the motors are number like this: X=0, Y=1, Z=2 and A=3.
On the TOAD5 PCB the motors are number like this: X=0, Y=1, Z=2, A=3 and B=4.

Typically you don't want to change this to avoid confusion.

Below this pop up there is 'LED' that indicates whether the home/ref switch for selected master motor
is active or not. 

If you use a home/ref switch in your system you can use this 'LED' to diagnose and
ensure that the switch works and is active at the lower (left,near,bottom) end of its movement.
To do that move for example the X-axis to the extreme left and ensure that the
'LED' turns ON, then as you move the X-axis to the right, the 'LED' should turn OFF when
the switch is no longer active. 

If the polarity is wrong, change it with the 'Home polarity' -popup.


\subsubsection{Slave Motor -popup menu }

On some mechanical setups paralleling two motors are handy and allow doubling of the motive power without using
larger motors.

EazyCNC allows you to select for each axis a second physical motor that is driven when that axis moves.

You select that with this popup. 

Below this pop up there is 'LED' that indicates whether the home/ref switch for selected slave motor
is active or not. 

When motors are paralled they perform the exact same movements, except when you press the HOME button. 
Homing is performed independently for each motor so that the system will align itself correctly based
on the home/ref switches.

Note that when motors are paralled then all the settings for both motors are taken the
same motor setup i.e. axis so the gearing, motors and switch setups need to be identical.






\subsection{Axis Limits -panel}

EazyCNC can guard movements against
set limits to prevent crashing the mechanism.

This is especially useful in Jogging where
it is too easy to run too fast to an end of
an axis.

However the limit checking is not fool proof
and it depends on the operators (that's you!)
diligence to work properly. If you for example
put a large cutter into the spindle chuck but
don't tell EazyCNC about it or move the axis manually
by turning the axes handles or forgot to 'home' the axes
there is nothing EazyCNC can do about it.


A 'bad'  G-code move may still crash the machine
and you need to visually satisfy yourself using
the simulation mode that this will not happen.

Also worth remembering is that the limits checking
does nothing to prevent crashing against the
workpiece, fixtures or other obstacles.

The actual movement limits are based a fixed
coordinate system independent of all the different G-code
coordinate systems. The limits are expressed in
the current unit system unscaled and unaffected by
any G-code coordinate system transformations.


The origin of the limits coordinate system is at 
the home/ref switch position, so if the home/ref switch 
is not used you should not enable the limits because
the position is physically undefined. 

If you want to use the limits you must remember to 'home'
all axes by pressing the 'HOME' buttons if the 
TOAD4 has been power cycled (it is TOAD4 who maintains the 
coordinates so it will loose track of the position
if it is turned off).

Also worth remembering is that if you use the limits
you should have the motors energized at all times
otherwise the motors will 'lose' their positions,
so don't use the 'Auto Off' current mode.

\subsubsection{Limits Enabled -checkbox}
\label{subsubsec:enable-limist}

Limits Enabled -checkbox controls weather EazyCNC
enforces the limits for the axes or not.

\subsubsection{Min/Max - entry fields / Touch -buttons}

These entry field controls the minimum and maximum 
coordinates allowed for the axis, expressed in the 
current unit system. 

The easiest way to set the limit is to disable the
limit checking and carefully 'jog' the mechanisms
to each end of the movement and press the corresponding
'Touch' button which will then set corresponding
limit based on the current axis position.

Remember to 'home' the axis before setting the
limits and don't forget to enable the limits once you have set them.

And don't forget to verify your limits!

\subsection{Safe Z -panel}

This panel only appears on the Z-Axis setup sub panel.

The 'Safe Z' feature allows you to define a Z coordinate that
to which you can move the Z-axis by pressing the Safe Z button
in the jog controls.

The idea is that you set this so high that it will always be
safe to move in XY at that Z setting without hitting anything.

This can be handy for example for changing tools/cutters.

This feature only works correctly if a HOME reference switch is
installed and the Z-axis is HOMEd before you use the feature.

\subsubsection{Safe Z - entry field}

In this entry field you can enter the safe Z value.

This value is in absolute machine coordinates, just like the
axis limits. Remember that the absolute machine coordinates
are defined by where you HOME-switch is located and 
the 'Home offset' for that axis.

As a safety measure the initial value of this field is 'NaN' 
which signals to both you and the software that the Safe Z
coordinate has not been set.


\subsection{Jogging -panel} \label{jogging-panel}

Jogging refers to manually moving the axes with
either the 'jog buttons' or with the joystick.

The settings in this panel control the details
of jogging.

When you 'jog' an axis, the axis first moves
slowly i.e. crawls. This goes on for some
short time after which the movement accelerates
until it reaches the jog speed. Jogging then
continues at that speed as long as you keep
jogging or you hit the end of movement, after
which the speed decelerates to a halt.

In many CNC mechanisms the axes are not created
equal, some motors are by necessity stronger
than others
and thus the desired jogging characteristics
are different and you want to set them individually
for each axis.




\subsubsection{Crawl Velocity -entry field}
\label{sec:crawl-veloc}  

This entry field controls the initial ie
crawl speed of jogging, you typically want
to have this pretty slow.

\subsubsection{Crawl Length -entry field}

This entry field controls how long a distance
the crawl will go  until the acceleration
starts if you keep on jogging.


\subsubsection{Min Crawl -entry field} \label{crawl-velocity}
\label{sec:min-crawl}

This entry field sets a minimum crawl distance,
ie the axis will always move at least this much
even if you jog very briefly.

\subsubsection{Acceleration -entry field}
\label{sec:accel}

This entry field control the acceleration/deceleration rate
of the movement, you probably want to have this as high
as possible but not so high that there is a risk
of stalling the motor or the motor skipping steps. 

Only experimentation can find a correct value for this,
start with a small value and first find the maximum
jog speed  before you try to maximize 
the acceleration.

EazyCNC has a motor test function, see  
section~\ref{subsec:test-panel}, to help you to 
determine the limits of your motors and mechanisms.

\subsubsection{Jog Velocity -entry field} 
\label{jog-velocity}
\label{sec:jog-veloc}


This entry field controls the jog or maximum speed
the axis will run when you jog it. You probably want
to have this set as high as possible but not so high 
that there is a risk of stalling the motor or the 
motor skipping steps. 

Only experimentation can find a correct value for this,
start with a small value and find the maximum at which
you can jog back and forth to a given DRO value 
watching that the tool tip hits exact same position
every time.


A manual way to check that the motor has not lost any
steps is to 'home' the axis and see that as the 
axis approaches the home position the DRO will not
suddenly jump to the reset value when the reference
switch is reached, this is not totally accurate if the
motor only loses a small number of steps but typically
it is a all or nothing with stepper motors.

\section{Setting the Motion limits}
\label{sec:setup-movement-limits}

The parameters in the 
\screencaps{mach-setup-view-Motion}{0.5}{The Motion Limits screen.}
 screen 
tell EazyCNC how fast
it is safe to accelerate and run the motors, how often
the motor position and speed should be updated 
and how accurately you want EazyCNC to follow
the tool path described by the G-code file.

Next to the individual axis/motor parameters these
are the most import parameters to carefully set as
if you set the speed and acceleration to too high values
the steppers will lose steps and the accuracy 
is ruined or the motors will completely stall.

On the other hand you will want to have the values
as high as reasonable so as not to waste time in
machining and with some cutters like plasma torches 
even the dimensions and quality of cut are dependent on
high enough speeds.

Note that these settings here set the maximum 
values, G-code programs specify the actual value
which can be lower or higher that what you specify
here. If the G-code specifies a lower value then that
applies but if the G-code specifies a higher value
then what you have set up here applies.

The only way to find out the maximum acceptable value
you can use is to try progressively higher
values and verify the accuracy of the 
motions for each trial.

Note that the values here are common to all axes. 
See section~\ref{subsec:test-panel}) on how to do this for
each axis. Once you have found out the maximum velocity
and acceleration for each axis, pick the minimum of those 
values and use that here as the maximum velocity and acceleration
for machining.


\subsection{Velocity -entry field}

Enter the minimum of the maximum velocities your motors
can handle.

Note that it is acceptable to use/have too high feed rate 
in G-code (the F-word) as EazyCNC will automatically
limit the feed rate to the maximum velocity you have
set here.

\subsection{Acceleration -entry field}

Enter the minimum of the maximum accelerations your motors
can handle.

\subsection{Path tolerance -entry field}
\label{sec:path-tolerance-entry-field}

This field tells EazyCNC how accurately during machining 
it should try to  follow the tool path described by the G-code 
program.

If you enter a value of zero here then the tool path is
accurately reproduced but this necessitates that the tool
comes to a complete stop between cutting movements (G-codes 
G1,G2 and G3).

This is because to have the tool
follow the path absolutely without stopping at corners would require infinite
acceleration and that strong motors are hard to find.

If you enter a non-zero value then EazyCNC will try to
follow the prescribed tool path to within the specified 
limit but using the leeway given by the tolerance to allow continuous movement
and not stopping between cuts. 

Rapid tool positioning (G0) always stops at the end of
the movement so this parameter does not apply.

If you use EazyCNC with a plasma torch  it is important to try to 
minimize the speed variations as the cut width depends
on the travel speed of the torch so use as large path
tolerance you can accept.

\subsection{Z-scaler -entry field}

Sometimes the Z-axis motor is different from the X and Y axis motors,
for example in a plasma cutting machine the X/Y movements need high acceleration
and velocities but the Z-movement is rather small and thus a smaller motor
that is not capable of such feats can be used. To prevent the system
from exceeding the Z-motor capabilities a Z-scaler value (smaller than one)
should be entered. This will effectively scale down the accelerations and
velocities for G-code movements that involve the Z-axis.

\subsection{Update Period -entry field}

Setting a suitable value for the Update Period is also
critical.

This value depends on the speed your computer. A faster
computer allows for a faster update period, however
there are limits on how fast it is acceptable to
update the speed and position into TOAD4.

USB limits the update period to a minimum value (maximum
update speed) of 1 msec, but typically you should aim
to a value of 10 to 20 msec on a modern PC hardware. Note
that this has nothing to do with step rate because
we are transferring position values to the TOAD4 and
the actual steps are generated on the TOAD4 board.

To understand how the update period affects things here
is brief description.

TOAD4 maintains a queue of movement commands so that
small pauses, interruptions or hiccups in the computer 
won't affect cutting movements. 

If TOAD4 runs out of movements commands ie the queues
run empty which happens if the computer does not
send new commands fast enough on average then the cutting 
movements will stop until more commands arrive.

At best this is not desirable as an interrupted cut can leave
a mark in the workpiece and  at worst the accuracy may be
lost if the movement was at such a high feed rate that
that the motor cannot be accurately stopped from such
speed.

The queue capacity is 16 commands, so an update period
of 20 msec means that there are commands for 16 x 20 msec
or 320 msec and the system can tolerate a pause, such
as Java garbage collection, for that length of time. So
longer period allows for longer pauses and hiccups in
the host computer system.

On the other hand TOAD4 can only change speed at the
interval of the update period so when the speed is
changing, like when the machine is cutting a circular
path, the speed is always partially 'wrong' for the duration of
the update period. 

So longer Update Period will result in an increased positional error,
which fortunately is not accumulative.

The upper theoretical
limit for such an error is  'feed rate * update period',
for example if you are cutting at 1200 mm/min which is 20 mm/sec
and your update period is 20 msec then the maximum error
caused by the update period is 20 mm/sec x 0.02 sec which
is 0.4 mm. This would probably be un-acceptable for milling
but such high feed rates when milling are rare and for plasma
cutting where such high feed rates are common this is in
the same ballpark as cutting accuracy anyway. Besides that is a worst
case error unlikely to happen in practice.


\section{Configuring  Options}
\label{sec:g-code-options}

While the design goal of EazyCNC is to reduce complexity by limiting options there
still are a number options that you may change.

You can change them in the Options screen, 
see \screencaps{mach-setup-view-G-code}{0.5}{The Options screen.}

\subsection{G-code options -panel}

G-code dates back to 1960s with the final revision RS274D approved in 1980. 
Over the years manufacturers of CNC system have added extension and variations to the standard.

EazyCNC tries to accommodate a common subset of the most popular systems
out there by allowing you to fine tune some details of the G-code interpretation to match your G-code program.







For a complete description of G-codes supported by EazyCNC see Chapter~\ref{chap:g-code-chapter}.

 

\subsubsection{Incremental IJK -checkbox}
\label{sec:incremental-ijk-checkbox}

If this check box is ticked then the I,J and K words in the arc cutting G2 and G3 
commands are interpreted relative to the start point of the arc.

If you see a lot of large erroneous arcs in te tool path graphics panel when 
your are previewing your G-codes then you can be pretty confident that this
tick box is in the wrong state.

\subsubsection{Incremental XYZ -checkbox}

If this check box is ticked then the X,Y and Z words in the movement commands
G0,G1,G2 and G3 are interpreted relative to the previous X,Y or Z words/positions.

\subsubsection{G4 P in msec -checkbox}

If this check box is ticked then the P-word value in the G4 dwell command
is interpreted in milliseconds instead of seconds.

If the execution of G-codes seems to stop at G4 commands you can be pretty sure
that this check box is not ticked but should be.

\subsection{G41/G42 code options -panel}

When tool compensation is turned on with either G41 or G42 G-codes the question arises 
how the tool should move in external corners.

Traditionally the tool moves around the corner in an arc.

The other option is for
the tool to move in a straigh line until it has 'cleared' the corner and the move
in a straight line along the next segment. If the (external) corner is very tight
then this would cause the tool to move very long away beyond the corner point, 
therefore a maximum length can be specified.

This later option maybe useful in plasma cutting as it takes the cutting flame further
away from the potentially sharp and narrow corner which tends to burn if the torch linger
around the corner too long.

\subsubsection{Use round join -radiobutton}

If this is selected then the traditional way of handling external corners is used, i.e.
the tool moves in an arc around the corner.

\subsubsection{Use miter join -radiobutton}

If this is selected then tool will move  in a straigh line until it has 'cleared' the corner and the move
in a straight line along the next segment. If, as a result of the corner geometry,
 the tool would move further from the corner than what
is specified in the 'bevel limit' -entry field, then the move is pruned as indicated.

subsection{Auto functions -panel}
LIP
\subsubsection{Auto SPINDLE OFF -checkbox}
If this checkbox is ticked then the spindle will automatically turn off when the system
enters HOLD or STOP states. This very useful for plasma cutting, not so much for 
milling because you will forget to turn the spindle back on when go from HOLD to RUN
and break a cutter.

\subsubsection{Auto COOLANT OFF -checkbox}

If this checkbox is ticked the coolant is turned off when the system enters HOLD or
STOP states. This can useful for milling when you want to pause the milling for 
observing work or changing the cutter.

\subsection{ZERO DRO on RUN -panel}

\subsubsection{N-axis auto zero -checkboxes}


If 'N-axis auto zero'  checkbox  is ticked then N axis DRO will be automatically
zeroed when you hit the RUN button if the machine was in STOPed state.

The DRO will NOT be zeroed when you hit the RUN button if the system
was in HOLD state.

This is handy for plasma cutting where you typically just jog the 
torch to the origin position, zero the DROs and hit RUN. By ticking
this box this becomes automatic and you cannot forget to zero the DROs.

\subsection{Export (Import)settings -panel}


\subsubsection{File format -popup menu}

The Tool Setup screen allows you to save and load the tool setup to/from a text file, 
see section \ref{sec:tool-setup}

With this popup you select the format in which tool setup is stored in the text file. 
Currently only CSV (Comma Separated Values) format is supported.

\subsubsection{Number format -entry field}

Into this entry field you enter a 'sample' of how the numbers should be
formatted. For example if you want to have three decimals
use '0.000' .

\subsubsection{List item Separator -popup menu}

Unfortunately CSV format does not specify whether the decimal separator is 
'.' (period) or ',' (comma). Worse than that, you probably want to edit
the text files in Excel which defaults to different separator depending
on which language version of the OS you have. 

As mentioned the ',' (comma) can be both desimal separator and list item
separator, but it should never be used for both purposes at the same time.

So in this popup you can select which (comma ',' or semicolon ';') character
is used as the list item separator, this changes the list decimal separator. 
If list separator is ',' then decimal separator is ',' and if list separator
is ';' then decimal separator is ','.

\subsection{Shutdown -panel}

\subsubsection{Shutdown on Quit -checkbox}

 A clean shutdown is important,
especially on Raspberry Pi running from SD-card, but without
a mouse or keyboard this is not possible.

The purpose of this option is to allow operation without a 
mouse or keyboard by enabling automatic system shutdown.

If this checkbox is ticked then, when press the Quit button 
EazyCNC will attempt to execute a shutdown script if one
exists. If there is no shutdown script then EazyCNC will
try to execute the system shutdown command. 

A shutdown script is a script named shutdown.BAT on Windows
and shutdown.sh on  macOS/Linux and which is placed in the
EazyCNC directory in your home directory.
If one exists EazyCNC will try to execute it using the operating system shell.

The purpose of the shutdown script is to allow you to customise the
shutdown process e.g. by shutting down additional hardware in 
you machining system.

\subsection{On-Screen Keyboard -panel}

\subsubsection{Enable Keyboard -checkbox}
If this checkbox is ticked then an on-screen virtual keyboard will
appear whenever you click an entry field. The on-screen
virtual keyboard allows operation with touchscreen only without
a real physical keyboard. This built in virtual keyboards is preferable
to the operating system provided because it is optimsed For
CNC operations and it plays nicely with the rest of the user interface.



\subsubsection{Sound on Click -checkbox}

If this checkbox is ticked then clicking any of the
buttons in the user interface will emit a click sound.

In addition, on Raspberry Pi a short pulse is generated
on the GPIO21 at the same time the click is emitted. 
This is so that you can connect a 
tactile feedback device to the GPIO 21 output to give a physical
feedback on the touchscreen. A simple tactile feedback 
device can be fashioned from a solenoid.



\section{Shortcuts setup}
\label{sec:shortcut-setup}


EazyCNC is really designed to be used on a PC or tablet computer with a touch screen. 
However you can use most of the functions with convenient single key
keyboard shortcuts on a conventional keyboard or with a gamepad function keys
and joystick.

The key assignments are fully user definable so you can configure these as
you best please.

To examine or change the key assignments go to the Shortcuts setup screen,
see \screencaps{mach-setup-view-Shortcuts}{0.5}{The Shortcuts setup screen.}


On the left side column you see the keys and on the right side column the
corresponding assigned functionality.

To change the key, click on the left side column at the key you want to change; 
this will popup a dialog prompting you to press the new key you want to
assign for that functionality.

To change the functionality click on the right side column of the key
assignment you want to change and from the popup pick the functionality
you want to assign.

To create a totally new keyboard shortcut scroll to the bottom of
the list and click at the 'New Shortcut' button at the bottom of the
left column.

To delete a shortcut click on the left column on the shortcut you
want to delete and from the dialog that pops up select 'Delete',
see \screencap{mach-setup-shortcuts-view-delete}{The Redefine Shortcut screen.}

\section{Info screen}

This screen displays bunch of system related information, 
see \screencaps{mach-setup-view-Info}{0.5}{The System Info screen}

Most of the information displayed is just to report back
to eazycnc@eazycnc.com if you are reporting a bug.

The two pieces of information that are useful for you
are the 'EazyCNC version' to identify which
version you are running (it is also shown in the window title
if that is visible) and 'EazyCNC parameter' which tells you
the location of the configuration file in case you have doubts
about which configuration file is being used.
-------------------------------------------------------------------------


%\section{Diagnostics screen}

%In addition to displaying a bunch of system related information 
%the Diagnostics screen, Figure~\ref{fig:mach-setup-diagnostics-view}, can be
%used to get insight into how consistently EazyCNC is able to communicate
%with the TOAD4 controller.

%When the controller is connected (the 'MACH' button is active) the histogram
%shows in real time the distribution of actual update periods.

%The middle of the histogram represents the current 'Update Period' so
%the access times should consistently pile up slightly to the right
%of that. When EazyCNC is executing the G-code program the update speed
%is actually double so the pile should be concentrated on the left side
%of the histogram.

%If the update times show up all over the place then there is something
%in your computer setup disturbing EazyCNC, possibly an other program
%or process running in the background.

%As time goes by the histogram represents a longer and longer history
%and thus new information makes little difference to it, that is why
%you should clear the histogram by pressing the 'Reset' button to
%get fresh look at the distribution of update periods.

%---------------------------------------------------------------------------
%\begin{figure}[htb]
%\includegraphics[scale=0.35]{mach-setup-diagnostics-view.png}
%\caption{The Diagnostics screen}
%\label{fig:mach-setup-diagnostics-view}
%\end{figure}
%---------------------------------------------------------------------------



\section{Test screen}
\label{sec:test-setup}


The Test screen in
\screencaps{mach-setup-view-Test}{0.5}{The Test screen.}
allows you to test all the inputs and output of your TOAD boars as well
as test or find out the limits of your motors.

To use this panel the system needs to be in MACH state so that it is communicating
with the TOAD board (this actually works in the SIMU mode as well, but then
everythings is, well, simulated).

\subsection{Inputs -panel}

This panel shows in real-time the state of the all digital inputs from the TOAD board.

The different inputs are self-explanatory except perhaps the MAN-CTRL which
indicates that the manual control panel is attached AND in manual control mode.


\subsection{Outputs -panel}

This panel allows you to manually control all the digital outputs of the TOAD board.

To exercise the output you need to first activate the orange OVERRIDE button. 

\begin{mdframed}[style=warning]
Warning!

Once the OVERRIDE is enabled all safety features associated with any of the outputs 
are disabled and the outputs are solely controlled with the buttons in this panel 
(and the DAC field in the analog I/O panel). 

This means that it is perfectly possible and very easy to turn the spindle or plasma 
torch on or make all the axes run wild if you do not know what you are doing.

\end{mdframed}

For some testing/debugging purposes it is of course necessary to do exactly that,
but for some it is best to disconnect or power down the machine being controlled.

The blue buttons (FORWARD, REVERSE, COOLANT, ARC-START) directly toggle the named 
outputs. When the button is activated the corresponding output is turned on.

Activating the green STEP/DIR button causes the ENABLE output for all axes to
turn on and the  STEP and DIR outputs emit pulses that 
effectively drive the connected axis motor back and forth as 
long as the button is activated.

Proceed with caution.

\subsection{Analog I/O -panel}

\subsubsection{ADC -field}

The ADC (short for Analog to Digital Conversion) field shows the digitized 
value of the analog ARC VOLTAGE input of  TOAD5 in percentages.

0\% indicates 0 Volts and 100\% indicates 5 Volts at the analog input. 

\subsubsection{DAC -field}
\label{subsec:dac-field}

The DAC (short for Digital to Analog Conversion) field can be used to control 
the SPEED output of TOAD5, when the orange OVERRIDE button in OUTPUTS panel is 
activated.

The DAC conversion is ratio metric which means
that the output voltage is always relative to
the voltage used to feed the circuitry (see TOAD5 manual).

Suppose the feed voltage is 10 Volts then setting the DAC field
to 100 (\%) causes the SPEED output to rise to aproximately 10 Volts.

The DAC output is not linear, any non zero value in the DAC field
will cause the SPEED output to be at least 10\% level of feed voltage.
Above that the response is more or less linear.

A zero value in the DAC field should always produce zero voltage at the output.

\subsection{Motor Test -panel}
\label{subsec:motor-test-panel}

The controls in this panel can be used to test and
find out the maximum acceleration and velocity for
an axis.

When you click the 'TEST' button EazyCNC will
perform a test movement on that motor/axis
and report the accuracy if you have a reference
switch installed. 

If you don't have a reference switch then you will
need to use a Dial Test Indicator or some such
to measure the accuracy.

\subsubsection{Pre-requisites}

\emph{Note that this test can damage your machine if not
performed carefully and as intended.}

The test run requires free travel of 25 mm or one inch plus
the distance is takes to accelerate from the start velocity
to the top velocity and back.

To use this test you must have the reference switch located
at least 30 mm away from the negative end of the axis movement.

If you don't have that 30 mm spare travel you run the risk of hitting 
the end of the axis movement range on the return leg of the
test run. It is acceptable to temporarily move the reference
switch to a suitable location for this test or use a temporary switch 
if you so desire.

If you don't have a reference switch then you need to ensure
that you start the test from a position on the axis from which
there is enough free travel on both sides of the starting position.


\subsubsection{The Test run}

When you click the the 'TEST button the test movement is performed as follows.

First EazyCNC 'homes' the  axis/motor ie it moves slowly 
towards the reference switch and then just out 
of it.

EazyCNC makes an internal note of this step position.

Next the axis/motor starts to move into the positive (or negative,
depending on which direction you have selected)
axis direction and accelerates at the given acceleration until
the motor reaches the given test speed. The movement continues
for 25 mm (about 1 inch) and then it
decelerates back to stand still. 

Then the axis is homed again and a note of the
step position at which the reference switch is
detected is noted down.

The difference between the two home positions noted down
is reported as the accuracy or repeatability of
the movement at the given acceleration and 
movement velocity.

A positive value indicates that steps were lost
on the way out i.e. during acceleration or high
speed movement. A negative value indicates that
steps were lost on the way back i.e. slow movement;
this should not really ever occur.

A small non zero value is acceptable or even expected 
as it is unlikely that  the system
can be absolutely accurate all things considered, but repeated tests at
given acceleration and velocity should show consistently
similar values.

To guarantee that the test is valid for machining
conditions the acceleration is performed step wise
at the current machine Update Period just as it
will when take place when executing G-codes. 

Therefore you need to remember that if you change 
the update period it is good to re-run the test
for each axis, especially if you are running close to the speed
and acceleration limits of your system.

%\subsubsection{Start Velocity -entry field}
%
%In this field enter the start velocity for the test, this should
%be a low speed at which the motor is more or less guaranteed
%to work without losing steps or stalling.

\subsubsection{Top Velocity -entry field}

This is the velocity you want to test for so keep
adjusting this and retesting 
until you have maxed out your system.

\subsubsection{Acceleration -entry field}

This is the acceleration you want to test for so keep
adjusting this and retesting 
until you have maxed out your system.


\subsubsection{Direction -popup menu}

This selects weather the test movement direction will be
along the positive direction or negative direction
from home/ref position.

This is the acceleration you want to test for so keep
adjusting this and retesting 
until you have maxed out your system.

