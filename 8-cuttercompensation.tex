\chapter{Cutter compensation}

G-codes specify the cutter or tool movement with coordinates. These
coordinates specify the position of the spindle of the milling machine.
Because the cutter of the milling has a non negligible size they  do \emph{not} 
specify the size and shape of the part you are cutting. It is the job a CAM software 
post processors software to turn CAD models into toolpaths that accommodate for
the diameter and length of the cutter.

However G-codes specify and EazyCNC supports limited automatic cutter length
and diameter compensation so that you can use the \emph{coordinates of the
part} to be machined instead of the coordinates of the spindle.

These can be useful with hand written G-code programs and
with some special machine configurations such as plasma cutters and when
working mainly in plane i.e. two dimensions.

\section{Tool length compensation}

Tool length compensation is rather simple to understand, basically
it just moves the machine higher by the length of the tool. So when the
G-codes specify for example \texttt{Z 200} and the current tool
length compensation is turned on (with \texttt{G43} code) and the
tool length for the current tool (set with the \texttt{T} word)
is 50  the actually Z coordinate the machine moves to is 250.

\section{Cutter diameter compensation}

Cutter diameter compensation is not difficult to understand 
but there are more details to consider.

Basically if you use the diameter compensation you put the
coordinates of the part your are machining in to your G-code
program and when you turn on the compensation you tell
EazyCNC on which side of the path you want the tool travel. 
The side of course depends on weather you are cutting an opening
or an outline of the path and weather you are traveling the
path clockwise or counter clockwise.

The cutter compensation cannot change the past, what is cut is cut,
so when you turn on the compensation (\texttt{G41} for and \texttt{G42})
for right) the compensation will only have an effect the next move.

Therefore you need pay attention to the first move when you
are planning your tool path and you need to pre-position (usually
with \texttt{G0} code) the tool outside of the part by at least
half the tool diameter. Also consider the direction this first 
move-in cut will make, ideally this should be in the same direction
as the first actual part outline to be cut, but for inside holes
or openings this is not always possible.

\subsection{Cutter compensation example - cutting part's outline}

To get a more concrete picture of above please study the 
following example G-code  
and the the Figure~\ref{fig:cutter-compensation-right} which illustrates various aspects of the diameter compensation
process.

\begin{verbatim}
N1 G40      ; make sure diameter compensation is off
N2 G0 X2 Y1 ; move to our starting position
N3 G42 P0.4 ; turn on compensation to the right for 0.8 unit tool diameter
N4 G1 X3 Y2 ; first move, from starting point to part perimeter
N5 G1 X8 Y2 ; cut the lower edge
N6 G1 X8 Y6 ; cut the right side
N7 G1 X3 Y3 ; cut the slanted top edge
N8 G1 X3 Y2 ; and finally back to where we started cutting the left edge
N9 G40      ; compensation is off
\end{verbatim}

Referring to the Figure~\ref{fig:cutter-compensation-right} the black line illustrates
the uncompensated coordinate path i.e. the coordinates in the G-code file.
The green lines illustrates the path of the center of the tool and the compensated
coordinates. And finally the red line illustrates the left edge of the groove
the cutter actually cuts, i.e. this is the actual outline of the part
we are making here, so any deviation between the black and red line (except 
for the initial move-in movement) is a indication of badly constructed 
G-code program.

%---------------------------------------------------------------------------
\begin{figure}[H]
\includegraphics[scale=0.5]{cutter-compensation-right.png}
\caption{G-code path versus compensated cutter path}
\label{fig:cutter-compensation-right}
\end{figure}
%---------------------------------------------------------------------------


Overall the part outline and the cut path seem to match but if you look
carefully in the blown up detail of the beginning of the cut in 
Figure~\ref{fig:cutter-compensation-detail} you see that cutting the
path as specified would leave a semi circular concave notch to the part
and besides, at least in theory, the part would be left hanging by
a thread as the loop is not completed. These problems come from the first two actual
moves (lines \texttt{N2},\texttt{N4} and \texttt{N4}) which are not parallel.

%---------------------------------------------------------------------------
\begin{figure}[H]
\includegraphics[scale=0.2]{cutter-compensation-detail.png}
\caption{Cutter compensation detail}
\label{fig:cutter-compensation-detail}
\end{figure}
%---------------------------------------------------------------------------
\subsection{Cutter compensation example - cutting holes and openings}

To illustrate cutting a hole or cutout to match the above part we can
just set compensation to the left and move the starting point to the inside
of the path. Following G-code does just that, the actual path is identical
to the previous example only the compensation side and starting point
are different. The starting point was
deliberately chosen badly and the cutter is way oversize relative
to the path dimensions to exaggerate the issues you may encounter
when designing your paths.

\begin{verbatim}
N1 G40      ; make sure diameter compensation is off
N2 G0 X5 Y3 ; move to our starting position
N3 G41 P0.4 ; turn on compensation to the left for 0.8 unit tool diameter
N4 G1 X3 Y2 ; first move, from starting point to part perimeter
N5 G1 X8 Y2 ; cut the lower edge
N6 G1 X8 Y6 ; cut the right side
N7 G1 X3 Y3 ; cut the slanted top edge
N8 G1 X3 Y2 ; and finally back to where we started cutting the left edge
N9 G40      ; compensation is off
\end{verbatim}

%---------------------------------------------------------------------------
\begin{figure}[H]
\includegraphics[scale=0.5]{cutter-compensation-left.png}
\caption{G-code path versus compensated cutter path}
\label{fig:cutter-compensation-left}
\end{figure}
%---------------------------------------------------------------------------
