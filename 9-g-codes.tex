

\chapter{G-code reference}\label{chap:g-code-chapter}

G-code is ancient by computer standards, you can trace it's roots
back to MIT labs in the 1950s. Wikipedia describes G-code as
"a language in which people tell computerised machine tools what 
to make and how to make it". 

Today this is not actually a good  description. Rather today 
G-code is the standard by which computer
aided design and manufacturing programs, CAD/CAM software, communicate
the machining instructions to CNC machines.

There is a EIA standard from year 1980 called RS274D that sort of
defines the G-codes but over the years manufactures have extended
and interpreted it differently so that today there are many dialects.

EazyCNC tries to support a common subset of what the two most
common hobby CNC controller programs, EMC2 and Mach3 support.

G-code should not be written by hand, instead a CAD/CAM program
should be used to generate it. Not only is this a lot faster but
it also reduces the chances of errors and improves the quality
of machining as the CAD/CAM program has a much better understanding
of the machining process as a whole than the CNC machine ever can.


CAM software actually produces a very simple, 'pidgin English'
kind of G-code which is rather universal in order to be
avoid the differences between machines. Different interpretation for
some of the G-codes can be configure in the Mach Setup screen,
see Section~\ref{sec:g-code-options}.

Therefore it is not really important to learn to write G-code, but
it is handy to be able to read it to some extent and this is
what this chapter tries teach you to do.


%----------------------------------------------------
\section{The Basics}

G-codes programs are text files made of lines of commands for the
CNC machine. The file names often have the file extension '.nc'
but this is by no means required, you can use '.txt' as well.

EazyCNC reads the program line by line executing the commands.

The commands are made of single letter 'words' possibly followed by number,
for example '\texttt{G0}' or '\texttt{X1000}'.

The words can be written in upper or lower case letters. Any number of
'space' characters can be placed anywhere on each line to improve readability;
they are ignored and make no difference to the execution.

In this document a hashtag, '\texttt{\#}' after a word is used to denote a
word followed by a value, i.e. number or expression.
These are optional arguments or parameters to the G-code
in question. For example :  '\texttt{G0 X\# Y\# Z\#'} indicates that 
you can write '\texttt{G0 X10 Y20}'. Weather parameters are
optional or not is explained in the text. A asterisk '\texttt{*}' is used to 
denote a compulsory or required word. 

Comments to the human reader i.e. text that EazyCNC will not try
to interpret as command can be placed on lines if preceded with
a semicolon ';' or enclosed in parentheses, '(' and ')'. 

Everything on after a semicolon 
'\texttt{;}' on a line is also treated as a comment.

The order of words on a line makes no difference.

No word or G-code can appear more than once on any given line.

Some G-codes are actually M-codes!

Line numbers are optional and effectively ignored but if you
want you can use line numbers. Line numbers are expressed as
the '\texttt{L}' word followed by up to five digits.

\subsection{Operator Messages}

Everything after the comma ',' in a comment
that starts with '\texttt{(MSG,}''
is display as message to the operator in the status display,
Figure Figure~\ref{fig:status-display}. 

\subsection{Debug Messages}

Everything after the comma ',' in a comment
that starts with '\texttt{(PRINT,}''
is output to the Java console. This can be
very handy when debugging G-code programs.  When outputting
text to the console \#number -references in the text are
replaced with value of the parameter with that number.

For example :  

\texttt{(PRINT,value of parameter 123 is \#123)}

will output the value of parameter 123 into the console.


%----------------------------------------------------
\section{Numbers, Expressions and Parameters}

%----------------------------------------------------
\subsection{Numbers}
Numbers in commands are expressed as one or more digits,
optionally followed by a decimal point and one or more digits. 
A number can optionally be preceded with a plus or minus sign.

It is also acceptable to leave out a zero preceding the decimal point.

Some valid examples:

\texttt{7 3.1415 -2.0 .1}


\subsection{Expressions}

Where ever a number can be used an arithmetic enclosed expression in brackets, '['
and ']', can be used.

For example :  the commands

\texttt{X100}

and

\texttt{X [ 50 + 70 ]}

are equivalent.

\subsubsection{Arithmetic operators}

Nine common arithmetic operations are supported.

The operations are dived to three groups and
denoted as follows. Operations in a higher precedence group are executed before those with lower precedence. 
Operators in on the same precedence level are executed from left to right. This
pretty much follows the common arithmetic expression calculation order.

The highest precedence group only has the exponentiation or power operator '\texttt{**}'.

The second highest precedence contains the multiplication, division and modulus operators
'\texttt{*}', '\texttt{/}' and '\texttt{MOD}' respectively.

The lowest precedence is for addition, substraction, i.e. '\texttt{+}' and '\texttt{-}', and the logical
operators '\texttt{AND}', '\texttt{OR}' and '\texttt{XOR}' aka exclusive or.

The logical operators work on numbers interpreting any non zero number as logical one
or true value.

The order of operations can be altered by enclosing the sub expression in square brackets 
('\texttt{[}' '\texttt{]}'),  do not use parentheses ('\texttt{(}' '\texttt{)}').


\subsubsection{Arithmetic functions}

Arithmetic expressions support a number of common arithmetic functions. See table Table~\ref{tab:g-code-functions}.


\begin{table}[!ht] 
\begin{center} 
\begin{threeparttable} 
\caption{Mathematical functions}
\label{tab:g-code-functions}
\centering
\begin{tabular}{l l}
\hline\hline
Operator & Function\\ [0.5ex] 
\hline
ABS& absolute value\\
ROUND& round to nearest integer\\
FIX & round down to previous integer\\
FUP & round up to next integer\\
COS & cosine function\\
SIN & sine function\\
TAN &  tangent function \\
ACOS & arcus cosine function\\
ASIN & arcus sine function \\
SQRT & square root function \\
EXP & e to the 'power of' function\\
LN & natural logarithm function\\
\hline
\end{tabular}
\label{table:nonlin}
\end{threeparttable}
\end{center}
\end{table}


A function is expressed as the name of the function followed by the
function argument enclosed in brackets, '[' and ']'. 



For example :  calculate the square root of two you would write:

\texttt{SQRT[2]}

Note that for some weird historical reason one function, ATAN, 
has an exceptional format, to write the equivalent of

$arctan(123/456)$

you need to write

\texttt{ATAN[123]/[456]}



%----------------------------------------------------
\subsection{Parameters}

Parameters are what most programming languages would call variables
i.e. they are storage locations for values which can be used in place
of numbers in expression.

Instead of names the parameters are referred to by numbers preceded with a '\texttt{\#}'\#sign. 

For example : 

\texttt{\#5324}

Some parameters have a special meaning, others you are free to use 
as you wish to make your code more useful.

You can for example create G-code programs that is parametrized 
to cut different sized parts by just changing a parameter or two. 

To change or set a parameter you follow the parameter name with the
equal sign '\texttt{=}'. For example : 

\texttt{\#5324 = 12.34}

It is possible to both set and refer to the value of a parameter, even
several times, on the same line. The parameter values are only assigned
or set once the whole line has been executed. If a parameter is 
assigned several times on a line the last one will take effect.

So for example after executing following lines:

\texttt{\#22=5}
\texttt{\#22=[\#22+1] \#23=\#22 \#22=[\#22-1]}

the parameter 22 has value 4 and parameter \#23 has value 5.

Note that the number following the '\texttt{\#}'\#sign can itself be
an expression which makes for some interesting possibilities, 
you can for example use it to index parameter to simulate
what most program languages would call arrays.



%----------------------------------------------------
\section{G-codes and M-codes}

%----------------------------------------------------
\subsection{Length Units, G20,G21 codes}

Coordinates i.e. positions or distances in G-codes are expressed in either millimetres
or inches. 

The G20 code tells EazyCNC that following coordinates and lengths are in inches, G21 tells it
that they are in millimeters.

It is recommended that no coordinates or length are specified on the same line as G20 or G21
code is used.


%----------------------------------------------------
\subsection{Coordinate Axes}

G-code specifies the movement of the machining tool using orthogonal Cartesian
coordinates. (There is also polar coordinate mode, see G-codes G15 and G16.)

The coordinate system follows the right-hand rule with the positive X-axis pointing to right,
Y-axis away from the operator and Z-xis pointing up, so values increase from left to right,
front to back, and bottom to top, as illustrated in Figure~\ref{fig:coordinate-axes}.

%---------------------------------------------------------------------------
\begin{figure}[htb]
%\includegraphics[scale=0.4]{cnc-system-overview.png}
\caption{Coordinate axes of a 3-axis CNC System}
\label{fig:coordinate-axes}
\end{figure}
%---------------------------------------------------------------------------

%----------------------------------------------------
\subsection{Setting the length units -- G20,G21}
\label{gcode-G20}
\label{gcode-G21}

G-codes specify the length and any related physical quantities like feedrate in
either  metric or imperial units. For the metric system the basic unit is one millimeter
and for the imperial system it is one inch. It is possible but not recommended 
to mix the units in a single G-code program, but it is perfectly feasible
machine 'metric' G-code programs in a machine configure for imperial units
as long as the program contains the proper G-code to set the length units.

The \texttt{G20} sets the imperial (inch) units mode and \texttt{G21} sets
the metric (mm) units mode.

\noindent
It is an error to have botn \texttt{G20}  and \texttt{G21} on the same line.


%----------------------------------------------------
\subsection{Feedrate -- F-word}
\label{gcode-F}

The \texttt{F}-word sets the current feedrate for all the other movement
commands than \texttt{G0} which is always performed at maximum speed.

Depending on the length units mode in effect the \texttt{F}-word value
specifies the feedrate either in inches per minute or millimeters per
minute.

\noindent
It is an error if the feedrate is not a positive number.


%----------------------------------------------------
\subsection{Spindle speed -- S-word}
\label{gcode-S}

The \texttt{S}-word value sets the current spindle speed in rotations per minute (rpm).

It is ok to specify a larger rpm value than what is
set up for the system, see Section~\ref{sec:spindle-setup},
but of course the spindle can't run faster that it's
maximum speed.

Setting the spindle speed does \emph{not} turn on the
spindle, you need to program \texttt{M3} or \texttt{M4}
to turn on the spindle.

Note that the speed change can take several seconds to take effect so
a dwell or \texttt{G4} code should be programmed right
after a spindle speed change.

Also note that specifying zero speed does not guarantee that
the spindle stops, on the contrary it is most likely
to remain turning at some low speed. To stop the spindle
use the \texttt{M5} command. 

Remember never to touch the spindle unless the kill switch
has been applied!

\noindent
It is an error if the value is not a positive number or zero.

%----------------------------------------------------
\subsection{Spindle on/off -- M3,M4,M5 codes}
\label{gcode-M3}
\label{gcode-M4}
\label{gcode-M5}

The \texttt{M3} code turns on the spindle in the clockwise
a.k.a. forward direction direction.

The \texttt{M4} code turns on the spindle in the counter clockwise
a.k.a. reverse direction direction.

Note that reverse running the spindle can be dangerous
if the chuck (in a lathe) is screw mounted and of course
trying to machine with reverse running cutter will
only damage the cutter and ruin the workpiece.

The \texttt{M5} code turns off the spindle.

Note that the turning the spindle on/off can take
 several seconds to actually take place so
a dwell or \texttt{G4} code should be programmed right
after commanding the spindle on or off.

Also note that depending on your spindle drive system
abruptly changing directions can damage the equipment or 
be dangerous.

\noindent
It is an error if more than one of \texttt{M4},\texttt{M5} or \texttt{M6} is
programmed on the same line.

%----------------------------------------------------
\subsection{Coolant on/off -- M7,M8,M9 codes}
\label{gcode-M7}
\label{gcode-M8}
\label{gcode-M9}

The \texttt{M7} (mist cooling) and \texttt{M8} (flood cooling) codes both 
turn on the coolant and mist/flood aspect of the codes is ignored.


The \texttt{M9} code turns off cooling.

Note that when turning coolant on it can take several seconds
for the coolant pump to react so you may want to program
a dwell or \texttt{G4} right after turning the coolant on.


\noindent
It is an error if more than one of \texttt{M7},\texttt{M8} or \texttt{M9} is
programmed on the same line.

%----------------------------------------------------
\subsection{Select a tool -- T-word}
\label{gcode-T}

The \texttt{T}-word designates a tooltable entry as the current tool.
Note that this alone does \emph{not} do anything else. 

To physically change the tool you need to program a
pause \texttt{M0} and change the tool yourself and to 
use the tool length you need 
to use the  \texttt{G43} command and to use the tool diameter info
you need to program \texttt{G40} or\texttt{G41} as needed.

\noindent
It is an error if the value is not a positive or is larger than 
the number of tools supported by EazyCNC.

%----------------------------------------------------
\subsection{Dwelling -- G44-code}
\label{gcode-G4}

The \texttt{G4}-word causes EazyCNC to wait for the
specified time before executing the next G-code program line,
this is useful for example after turning on or changing the spindle 
speed. 

The wait time is specified with the \texttt{P}-word
in either seconds or milliseconds depending on which G-code interpretation
options has been selected, see Section~\ref{sec:g-code-options}. 
If a G-code program seems to hang for a long time then probably that
setting for the \texttt{P}-word interpretation is wrong.

\noindent
It is an error if the value is not a positive or zero.





%----------------------------------------------------
\subsection{Coordinates/moving axes -- XYZABC -words}
\label{sec:moving-axes}

Coordinates in G-code programs are expressed as an axis letter, see followed 
by the coordinate position either as a number or as an expression enclosed
in brackets.

The presence of an axis letter in a G-code line is an implicit command
for the named axis to move. 

The movement is carried out in the 
current motion mode either \texttt{G0},\texttt{G1},\texttt{G2} or \texttt{G3}.
This allows for more compact representation of the toolpath as most of a 
G-code program will be just a long list of coordinates to move the tool.

Axis letter names the axis that will move, valid letters are 
'\texttt{X}','\texttt{Y}','\texttt{Z}','\texttt{A}','\texttt{B}','\texttt{C}'.

If an axis does not need to move on a given line it is not necessary to 
specify the axis and its position at all.

The axis movements are co-ordinated so that the movements start and
stop at the same time and the axis speeds are such that the tool will 
move at the specified rate.

Here is an example that will cause X and Y axis to move to the position 120,140
in current motion mode:

\texttt{X 120.0 Y [100.0+40.0]}

It is an error if the same axis word appears twice any given line.

Note that actual axis movements are subject to machine limits
acceleration limits and if the best speed mode (\texttt{G64}) is
enabled then the movements may 'cut corners'.    

Also note that the coordinates specified with the axis words
maybe scaled, offset and event rotated, see XXX

%----------------------------------------------------
\subsection{Motion mode -- G0,G1,G2 and G3 codes}

As describe above specifying coordinates with axis words causes
EazyCNC to move the axis according to the current motion mode.

The \texttt{G0},\texttt{G1},\texttt{G2} and \texttt{G3} codes
are used to specify the current motion mode, which will stay
in effect until and other motion mode is specified.

If the motion mode is explicitly specified with one of the
above G-codes, then at least one axis word needs to 
specified on the same line.

\noindent
It is an error to specify more than one motion mode in a single line.
 
\noindent
Is is an error to specify a motion mode without any axis words.
 
%----------------------------------------------------
\subsection{Rapid positioning -- G0 code}
\label{gcode-G0}

Rapid positioning will move the specified axes in co-ordinated
fashion as fast as possible i.e. at the Max Velocity set up 
in the Mach Setup, see YYY.

\texttt{G0} is used to rapidly position the tool for the beginning
of a cut and is not meant for machining.

\subsection{Linear interpolation -- G1 code}
\label{gcode-G1}


For some archaic reason the machining G-codes are called interpolations.

\texttt{G1} is used to tell EazyCNC to move the tool at the current
feedrate  to the given position. This is the 
'work horse' mode of all G-codes, most machining will take
place in this mode.

\subsection{Clockwise Arc interpolation -- G2 code}
\label{gcode-G2}



\texttt{G2} is used to tell EazyCNC to move the tool at the current
feedrate from its current position to the given position following
a circular arc path clockwise on the active plane as set by the \texttt{G17} (XY-plane),
\texttt{G18} (XZ-plane) or \texttt{G19} (YZ-plane).

Clockwise means as if you were looking down at the arc 
on the active plane from the third axis i.e. if your are
cutting in the XY plane and looking down at it from the positive Z axis.

The end point of the arc is specified with the \texttt{'X'},\texttt{'Y'}
\texttt{'Z'} words. It is acceptable to leave out axes which do not
need to move, for example typically if you are cutting an arc
in the XY plane then you don't specify the \texttt{'Z'} coordinate.

If a movement in the direction perpendicular to the current plane
is specified then the tool will actually follow a three dimensional
helical path.

The curvature of the arc is specified either by giving the centre
point of the arc or by specifying it's radius.

%----------------------------------------------------
\subsubsection{Specifying arc using center point}

\paragraph{\texttt{G2 X\# Y\# Z\# I\# J\# K\#}} specifies the arc curvature using
the radius method.

The center point is specified with the \texttt{'I'} and \texttt{'J'}
and \texttt{'K'} words for the coordinates in the  X,Y and Z 
planes respectively. The  the IJK words specify the center coordinates 
relative to the starting point of the arc or as coordinates in the 
current coordinate system. 

Which interpretation is used depends
on the machine setup see Section~\ref{sec:incremental-ijk-checkbox} or G-codes G90.1 and G91.1 see
Section~\ref{gcode-G90_1}. 

If your toolpath preview shows large
arcs that don't make sense then it is likely that the interpretation 
mode of the IJK words is wrong.

It is acceptable to omit any of the any but not all of the end point words 
(XYZ) and center point words (IJK) in which case the last specified 
word values are used.

Note that by specifying the start,end and center point of the arc
you are over specifying the arc, this may result in an error
message if the distance from the start point to the center differs too
much from the distance from the end point to the center.

%----------------------------------------------------
\subsubsection{Specifying arc using radius}

\paragraph{\texttt{G2 X\# Y\# Z\# R\#}} specifies the arc curvature using
the radius method.

If the center point method of specifying the arc curvature is not
used then radius of the arc must be specified with the \texttt{'R'}
word. The start and end points alone do not specify an unambiguous 
arc, mathematically for any two points and radius there are two arcs 
that connect the end points, one arc is less than 180\degree and the other is
larger.

If the radius specified with the \texttt{'R'} word is positive then
this is interpreted as the arc that turn less than 180\degree and
if the radius is negative then it is interpreted to mean the
arc that makes the longer turn and the absolute value of the
\texttt{'R'} word is used as the radius.

Do not try to cut full or nearly full circles with this method
as this makes the start and endpoints very nearly the same
which means that any roundoff error in the calculations has
large effect in the internal arc center point calculation.

%----------------------------------------------------
\subsection{Counter Clockwise Arc interpolation -- G3 code}
\label{gcode-G3}

The \texttt{G3} command performs the same as \texttt{G2} but the arc is 
'drawn' counter clockwise. 

%----------------------------------------------------
\subsection{Perform probing move -- G31}
\label{gcode-G31}

The probing command \texttt{'G31'} works the same as the \texttt{'G1'}
in other words it programs  a linear movement at current feedrate, 
except that this command must always be explicitly specified on a line
and does not 'carry over' from line to line like the modal \texttt{'G1'}.

If the probe input becomes activated during the movement
the movement is stopped as soon as feasible without losing
any steps and the parameters DRO values at which the
probe became active are copied to the parameters 2000-2005.

Note that it is important that the feedrate is low enough because
the probe movement will overshoot the position at which
the probe trips/triggers by an amount that  depends on the feedrate and
it is equally important that the probe design allows for the overshoot, otherwise
the probe and/or workpiece can be damaged.

%----------------------------------------------------
\subsection{Pause Machining -- M0,M1}
\label{gcode-M0}
\label{gcode-M1}

The command \texttt{'M0'} pauses the execution of G-codes
and puts the system into the HOLD-state. 

The \texttt{'M1'} works the same way except it only pauses if
the M1-switch on the user interface is activated, see Figure~\ref{fig:m1-lock}.

%----------------------------------------------------
\subsection{Stop Machining -- M2}
\label{gcode-M2}

The command \texttt{'M2'} stops the execution of G-codes
and puts the system into the STOP-state. 


%----------------------------------------------------
\section{Coordinate systems}

As said in Section~\ref{sec:moving-axes}, the coordinates
are specified using the axis words  '\texttt{X}','\texttt{Y}','\texttt{Z}',
'\texttt{A}','\texttt{B}','\texttt{C}' and  '\texttt{I}','\texttt{J}','\texttt{K}'
for the arc centers in \texttt{G2} and \texttt{G3} commands.

This is not the end of the story though.

G-codes provide number of ways to transform the axis word
values before they become final physical positions of the
CNC machine axes. 

This section provides the nitty gritty details.

Table~\ref{tab:order-of-transformations} lists, in
order of application, all the coordinate transformations 
every axis word value goes through.

If the absolute mode G53 is active then none of the
offsets or the rotation is applied.

\begin{table}[!ht] 
\begin{center} 
\begin{threeparttable} 
\caption{Coordinate transformations} 
\begin{tabular}{ l }
\hline
\label{tab:order-of-transformations} 
  Apply G51 Scaling \\
  If in G91 mode interpret words as incremental coordinates \\
  If in G16 mode interpret words as polar coordinates\\
  Apply G52 temporary coordinate offsets\\
  Apply G68 rotation \\
  Apply G44 tool length offset\\
  Apply G54 work/fixture offsets\\
\hline
\end{tabular}
\end{threeparttable} 
\end{center} 
\end{table}% 



%----------------------------------------------------
\subsection{Scaling -- G50,G51 codes}
\label{gcode-G50}
\label{gcode-G51}

\texttt{G50} turns off scaling and sets scale factors
for all axis words to 1.

\paragraph{\texttt{G51 X Y\# Z\# A\# B\# C\#}} turns on scaling and specifies
the scaling factor for the given axes. The scale factors for 
axes that are not specified remain at their previous values.

The purpose of scaling is to scale the part being cut and thus it
is the first operation to be applied to coordinates and so it will
not affect position of the part on the workpiece which is set by
the fixture offsets nor does it affect the tool length or
temporary offsets.

\noindent
For example :  following will scale a 2D part design to three times it's original size.

\begin{verbatim}
G50       ; ensure all scaling is off and set to 1.0
G51 X3 Y3 ; scale 2D design by three
\end{verbatim}

\noindent
It is an error to have both \texttt{G50} and \texttt{G51} on the same line.


%----------------------------------------------------
\subsection{Incremental mode -- G90,G91 codes}

\paragraph{\texttt{G91}} turns on incremental coordinate mode. In incremental
model all axis words are interpreted relative to the current to position
tool position.

\noindent
For example :  following


\begin{verbatim}
G91     ; turn on incremental mode
G1 X100 ; mill hundred units to the right
Y50     ; hundred units forward
X-100   ; back to start X coordinate
Y-50    ; and we are where we started from
\end{verbatim}


defines a rectangular toolpath 100 units wide by 50 units high to
the right and front of current tool position.

%----------------------------------------------------
\paragraph{\texttt{G90}} turns off incremental coordinate mode

\noindent
It is an error to have both \texttt{G90} and \texttt{G91} on the same line.


%----------------------------------------------------
\subsection{Polar coordinate mode -- G15,G16 codes}
\label{gcode-G15}
\label{gcode-G16}

\paragraph{\texttt{G16}} turns on the polar coordinate mode and sets
the current tool position as the origin of the polar coordinate system.
In polar coordinate mode the '\texttt{X}' word is interpreted
to mean the polar coordinate distance relative to the polar coordinate 
origin and '\texttt{Y}' word is interpreted as the angle (in degrees)
of the polar coordinate.

In incremental mode '\texttt{X}' and \texttt{Y}' are interpreted 
relative to the previous distance and angle, not the current 
Cartesian X,Y position.


\noindent
The following example defines a hexagonal tool path where
each side of the hexagon is 10 units long.

\begin{verbatim}
G16     ; turn on polar mode
G91     ; turn on incremental mode
X0 Y0   ; 'reset' distance and angle for incremental mode
X10 Y60 ; Cut the 1st side
X10 Y60 ; ... and 2nd side
X10 Y60 ; ... and 3rd 
X10 Y60 ; ... 4th 
X10 Y60 ; ... 5th 
X10 Y60 ; 6th side and back to where we stared from
\end{verbatim}

\noindent
Following example the defines a pentagonal tool path with radius of 20 units.

\begin{verbatim}
G16      ; turn on polar mode
G90      ; turn off incremental mode
X20 Y72  ; 
X20 Y144 ; 
X20 Y216 ; 
X20 Y288 ; 
X20 Y360 ; 
\end{verbatim}



\paragraph{\texttt{G15}} turns off the polar coordinate mode.

\noindent
It is an error to have both \texttt{G15} and \texttt{G16} on the same line.

%----------------------------------------------------
\subsection{Temporary coordinate system offsets -- G52}
\label{gcode-G52}



The {\texttt{G52 X\# Y\# Z\# A\# B\# C\#}} code sets the temporary
coordinate offsets for the given axes to the given values. The offsets for axes
that are not specified remain at their previous values.

%----------------------------------------------------
\subsection{Temporary coordinate system offsets -- G92,G92.1,G92.2,G92.3 codes}
%\label{gcode-G52}
\label{gcode-G92}
\label{gcode-G92_1}
\label{gcode-G92_2}
\label{gcode-G92_3}


All these codes are legacy features that are best left unused.

As all \emph{G52} and \emph{G92} codes 
all use the same mechanism mixing them requires extreme care, one more
reason to leave all this well alone!


\paragraph{\texttt{G92 X\# Y\# Z\# A\# B\# C\#}} sets the temporary
coordinate offsets for the given axes so that the current
tool position has the given coordinates. 

\paragraph{\texttt{G92.1}} saves the temporary offsets to 
G-code parameters 5211..5216. 

\paragraph{\texttt{G92.2}} clears the temporary offsets.

\paragraph{\texttt{G92.3}} restores the temporary offsets from
the parameters 5211..5216.




%----------------------------------------------------
\subsection{Coordinate system rotation -- G68,G69 codes}
\label{gcode-G68}
\label{gcode-G69}


\paragraph{\texttt{G68 A\# B\# I\# R\#}} sets the coordinate system
rotation.

The '\texttt{A}' and '\texttt{B}' specify in the local coordinate system
the X,Y coordinates around which the coordinate system is rotated.

The '\texttt{R}' word specifies the rotation in degrees, positive values
giving counter clockwise rotation when viewed down from positive Z-axis
i.e. looking down at the work piece.

If the '\texttt{I}' word is present then the '\texttt{R}' word value
is treated as an increment to the current rotation, the value of
'\texttt{I}' word is ignored.

Rotation is only available in the XY-plane. Rotation can only be
turned on if the active plane is XY (\texttt{G17}).

\paragraph{\texttt{G69}} turns off the coordinate system rotation.

\noindent
It is an error if '\texttt{A}', '\texttt{A}' or '\texttt{R}' is not
specified.

\noindent
It is an error to have both \texttt{G68} and \texttt{G69} on the same line.

%----------------------------------------------------
\subsection{Active plane -- G17,G18,G19 codes} 
\label{gcode-G17}
\label{gcode-G18}
\label{gcode-G19}

The arc interpolation i.e. arc cutting (\texttt{'G2'} and \texttt{'G3'}) works by calculating
a circular path in the active plane which is one of the main coordinate planes XY,XZ or YZ
plane.

The \texttt{'G17'} sets the active plane to XY-plane.

The \texttt{'G18'} sets the active plane to XZ-plane.

The \texttt{'G17'} sets the active plane to YZ-plane.

\noindent
It is an error to use more than one of \texttt{G17},\texttt{G18} or \texttt{G19} on the same line.

\noindent
It is an error to program \texttt{G18} or \texttt{G19} if the coordinate system rotation \texttt{G(} is on.

%----------------------------------------------------
\subsection{Tool length compensation -- G43,G44,G49 codes}

\paragraph{\texttt{G43 H\#}} sets the tool length compensation based
on the tool number specified with the '\texttt{H}' word and 
the tool length of set up for that tool in the Tool Setup panel.

\paragraph{\texttt{G44 H\#}} works the same as \texttt{G43} but
it expect that the length values in tool set up are negative,
this should not be used and is provided for compatibility only.
 
A '\texttt{H}' value of 0 can be used to turn off tool length
compensation and is equivalent to the \texttt{G49} command.

It is an error '\texttt{H}' word is missing, is not an integer,
is negative or larger than the number of tools EazyCNC supports.

\noindent
It is an error more than one of \texttt{G43},\texttt{G44} and \texttt{G49} on the same line.

\paragraph{\texttt{G49}} turns the tool length compensation off by setting it to 0.0
value.


%----------------------------------------------------
\subsection{Work/fixture offsets -- G54,G55,G56,G57,G58,G59 codes}
\label{gcode-G54}
\label{gcode-G55}
\label{gcode-G56}
\label{gcode-G57}
\label{gcode-G58}
\label{gcode-G59}

Any of the six first work/fixture coordinates systems/offsets can
be selected with the \texttt{G54}...\texttt{G59} codes. The \texttt{G54}
selects the first i.e. number one coordinate system specified
in the Work Offsets panel.

\paragraph{\texttt{G59 P*}} selects the work/fixture coordinate
system specified with the '\texttt{P}' word. The '\texttt{P}' word
is optional and if not given it behaves as described above.


It is an error the'\texttt{P}' word is given with \texttt{G59}, 
is not an integer, is smaller than 1 or larger than the number of work/fixture coordinate
systems EazyCNC supports.

%----------------------------------------------------
\subsection{Absolute coordinates -- G53 code}
\label{gcode-G53}

If a line that causes linear interpolation (i.e. implicit or explicit \texttt{G0} or \texttt{G1}
command) contains the \texttt{G53} command then all the coordinates on that line are treated as absolute coordinates without applying any offsets or rotations. Scaling and cutter radius compensation are applied regardless if they are enabled.

%----------------------------------------------------
\subsection{Cutter radius compensation -- G40,G41,G4 codes} 
\label{gcode-G40}
\label{gcode-G41}
\label{gcode-G42}

EazyCNC can adjust the programmed tool path automatically to compensate
for the non-negligible width of the cutter. While this works the cutter
compensation is best carried in the CAM software that you should be using to create the 
G-code program. 

When you use the G-code cutter radius compensation you 
program your tool paths as if you were cutting the outline
of the part with an infinitely thin cutter i.e. the 
XYZ coordinates specify the outline of the part to be cut.

Of course the real cutter has a substantial size so 
you need to tell EazyCNC what is the cutter diameter 
and on which side the part outline the cutter should
cut.

Use \texttt{G41} to indicate that the cutter should stay to left of the 
toolpath, use this if you are cutting the outline of a part
clockwise (or if you are cutting a hole counter clockwise). 

Use \texttt{G42} to indicate that the cutter should stay to the right.

Left and right are defined as if you were riding on the
cutter and facing the direction of the travel.

Note that when you turn on the cutter compensation it
will only affect the next movement of the tool ie the
tool does \emph{not} move from where it is when you
turn on the compensation, rather the next position
will be offset by the tool radius and the next move
will then move from the current position to that
compensated position. Therefore you should always
plan a move-in movement when you turn on the compensation.

To turn off the compensation use \texttt{G40}.

There are several ways you can specify the tool/cutter
radius. Following shows them for \texttt{G41} but
\texttt{G42} works just the same, only the compensation
is taken to the right.

To turn on the compensation to the left and use the tool/cutter
diameter stored in the tool table for the currently selected
tool (as specified with the \texttt{T} -word) use \texttt{G41}.
 
To turn on the compensation to the left and use the tool/cutter
diameter stored in the tool table use \texttt{G41 D\#}, where
the  \texttt{D} word specifies the number of the tool in the
tool table. 

To explicitly set the compensation to the left with
given amount of compensation use \texttt{G41 P\#}, where
the  \texttt{P} word specifies the \emph{radius} of the cutter.

%----------------------------------------------------
\subsection{Feedrate mode -- G93,G94,G95 codes} 
\label{gcode-G93}
\label{gcode-G94}
\label{gcode-G95}

The EazyCNC does \emph{not} support the inverse time feedrate mode 
\texttt{G93} \emph{nor} the units per ref feedrate mode \texttt{G95}.

The \texttt{G94} code programs the feed per minute mode in which
axes movements are carried out so that the that tool moves at the rate specified with
the \texttt{F}-word inches or millimeters per minute depending on which length
unit mode, \texttt{G20} or \texttt{G21}, is in effect.


\noindent
It is an error to use \texttt{G93} or \texttt{G95} code.

\texttt{G40}\emph{nor}

 
%----------------------------------------------------
\subsection{Feedrate override on/off -- M48,M49 codes} 
\label{gcode-M48}
\label{gcode-M49}

The feed override that the machine operator can adjust
during machining, see Section~\ref{sec:feed-override},
can be turned on or off in the G-code program, but programming
it in G-code does not prevent the operator form turning
it on and off again.

The idea is that the G-code programmer knows the 
best what at what feedrate the part should be cut
but sometimes it is necessary to adjust or fine
tune that while machining.

The actually override percentage \emph{cannot} be set
with G-codes.

The \texttt{M48} code turns the feed override on.

The \texttt{M49} code turns the feed override off.

\noindent
It is an error to have both \texttt{M48} and \texttt{M49} codes on the same line.


%----------------------------------------------------
\subsection{Tool change -- M6 code}
\label{gcode-M6}

EazyCNC does not support the \texttt{M6}  tool change command
but ignores it.


%----------------------------------------------------
\subsection{Tool length compensation -- G43,G44,G49 codes}
\label{gcode-G43}
\label{gcode-G44}
\label{gcode-G49}

The length compensation basically works by just offsetting
the Z-axis value so that there will be room for the specified
tool length between the spindle/chuck and the work piece.

To turn off the tool length compensation program \texttt{G49}.

To turn on and set the tool length compensations from the length 
 stored in the tool table use \texttt{G43 H\#}, where
the  \texttt{H}-word specifies the number of the tool in the
tool table. If the \texttt{H}-word is zero i.e. specifies no
tool number then compensation is set to zero and effectively
turned off, in fact this equivalent to programming \texttt{G49}.

The \texttt{G44} works the same as \texttt{G43} but expects
that the length values in the tool table are negative. Since
you can't enter negative values this is provided for compatibility
with existing practice only.

\noindent
It is an error if the \texttt{H}-word value is negative or
larger than the number of tool supported.

\noindent
It is an error to have more than one of these G-codes on the
same line.

%----------------------------------------------------
\subsection{Path mode -- G61,G61.1,G64 codes}
\label{gcode-G61}
\label{gcode-G61_1}
\label{gcode-G64}

% all this text should be, now in G-code reference section.

In any CNC system there are basically two options how
the system tries to follow the specified tool path. Either
the system tries to obey the specified coordinates i.e. position
or the specified speed. You can't have both at the same time,
think about it: if you need to move from point A to point B
at a given speed you would need infinite acceleration and
deceleration at the beginning and end of travel.

The path mode along with the machine limits, 
Section~\ref{sec:setup-movement-limits}, 
determine how EazyCNC calculates the actual tool path.

The set exact path mode program \texttt{G61}, in this mode
the path follows the specified path as closely as possible 
which results in the axes velocities coming to a complete stop 
at the end of movement. This is fine when milling but slows
down the machining, especially if a lot of small cuts are
used. When cutting with a torch stopping at the end
of the movements causes local 'burn outs' so this is not
an acceptable mode for plasma cutting.

The set best speed mode program \texttt{G64}, in this mode
the path tries still to follow the specified path but
is allowed to deviate from it by as much as the 'Path tolerance',
Section~\ref{sec:path-tolerance-entry-field}, allows trading
accuracy for speed. Because of limited path lookahead many small
cuts in a row still result in a severely limited speed, so
when programming tool paths for plasma try to avoid
small movements.

For all practical purposes \texttt{G61.1} performs the same
as \texttt{G61} and this is supported for compatibility only.

\noindent
It is an error to have more than one of these commands in
the same line.


%----------------------------------------------------
\subsection{Incremental XYZ mode -- G90,G91 codes}
\label{gcode-G90}
\label{gcode-G91}

The axis words, '\texttt{X}','\texttt{Y}', '\texttt{Z}' etc
can be interpreted either as coordinates or as a change
of coordinates relative to the previous coordinates.

To treat coordinates as 'absolute' positions in the
local/current coordinate system program \texttt{G90}. 
This is the usual way to specify coordinates in G-code 
programs.

To turn on the incremental interpretation program 
\texttt{G91}, in this mode the axis word values
are treated as increments to the previous axis word
values.

\noindent
It is an error to have both of these codes in the same line.

%----------------------------------------------------
\subsection{Incremental IJK mode -- G90.1,G91.1 codes}
\label{gcode-G90_1}
\label{gcode-G91_1}

Interpretation of the IJK values in the arc interpolation
codes \texttt{G2} and \texttt{G3}
can be either absolute or incremental.
 
To set the absolute interpretation use \texttt{G90.1}, in
this mode the IJK words specify the center coordinates
in the local/current coordinate system.

To set the incremental interpretation use \texttt{G91.1}, in
this mode the IJK words specify the center coordinates
relative to the starting point of the arc.

Incorrect settings of this mode will usually result in large 
and incorrectly  oriented arcs in the toolpath display.

You can also set this mode in the Mach Setup screen, 
see Section~\ref{sec:incremental-ijk-checkbox}.

\noindent
It is an error to have both of these codes in the same line.


%----------------------------------------------------
\subsection{Set tool table -- G10 L1 code}
\label{gcode-G10_L1}

It is possible change tool table entries with G-code commands.
This makes it possible to maintain different tool sets.


\paragraph{\texttt{G10 L1 P\# A\# Z\# X\#}} sets the tool table
entry for tool number specified by the \texttt{P}-word. The
\texttt{Z}-word sets the tool height and the \texttt{X}-word
sets the tool \emph{radius}; the \texttt{A}-word, tool tip
radius, is ignored but accepted for compatibility reasons.

\noindent
The \texttt{A},\texttt{Z} and \texttt{X} -words are all
optional and it is ok to leave any or all of them out.

\noindent
It is an error if the P word is missing, smaller than one or
larger than the number of tools EazyCNC supports.

\noindent
It is an error if the L word is missing.



%----------------------------------------------------
\subsection{Set work/fixture offsets -- G10 L2 code}
\label{gcode-G10_L2}

It is possible to set the work/fixture offsets in G-code.
This makes it possible to maintain multiple different jig set ups
easily.



\paragraph{\texttt{G10 L2 P\# X\# Y\# Z\# A\# B\# C\#}} sets
the work/fixture offsets for the fixture number specified
with the \texttt{P} word. The axis words \texttt{X},\texttt{Y} etc
specify the corresponding offsets. It is ok not specify all
or none of the axis and those that are not specified are left
untouched.
 
\noindent
It is an error if the P word is missing, smaller than one or
larger than the number of work/fixture offsets EazyCNC supports.

\noindent
It is an error if the L word is missing.





%----------------------------------------------------
\section{Canned Drilling Cycles}

Canned drilling cycles (G73,G81,G82,G83 and G85) all work more or less 
the same with detail variations to help with real world drilling issues
such chip breaking etc.

The canned cycles are modal i.e. once a command is given it causes repeated execution of
the cycle for each line that specifies an X/Y position until the mode is
cancelled with \texttt{'G80'} or some other modal movement command.

In the following in the interest of simplicity canned cycles are described
as working in the XY plane with the drilling happening Z directions. However
they work equally well in any other plane (\texttt{G17,G18,G19}).

Note that in the following desctiption when I talk about 'depth' it really refers to a 
Z-position, not depth as a you as a machining would define the depth
of a hole.


All canned drilling commands accept an optional repeat count specified with the L-word.


At first sight this seem silly, why would you drill the same hole 
multiple times? Of course you don't but when combined with 
incremental distance mode (\texttt{'G91'})
it allows simple commands to produce reqular lines or arcs of holes.

If you plan to use the incremental distance mode note that the first
hole is drilled after the increment has been applied so it does
not end up where you position the tool.

Interweb has a lot of great illustrations about canned cycles so
I will leave a more graphical presentation to more capable hands.


It is an error if tool radius compensation is on when a canned cycle
command is given.

It is an error if rectraction level R is greater than initial Z-level.
 
It is an error if hole bottom level is higher than the rectraction plane R.


Below is a simplified description how the different canned cycles differ from each other.

G81 - Drilling

This is the mother of all drilling cycles, a single peck move  down to the
desired depth at feedrate and rapid retract.

G83 - Peck Drilling 

Same as G81 but makes a multiple increasingly deeper pecks at feedrate to drill a
deep hole with full rapid retraction between pecks to clear the hole from
chips.

G73 - High Speed Peck Drilling 

Same as G83 but to speed things up the retraction is shorter as the retraction is
for just breaking the chip.

G82 - Spot Facing 

Same as G81 drilling but the tool dwells at the bottom of the hole for a specified time
before retracting.

G85 - Boring 

Same as G81 drilling but the retraction as well as the drilling is done at
feedrate rather than as a rapid motion.
%----------------------------------------------------
\subsection{Cancel Canned Cycle -- G80 code}
\label{gcode-G80}

The command \texttt{'G80'} cancels canned cycle modal mode.

%----------------------------------------------------
\subsection{Canned Cycle Return level -- G98,G99 codes}
\label{gcode-G98}
\label{gcode-G99}

After each hole the drill returns to either to the Z-position it was 
before the cycle started or to the position set by the R-word.

Use \texttt{'G98'} to cause the tool to return to the original 
Z-position after the cycle.

Use \texttt{'G99'} to cause the tool to return to level set by the R-word.

%----------------------------------------------------
\subsection{High Speed Peck Drilling -- G73 code}
\label{gcode-G73}

\paragraph{\texttt{G73 X\# Y\# Z\# R* Q* F\# L\#}} command is used to
drill a hole to depth set by the Z-word in location specified with
the X and Y -words with pecks of distance set by the Q-word
and rectracting the drill between pecks to depth set by the R-word.

Pecking is performed at feedrate set by the F-word,
retraction happens at maximum velocity. 

An optional repeat count can be given with the L-word.

%----------------------------------------------------
\subsection{Drilling -- G81 code}
\label{gcode-G81}

\paragraph{\texttt{G81 X\# Y\# Z\# R* F\# L\#}} command is used to
drill a hole to depth set by the Z-word in location specified with
the X and Y -words with single peck and then retract.

Pecking is performed at feedrate set by the F-word,
retraction alway happens at maximum velocity. 

An optional repeat count can be given with the L-word.

%----------------------------------------------------
\subsection{Spot Facing -- G82 code}
\label{gcode-G82}

\paragraph{\texttt{G82 X\# Y\# Z\# R* F\# P\# L\#}} command is used to
drill a hole to depth set by the Z-word in location specified with
the X and Y -words with single peck and then retract.

Pecking is performed at feedrate set by the F-word,
retraction always happens at maximum velocity. 

The drill will dwell at bottom of the hole a for time set by the P-word
if given.

An optional repeat count can be given with the L-word.

%----------------------------------------------------
\subsection{Peck Drilling -- G83 code}
\label{gcode-G83}


\paragraph{\texttt{G83 X\# Y\# Z\# R* Q* F\# L\#}} command is used to
drill a deep hole to depth set by the Z-word  in location specified with
the X and Y -words with pecks of  distance set by the Q-word
and  rectracting the drill fully between pecks to depth set by the R-word.

Pecking is performed at feedrate set by the F-word,
retraction always happens at maximum velocity. 

An optional repeat count can be given with the L-word.

%----------------------------------------------------
\subsection{Boring -- G85 code}
\label{gcode-G85}

\paragraph{\texttt{G85 X\# Y\# Z\# R* Q* F\# L\#}} command is used to
drill a hole to depth set by the Z-word in location specified with
the X and Y -words with single move and then retract.

Both  drilling and retraction is performed at feedrate
set by the F-word.



An optional repeat count can be given with the L-word.


%----------------------------------------------------
\section{Using subroutines -- M98/M99}
\subsection{Call subroutine -- M98 code}
\label{gcode-M98}

It is possible to create subroutines in a G-code program.

A subroutine is any continuous sequence of G-code lines that
ends with the  \texttt{M99} G-code.

The first line of subroutine should contain the \texttt{O}-word, 
to give the subroutine an id-number. When calling the
subroutine with the \texttt{M98} code a \texttt{P} word
with the same id-number must be used.

Typically subroutines are placed at the end of a G-code
file after a M2, M30 or M99 code so that the normal G-code
execution does not reach them.

Subroutines can be nested.

A subroutine can be in the same file (recommend) where
the calling code is or in a separate file. 

An optional repeat count can be specified with the
\texttt{Q} or \texttt{L} words.

To call a subroutine use one of the following forms.


\paragraph{\texttt{M98 P1234}} calls
a subroutine that starts with  \texttt{O1234} in the 
same file as the calling M98 code.

\paragraph{\texttt{M98 P1234 (example.txt)}} calls
a subroutine that starts with  \texttt{O1234}
in a separate file called 'example.txt' .


\paragraph{\texttt{M98 P2000 L10}} calls ten times
a subroutine that starts with  \texttt{O2000} in the 
same file as the calling M98 code.

\noindent
It is an error if the P word is missing.

\noindent
It is an error if the both L and Q words are used.

\noindent
It is an error if the L or Q word specifies a count smaller than one.

\noindent
It is an error if 

\subsection{End of subroutine -- M99 code}
\label{gcode-M99}

\paragraph{\texttt{M99}} returns G-code execution to the place whence the subroutine was called from.

\noindent
It is an error this code is executed when no subroutine (M98) has been called.



