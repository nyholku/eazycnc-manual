\chapter{Hardware and Software Requirements}

This chapter gives you important information about the software and hardware
requirements for a system based on EazyCNC and TOAD4. 


\section {Dedicated Computer}

The computer used for CNC machining should be dedicated for the CNC system and not shared for other use. 

A CNC machine should not be directly connected to Internet because of the possibility
of viruses and malware causing havoc in an environment where they can 
cause actual physical damage and injury. 

If the dedicated CNC PC is connected to a local network it is important
that the network is secure and runs behind a firewall and the all 
network connectivity on the CNC PC is kept to minimum.

No other software besides the operating system and EazyCNC should be running
on the CNC PC during machining. You should especially watch out for 
programs that start automatically behind the scenes when the computer
boots up. Utility programs to kill non required applications and processes
exist.

All screen savers, auto log-off and power save features and modes should be turned off.

Once the system has been set up, configured properly and tested it should be 'freezed' and \underline{any
upgrades and changes to the system should be approached with due caution and care}.


\section {Hardware}

It is impossible to give hard limits as to which kind of PC computer should be used with EazyCNC.

In a less demanding application at moderate feed rates you can get by with a less powerful computer.

It is tempting to utilize that old PC that is just lying there gathering dust, but that is not recommended. 

Consider that you are building an automated machining tool and for that you want reliability and responsiveness. A suitable PC can be purchased for a few hundred euros/dollars and is well worth it considering that you are building and what your are going to do with it.

A fairly recent and decent PC with at least 1 GHz processor, good graphics card, a minimum of 2 GB RAM is recommended as the minimum. To install and run EazyCNC you need at least 200 MB of free disk space even if the application itself is only about 12 MB.

During machining all the G-code data as well as the tool path graphics is kept in memory so there is no such thing as too much memory, too fast graphics card or too fast CPU!


\section {Operating System}

The following operating systems/version have been tested:



\begin{itemize}
\item Mac OS X 10.8.5 
\item Windows XP sp 2
\item Windows 7
\item Linux Lubuntu 13.10
\end{itemize}


EazyCNC is built on Java which is fairly operating system independent so 
it is likely it will run with a wide range of version of above mentioned operating systems
but of course it is not possible to guarantee that.

Also worth noting is that operating system version tend to become obsolete and 
unsupported in a matter of some years and while I try to maintain compatibility
with older OS versions it may become impossible at anytime.

As an example, the trusty old Windows XP is well beyond its sell-by-date and
no longer supported in any shape or form by Microsoft but lots of people still
use it. Today EazyCNC runs on Windows XP with but this situation may
change.


\section {Java}

Starting from version 0.0.0.20a EazyCNC embeds the Java
virtual machine so you do not need to install it, you
can skip this section altogether.

EazyCNC requires Java Runtime Environment (JRE). 

Java is an operating system independent virtual machine that
allows software written on one operating system to run more or
less seamlessly on other operating systems.

Java was originally developed by Sun and later acquired
by Oracle and Open Sourced. Before Java was Open Sourced
the GNU movement created an incomplete but Free clone
of Java.

EazyCNC has been tested on the Sun/Oracle Java and
you are advised to use that. Having said that it is
likely that EazyCNC will also work fine with OpenJDK Java,
but it will not work with GNU Java.

EazyCNC has been tested with Java version 7 but it
should also run just fine on Java 6, especially on Mac OS X. 

Note that the JRE version has an effect on the real-time qualities of the 
EazyCNC so you should not trivially and without due diligence upgrade the JRE 
version, even if recommended by media hype and vulnerability scares -- a computer
not connected to Internet is not subject to the alleged Java vulnerabilities.

In the future EazyCNC will have Java JRE built in so the Java version question
will become moot.

You can download Oracle Java JRE from:

http://java.com

On pre Mountain Lion Macs Java is pre-installed so you do not need to and
should not install a newer version from Oracle.


\section {Anti-Virus Software}

An anti-virus software is not recommended on the PC running EazyCNC as such
software is very intrusive and can cause real-time violations and machining
failures. Further, it is not very necessary because the CNC PC should not be connected
to Internet.

It is of course important that anti-virus software or some other means
are used to ensure clean operation of the computer 
producing or transferring the G-code file as a virus
might infect the media, such as a USB memory stick, used to transfer the G-code to the CNC system.





