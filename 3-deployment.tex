\chapter{Hardware and Software Requirements}

This chapter gives you important information about the software and hardware
requirements for a system based on EazyCNC and TOAD4. 


\section {Dedicated Computer}

The computer used for CNC machining should be dedicated for the CNC system and not shared for other use. 

A CNC machine should not be directly connected to Internet because of the possibility
of viruses and malware causing havoc in an environment where they can 
cause actual physical damage and injury. 

If the dedicated CNC PC is connected to a local network it is important
that the network is secure and runs behind a firewall and the all 
network connectivity on the CNC PC is kept to minimum.

No other software besides the operating system and EazyCNC should be running
on the CNC PC during machining. You should especially watch out for 
programs that start automatically behind the scenes when the computer
boots up. Utility programs to kill non required applications and processes
exist.

All screen savers, auto log-off and power save features and modes should be turned off.

Once the system has been set up, configured properly and tested it should be 'freezed' and \underline{any
upgrades and changes to the system should be approached with due caution and care}.


\section {Hardware}

It is impossible to give hard limits as to which kind of PC computer should be used with EazyCNC.

In a less demanding application at moderate feed rates you can get by with a less powerful computer.

It is tempting to utilize that old PC that is just lying there gathering dust, but that is not recommended. 

Consider that you are building an automated machining tool and for that you want reliability and responsiveness. A suitable PC can be purchased for a few hundred euros/dollars and it is well worth it, considering what you are building and what you are going to do with it.

A fairly recent and decent PC with at least 2 GHz processor, good graphics card, a minimum of 4 GB RAM is recommended as the minimum. To install and run EazyCNC you need at least 400 MB of free disk space.

During machining all the G-code data as well as the tool path graphics is kept in memory so there is no such thing as too much memory, too fast graphics card or too fast CPU!


\section {Operating System}

The following operating systems/version have been tested:



\begin{itemize}
\item Mac OS X 14.1 
\item Windows 10
\item Linux Ubuntu 18.04 LTS 
\item Raspberry Pi OS 4.3 (bullseye)  
\end{itemize}


EazyCNC is built on Java which is fairly operating system independent so 
it is likely it will run with a wide range of versions of above mentioned operating systems
but of course it is not possible to guarantee that.

Also worth noting is that operating system version tend to become obsolete and 
unsupported in a matter of some years and while I try to maintain compatibility
with older OS versions it may become impossible at any time.


\section {Anti-Virus Software}

An anti-virus software is not recommended on the PC running EazyCNC as such
software is very intrusive and can cause real-time violations and machining
failures. Further, it is not very necessary because the CNC PC should not be connected
to Internet.

It is of course important that anti-virus software or some other means
are used to ensure clean operation of the computer 
producing or transferring the G-code file as a virus
might infect the media, such as a USB memory stick, used to transfer the G-code to the CNC system.





