%XXX add a chapter about limit switch wiring somewhere XXX
%XXX add a chapter where to put the reference switch XXX
%XXX add about stepper motor coils XXX
%XXX add chapter about the units XXX
%XXX add chapter steppers in general XXX
%XXX implement automatic checking for lost steps XXX
%XXX implement G-code limit checks in software and maybe tool size to jogging
%XXX implement homing reset value
%XXX backslash compensation

\chapter{Setting Up and Configuring}


EazyCNC should work out of the box without any setup or configuration,
so you can play around and test it right away.

However before you can actually use it with TOAD4 and machine something
there are a few things you need to set up and configure.

This chapter proceeds in the preferred order of setting things up,
however I suggest you read it all through once before setting up
the system.

The two most important things to set up are communication between
your computer and you TOAD4 and configure each axis to match the motor
characteristics and axis gearing.

\section{Saving Your Setup}

EazyCNC stores all setup and configuration information as well
as the current machine state in file named 'EazyCNC-Mach-Config.ecnc' 
in a folder/directory named 'EazyCNC' in your 'home' directory.

Your 'home' directory location depends on your operating system as
follows:


\begin{tabular}{ll}
\hline
Mac OS X & \PVerb{/Users/username} \\
Windows XP & \PVerb{C:\Documents and Settings\username} \\
Windows 7 & \PVerb{C:\Users\username} \\
Linux & \PVerb{/Users/username} \\
\hline
\end{tabular}

where \emph{username} is the name you use when you log into your computer.


Note, EazyCNC does not ever automatically save the settings,
this is to protect you from accidentally altering your
carefully crafted machine setup and configuration. To
save your settings or any changes you've made you need to
click the 'SAVE' button. Also don't forget to make backup copies
of this file.

If the file or folder does not exist EazyCNC will create them
with reasonable default settings when you click the 'SAVE' 
button.

You do not have to 'worry' about this file but it is good to know
about it as you may want to make backup copies of it or maintain
several different ones for different system configurations.

The file is in a plain text format so it is ok to view and even
edit it manually, though that is not recommend unless you
know what you are doing. 

Many programs like Mac OS X Textedit may mess things up by 
adding formatting and changing the file name extension with
.txt or .rtf, so learn to avoid those if you venture there.



\section{Setting the Communication Port Name}
\label{sec:setting-up-commp-port}


Remember that you do not need to setup the communication port just to play with the program 
or to use it to simulate machining, perhaps on a different computer 
from your CNC computer.

TOAD4 looks like a virtual serial port to your computers
operating system.

EazyCNC uses the operating systems standard serial port
programming interface and device drivers to talk to TOAD4.


To set the communication port name start EazyCNC and
from the top/left corner buttons click 'Mach Setup'
button and then from the top row of buttons click 'Comm',
see Figure~\ref{fig:mach-setup-comm-view}.

%---------------------------------------------------------------------------
\begin{figure}[htb]
\includegraphics[scale=0.35]{mach-setup-comm-view.png}
\caption{The communication port setup screen}
\label{fig:mach-setup-comm-view}
\end{figure}
%---------------------------------------------------------------------------


Click on the entry field labeled "Port:" and type
in the port name. To find out the port name you need
to follow instruction in the following section for
your operating system, or you can use use the Auto Detect
feature, see section ~\ref{subsec:using-auto-detect-port}.

Note that the serial port name changes if you you
plug the USB cable to a different port in your 
computer. In some version of Windows it sometimes
seems to change for no apparent reason.

To test that the communication works click the 'MACH'
button at the bottom of the screen.

If everything is correct the very wide entry field 
just above the bottom row of buttons should
show the TOAD4 firmware version number. 

If it shows 'Controller: Device Not Found' then EazyCNC
was unable to open the serial port connection.

If it shows 'Controller: Device in use' this is means that
EazyCNC found the device but was unable to open it. This
message covers a variety of reasons. It is possible that
some other software is using the device, in which case it
is most likely an other instance of EazyCNC running on the
same computer. However in Linux the most likely cause is that
you do not have permissions to use the device, in which
case see section ~\ref{subsec:setting-up-permissions-in-linux}.


Don't forget to click 'SAVE' so that the port 
name is saved and you do not need to set it again!

\subsection{Using the Auto Detect port function}
\label{subsec:using-auto-detect-port}

Note that on Windows you first have to install the driver,
see below section ~\ref{subsec:installing-windows-serial-driver}.

To Auto Detect the port you need to have the TOAD4 powered 
on and the USB cable plugged into the computer; next click the 
'Auto Detect' button, and then un-plug and re-plug the USB
cable pausing for about one second in-between.

If the port is successfully detected a dialog will inform
you and ask if you want to use that port, click Yes and
the port name will be automatically entered in to the
'Port:' field.

If the port is not found a dialog will inform you and
you can try again more slowly or try to use the manual
methods outlined in the following sections.

Note for Mac OS X users: prior to EazyCNC version 0.0.0.24a the port
is sometimes erroneously detected as 'cu.usbmodemXXXX'
instead of 'tty.usbmodemXXX'. In theory both
port names should work but it appears that some un-identified 
combinations of Mac OX X, EazyCNC and TOAD4 have problems
with the 'cu.' named ports so ensure the name starts with
'tty.'.

\subsection{Installing the Windows Serial Port Driver}
\label{subsec:installing-windows-serial-driver}

As said above all operating systems, even Windows, come with the serial port 
driver installed.

Except that on Windows you have to tell the Windows where
it is and how to use it! 

Installing the driver is a bit complicated process.

Further more the details of the driver installation depend
on the Windows version.

Following is a brief outline how to do the driver installation
on Windows 7. For better and detailed instructions I refer to
google.

You need a file that contains the instructions for Windows
on how to locate and use the driver it already has and you 
need to tell Windows where that file is.

The file is called

\begin{verbatim}
eazycnc.inf
\end{verbatim}

and you can download it from the same place you downloaded
the EazyCNC application file.

To tell Windows about this file plug in the
USB cable from your TOAD4 to your computer make and
sure TOAD4 is powered on.

Then Go to the Windows Start -menu and from there select Printers and Devices.

From the window that appears select (somewhere at the bottom) 
the 'unknown device' that represents the TOAD4 at this stage, 
right click on it and select Properties.

From the dialog that appears select the Hardware-tab, then in 
that tab , from the Device Functions list, select the Driver 
and click Properties.

From the dialog that appears, select the Driver-tab and 
from that tab select Update Driver...

From the dialog that appears select Browse my computer 
for driver software, then click the 'Browse...'
button and browse to the directory/folder where 
you have the 'eazycnc.inf' file and click next.

This should install the driver for you.

Once the installation is complete go back to the Printers
and Devices window and make a note of the COM port name
that is now show in place of the 'Unknown Device'.

Enter the COM port name in the port field 
in the 'Mach Setup' screen.

\subsection{Finding out the virtual serial port name on Mac OS X}

In Mac you need to go to the Terminal program which is in the
the /Applications/Utilities folder or you can use Spotlight to
locate it.

Plugin TOAD4 and make sure it is powered on.

Open the terminal and type the following:

\begin{verbatim}
ls /dev
\end{verbatim}


this will produce a long list of devices, from this list
you need to look for something like:

\begin{verbatim}
tty.usbmodemNNNN 
\end{verbatim}

where NNNN is a series of digits.

If you find several, unplug TOAD4 and give the 'ls /dev'
command again, the one device that disappeared is the
one you are looking for.

Make a note of the device name and enter it in the port field 
in the 'Mach Setup' screen.



\subsection{Finding out the virtual serial port name in Linux}
\label{subsec:finding-out-serial-port-in-linux}


In Linux you need to go to the terminal which typically
can be found from the Accessories menu of the Desktop
application.

Plugin TOAD4 and make sure it is powered on.

Open the terminal and type the following:

ls /dev

this will produce a long list of devices, from this list
you need to look for something like:

\begin{verbatim}
tty.ACMnnn 
\end{verbatim}

where nnn is a series of digits.

If you find several, unplug TOAD4 and give the 'ls /dev'
command again, the one device that disappeared is the
one you are looking for.

Make a note of the device name and enter it in the port field 
in the 'Mach Setup' screen.

\subsection{Setting up permission in Linux}
\label{subsec:setting-up-permissions-in-linux}

Most Linux distros take a very serious view of security. This means that by default
you are not even allowed to use your own devices! To complicate matters there is
no easy way to give yourself the necessary permissions, instead you have to do
the following.

To allow EazyCNC to talk to the TOAD4 you need to create a file named:
 
\begin{verbatim}
99-TOAD4.rules 
\end{verbatim}

 and put it in the directory 

\begin{verbatim}
 /etc/udev/rules.d 
\end{verbatim}

In the  file '99-TOAD4.rules' you need to put the following line (all on one line):

\begin{verbatim}
SUBSYSTEMS=="usb", ATTRS{idProduct}=="000a", ATTRS{idVendor}=="0408", MODE="0660",
 SYMLINK+="ttyACM0", GROUP="plugdev"
\end{verbatim}

Basically you can do that with any text editor. Unfortunately you do not have the permission to write it to the directory '/etc/udev/rules.d'. So I recommend creating the file in your home directory and then use the Terminal and type in the following command
to copy the file over:

\begin{verbatim}
sudo cp ~/99-TOAD4.rules /etc/udev/rules.d
\end{verbatim}

This command will ask for your password to allow you temporarily write to that directory.


\section{Setting the User Interface}
\label{sec:mach-setup-units}

Now that the communication works and before we set up the 
motors and axis we want to select the length units you are comfortable
with. 

To do that go the the 'User Interface' seyup screen, Figure~\ref{fig:mach-setup-user-interface-view}. 

%---------------------------------------------------------------------------
\begin{figure}[htb]
\includegraphics[scale=0.4]{mach-setup-user-interface-view.png}
\caption{The User Interface setup screen}
\label{fig:mach-setup-user-interface-view}
\end{figure}
%---------------------------------------------------------------------------

EazyCNC supports working in millimeters or inches. All displays and entry fields
will always show values in the selected units and accept values in these
units. You can change the units at any time and it will not confuse EazyCNC,
so you can use millimeters to set up things and then switch to inches; however
even if EazyCNC will not get confused, you may, so it is best to select
one system of units and stick with it.

Note that regardless of the units selected here the G-code file can contain
coordinates that are expressed in millimeters (G21 mode) or inches (G20 mode),
and this is perfectly fine, as long as the correct G20/G21 mode is specified
in the G-code file.


\subsection{Units -popup menu}

There are two options.

'mm' -- with this setting all the entry fields and DROs displays are
in millimeters.

'inch' -- with this setting all the entry fields and DROs are
in inches.

\subsection{DRO-format -entry field}

With this entry field you can control how numbers in the entry fields and DROs are
displayed.

The main usage is to control the number of decimals you want displayed, to do 
that just enter '0.' followed by as many '0' characters as you want decimals.

For example with inches it is probably preferable to use three decimals to see
the 'thous' so enter '0.000' in this field, for working with millimeters
'0.00' is probably best.

\subsection{Update rate -entry field}

This entry field controls how many times per second (frames per second, fps) the
toolpath display is updated. Smaller values than 10 may produce jerky updates and
higher values than 30 are unnecessary and may bog down the computer.

\subsection{Language -popup menu}

This popup allows you to select the language used in the user interface. 


\subsection{Screen Size -popup menu}

This popup allows you to select between 'Standard' and 'Compact' screen layouts.

Depending on the operating system and the various docks and toolbars in them
the 'Compact' layout should fit a screen as small as 1024 x 600 at a pinch.

Note that you need to both Save the configuration and re-start the program. If
you are running in a small screen the Save-button may not be visible, in that
case use the Alt-Cmd-S for Mac OS X or Alt-Ctrl-S for Windows/Linux to save the
the configuration before re-starting EazyCNC.

\subsection{Machined Path -settings}
\label{subsec:machined-path-color-settings}

The settings in this panel control how the tool path for the already cut path is 
displayed in the Toolpaht display panel in the main screen.

A different color can be specified depending on weather the spindle was on or off
when the path was cut.

The width of the path in the Toolpath display panel, in \emph{pixels}, can be specified 
in the 'Width:' -entry field. If this is set to zero then the actual tool width from the
tool table for the tool selected by the G-code program is used in displayin the path.

\subsection{Planned Path -settings}
\label{subsec:planned-path-color-settings}


The settings in this panel control how the tool path for the planned i.e. yet to be cut
path is displayed in the toolpath display panel in the main screen.

A different color can be specified depending on weather the spindle will be on or off
when the path will be cut.

The width of the path in the Toolpath display panel, in \emph{pixels}, can be specified 
in the 'Width:' -entry field.

\subsection{Tool Display -settings}
\label{subsec:tool-color-settings}

The color of the tool in the Toolpath display can be spefied as well as the size of 
it by entering the desired tool width , in \emph{mm or inch}, into the 'Width:' -entry field.

If the tool width is set to zero then the actual tool width from the
tool table for the tool selected by the G-code program is used in displayin the path.

\subsection{Axis Display -settings}
\label{subsec:axis-color-settings}

The colors used to display the  axis (grids) in the Toolpath display can be spefied here.

\section{Configuring Motors and Axes} \label{axis-configuration}

To configure the motors and axes go to the 'Axis Setup' screen, 
see figure~\ref{fig:mach-setup-axis-view}.

%---------------------------------------------------------------------------
\begin{figure}[htb]
\includegraphics[scale=0.4]{mach-setup-axis-view.png}
\caption{The Axis setup screen}
\label{fig:mach-setup-axis-view}
\end{figure}
%---------------------------------------------------------------------------

On the top of the 'Axis Setup' panel you see four buttons that 
correspond to the four axes that TOAD4 can control. By clicking at 
those buttons you control which axis parameters are shown
on the panel.

There are three group of parameters for each motor and axis.


\subsection{Motor Config -panel}

There are five parameters for each motor.

\subsubsection{Axis On -checkbox}

Axis On -parameter controls weather the G-code commands control
that axis/motor or not. If the motor is not controlled by the
G-code (Axis On checkbox is not 'ticked') then that motor/axis is
available for manual jogging during machining or it can be
controlled with EazyCNC plugin extensions. 

Typically you want to use G-code to control the motors so
make sure the Axis On checkbox is ticked, but if an axis
is not used (say you only have a three axis set up) then 
un-tick the box so that you do not need to set up the
motor properly.

\subsubsection{Current -popup menu}

TOAD4 supports two different drive currents for each motor,
named High and Low, in addition to which the current can be totally
off. The actual motor current depends on the current measurement
resistors mounted to the TOAD4 board and the jumper settings
on the TOAD4 board, see TOAD4 Hardware Manual for details.

The Current -popup menu controls how the three different currents
are used when driving the motors.

There are four different options.

Low -- with this setting the motor current is always set to Low. 
You might want to use this if the High current is too much for 
your motor.

High -- with this setting the motor current is always set to High. This 
provides the most 'stiff' setup but means that motors will have full
full current applied and 'run' hot.

Auto Off -- with this setting when the motor/axis is moving or the G-code 
program is being executed the current will be set to High, but once 
the movement or machining stops the motor current will be turned off 
completely within two seconds.

Depending on the mechanics and usage this may not be ideal as the 
motors may move under external forces if there is no current and thus
the axis may lose its accurate position.

Auto Hold -- with this setting when the motor/axis is moving or a G-code 
program is being executed the current will be set to High, but once the movement or
machining stops the motor current will be set to Low within two seconds.

This is often the most desirable motor current setting as full current
and force is used during machining but the current and heat is reduced
when the motors are not being used.

\subsubsection{Forward -popup menu}

The Forward popup menu controls weather the direction output on
the TOAD4 board is 1 or 0 when the motor is driven forward.

Forward means that the coordinates of the axis are increasing.

There are two options.

Output = 0 -- with this setting the (internal to TOAD4 board)
DIR signal is set to logic zero to when the axis/motor
is driven forward.

Output = 1 -- with this setting the (internal to TOAD4 board)
DIR signal is set to one zero to when the axis/motor
is driven forward.

You need not to care about zeros or ones, just make sure this
setting is right! When you press the axis jog buttons (+X,+Y,+Z or +4) 
the motor should be running in the direction that you have designated
as the increasing coordinate for that axis.

If the motor runs in the wrong direction just change the setting
in this popup.

\subsubsection{Home -popup menu}

TOAD4 supports one home/reference position switch input for each axis. 

The Home popup menu controls weather the REF input on
the TOAD4 board is 1 or 0 when home/reference is switch is
active. 

There are three different options.

None -- with this option the REF input is ignored and when
you press the HOME button no movement happens, only the
DRO for that axis is reset.

Input = 0 -- with this setting EazyCNC expects that the
REF input is a logical zero (closed) when
the reference switch is active.

Input = 1 -- with this setting EazyCNC expects that the
REF input is a logical one (open) when
the reference switch is active.

Again you should not care if the signal is active or non-active,
zero or one, just make sure it works for you. If, when you press 
the 'HOME' button, the axis does not begin
to move towards the reference switch the setting of this
input is wrong. Note that you should first ensure that
DIR signal is correctly configures, see previous section.

When you press the 'HOME' button for an axis EazyCNC will drive that 
axis until it finds the home/reference position at which point
that axis DRO is automatically reset. 

The way this works when you press the 'HOME' button 
is that if the the REF input is active the axis is driven to
the positive axis direction until the signal becomes non-active. If the
the signal is non-active to begin with then the axis is driven 
in the negative direction until the REF signal becomes active and
then to the positive direction until it becomes non-active again.

This ensures that even though there is some backlash in the mechanism
or hysteresis in the switch the mechanism position will always be
correct.

You do not need to use a reference switch but by having one for
each axis allows the system to know its absolute physical position 
which in turn makes it possible for EazyCNC
to guard the movements against the physical limits of your system
preventing crashes.

Using reference switches it is also possible to continue machining
after a sudden loss of power because the absolute axis positions
can be re-covered by homing the axes.

\emph{\color{red} Note that the home or reference switch is no substitute for limit
switches that should be installed at each end of the movements
and wired to act on the emergency stop system}. 

The optimal placement for a reference switch is around the
middle of the axis travel, but this requires that the switch
is so configured that the REF signal is always on or off 
depending on which side of the switch the 'axis' is; it 
should not be possible to drive the axis 'beyond' the switch.

\subsubsection{Home -popup menu}

TOAD4 supports one home/reference position switch input for each axis. 

The Home popup menu controls weather the REF input on
the TOAD4 board is 1 or 0 when home/reference is switch is
active. 

\subsubsection{Pair X and A -checkbox}

This checkbox is only available in the X-Axis setup panel.

If this checkbox is ticked then the A-axis motor is used in
parallel with the X-axis allowing the use of two motors to
drive the X-axis. 

For jogging and machining the motors run in parallel making
equals amounts of steps. When you home the motors with the
HOME-button both motors start at the same time but find their
reference switch positions independently thus 'alligning' the 
motors.


\subsubsection{Steps/mm -entry field, Steps/inch -entry field}

This entry fields tells EazyCNC how many steps it takes
to move the axis a unit (mm or inch) length. We call
this value the step ratio.

Note that this is not an integral value and it should be entered
with as many significant digits as required  to achieve
the desired accuracy. As rule of thumb use at least six 
significant digits in calculations and entry to achieve 0.01 mm 
accuracy over 1000 mm axis movement range.

To calculate this value you need to know following:

\begin{itemize}
\item $mode$, a factor dependent on TOAD4 step mode
\item $steps$, the number of steps per revolution for the motor
\item $pitch$, the axis movement per motor revolution (including possible gearing)
\end{itemize}

Then you calculate the step ratio as follows:

\begin{equation}
\label{step-ratio} 
step_{ratio} =  \frac{mode *steps}{pitch}
\end{equation}

The step mode depends on the M1 and M2 jumpers for each motor on the TOAD4 board. 

See  Table~\ref{tab:jumper-labels} for the jumper labels for each motor
and Table~\ref{tab:step-modes} for the jumpers that need to be installed
to get the desired step mode and the mode value to use in Equation~\eqref{step-ratio}


%---------------------------------------------------------------------------
\begin{table}[H] 
\begin{threeparttable}[b]
\renewcommand\arraystretch{1}
\caption{Motors versus configuration jumpers} 
\label{tab:jumper-labels} 
\begin{tabular}{c c c p{8cm}} 
Motor & M2 & M1 &\\
\hline
X & U29 & U25\\
Y & U30 & U26\\
Z & U31 & U27\\
4 & U32 & U28\\
\end{tabular} 
\end{threeparttable}
\end{table}
%---------------------------------------------------------------------------


The driver chip supports four different step modes: full step, half step, fine step and micro step. Fine step provides eight intermediate sinusoidal current values for each full step and micro stepping provides sixteen intermediate current values.

Typically, micro stepping is preferred for its smooth ride, but sometimes speed requirements dictate the use of half or even full step.

%---------------------------------------------------------------------------
\begin{table}[H] 
\begin{threeparttable}[b]
\renewcommand\arraystretch{1}
\caption{Jumpers versus Step Mode} 
\label{tab:step-modes} 
\begin{tabular}{c c c  p{8cm}} 
M2 & M1 & mode & \\
\hline
- & - & 1 & Full Step (2 phase)\\
- & X & 2 & Half Step (1-2 phase)\\
X & X & 8 & Fine Step (2W1-2 phase)\\
X & - & 16 & Micro Step (4W1-2 phase)\\
\end{tabular} 

\begin{tablenotes} 
\item
\item '-' indicates no jumper installed
\item 'X' indicates jumper is installed
\item
\item (value) refers to TB6560 excitation mode, see data sheet for details
\end{tablenotes}

\end{threeparttable}
\end{table}
%---------------------------------------------------------------------------


Above may feel a bit complicated so an example maybe useful. 

Most stepper motors have 200 steps or step angle of 1.8\degree so we have:

\begin{equation*}
steps = 200 (steps/rev)
\end{equation*}

For this example we assume that we want to run X-axis motor at 'Half Step' 
mode so from Table~\ref{tab:step-modes} we see that we need to have jumper
M1 installed and from Table~\ref{tab:jumper-labels} we see that for X-motor
M1 jumper is labeled U25 (don't forget to see the errata for the TOAD4 board,
some of the jumper labels in the early board are wrong). 

While looking at Table~\ref{tab:step-modes} we also
note that 'mode' value for 'Micro Step' is 2 so we have:


\begin{equation*}
mode = 2
\end{equation*}

To make this more interesting and life-like let's suppose we use a lead screw
to move the X-axis and the screw has a pitch of 3 mm/revolution and that
we use a toothed belt to drive it with a 16 tooth
pulley on the motor axis and 45 pulley on the lead screw, so we have:

\begin{equation*}
pitch = 3 * \frac{16}{45} =  1.066666 (mm/rev)
\end{equation*}

Putting it all together we have

\begin{equation*}
step_{ratio} =  \frac{2 * 200}{ 1.0666666} = 375.000 (steps/mm)
\end{equation*}

Now would be good time to check how fast we can move the X-axis.

The maximum theoretical step rate is about 11 kHz, for jitter
and other reasons the maximum recommended pulse rate is about
$\frac{1}{5}$ of that, say 2000 pulses/sec. You need to divide
this by the $mode$ factor we looked up above so in our example
the maximum step rate is 1000 steps/sec and so our max speed
is

\begin{equation*}
\frac{1000}{375} \approx 2.6 mm/sec
\end{equation*}

Which is not very fast so maybe we should have 
geared up instead of down!


\begin{framed}
Here is a Top Tip!

Even though EazyCNC guides you if you try to enter too big (or small) value
to an entry field and even tells you what the limiting parameter is
and  further tells you what is the maximum value you can use, sometimes this can
turn into a bit of a chore.

So before you enter the step ratio, set the maximum jog acceleration and velocity
for all axes (Mach Setup / Axis Setup) and movement (Mach Setup / Movement) 
to a very small values, say 1 mm/sec or 0.01 inch/sec.
This will allow you to enter almost any step ratio, which is necessary as
the step ratio is a function of your gearing and thus won't budge.

Once you have entered the correct step ratios try to set the accelerations
and velocities to a very large values, say 1000 mm/sec or 10 inch/sec and EazyCNC will tell the
maximum possible values, so enter and use those.

\end{framed}


\subsection{Axis Limits -panel}

EazyCNC can guard movements against
set limits to prevent crashing the mechanism.

This is especially useful in Jogging where
it is too easy to run too fast to an end of
an axis.

However the limit checking is not fool proof
and it depends on the operators (that's you!)
diligence to work properly. If you for example
put a large cutter into the spindle chuck but
don't tell EazyCNC about it or move the axis manually
by turning handles or forgot to 'home' the axes
there is nothing EazyCNC can do about it.


A 'bad' a G-code move may still crash the machine
and you need to visually satisfy yourself using
the simulation mode that this will not happen.

Also worth remembering is that the limits checking
does nothing to prevent crashing against the
workpiece, fixtures or other obstacles.

The actual movement limits are based a fixed
coordinate system independent of all the different G-code
coordinate systems. The limits are expressed in
the current unit system unscaled and unaffected by
any G-code coordinate system transformations.


The origin of the limits coordinate system is at 
the home/ref switch position, so if the home/ref switch 
is not used you should not enable the limits because
the position is physically undefined. 

If you want to use the limits you must remember to 'home'
all axes by pressing the 'HOME' buttons if the 
TOAD4 has been power cycled (it is TOAD4 who maintains the 
coordinates so it will loose track of the position
if it is turned off).

Also worth remembering is that if you use the limits
you should have the motors energized at all times
otherwise the motors will 'lose' their positions,
so don't use the 'Auto Off' current mode.

\subsubsection{Limits Enabled -checkbox}
\label{subsubsec:enable-limist}

Limits Enabled -checkbox controls weather EazyCNC
enforces the limits for the axes or not.

\subsubsection{Min/Max - entry fields / Touch -buttons}

These entry field controls the minimum and maximum 
coordinates allowed for the axis, expressed in the 
current unit system. 

The easiest way to set the limit is to disable the
limit checking and carefully 'jog' the mechanisms
to each end of the movement and press the corresponding
'Touch' button which will then set corresponding
limit based on the current axis position

Remember to 'home' the axis before setting the
limits and don't forget to enable the limits once you have set them.

And don't forget to verify your limits!

\subsection{Jogging -panel} \label{jogging-panel}

Jogging refers to manually moving the axes with
either the 'jog buttons' or with the joystick.

The settings in this panel control the details
of jogging.

When you 'jog' an axis, the axis first moves
slowly i.e. crawls. This goes on for some
short time after which the movement accelerates
until it reaches the jog speed. Jogging then
continues at that speed as long as you keep
jogging or you hit the end of movement, after
which the speed decelerates to a halt.

In many CNC mechanisms the axes are not created
equal, some motors are by necessity stronger
than others
and thus the desired jogging characteristics
are different and you want to set them individually
for each axis.




\subsubsection{Crawl Velocity -entry field}

This entry field controls the initial ie
crawl speed of jogging, you typically want
to have this pretty slow.

\subsubsection{Crawl Length -entry field}

This entry field controls how long a distance
the crawl will go  until the acceleration
starts if you keep on jogging.


\subsubsection{Min Crawl -entry field} \label{crawl-velocity}

This entry field sets a minimum crawl distance,
ie the axis will always move at least this much
even if you jog very briefly.

\subsubsection{Acceleration -entry field}

This entry field control the acceleration/deceleration rate
of the movement, you probably want to have this as high
as possible but not so high that there is a risk
of stalling the motor or the motor skipping steps. 

Only experimentation can find a correct value for this,
start with a small value and first find the maximum
jog speed  before you try to maximize 
the acceleration.

EazyCNC has a motor test function, see  
section~\ref{subsec:test-panel}), to help you to 
determine the limits of your motors and mechanisms.

\subsubsection{Jog Velocity -entry field} \label{jog-velocity}

This entry field controls the jog or maximum speed
the axis will run when you jog it. You probably want
to have this set as high as possible but not so high 
that there is a risk of stalling the motor or the 
motor skipping steps. 

Only experimentation can find a correct value for this,
start with a small value and find the maximum at which
you can jog back and forth to a given DRO value 
watching that the tool tip hits exact same position
every time.


A manual way to check that the motor has not lost any
steps is to 'home' the axis and see that as the 
axis approaches the home position the DRO will not
suddenly jump to the reset value when the reference
switch is reached, this is not wholly accurate if the
motor only loses a small number of steps but typically
it is a all or nothing with stepper motors.

\subsection{Test -panel}
\label{subsec:test-panel}

The controls in this panel can be used to test and
find out the maximum acceleration and velocity for
an axis.

When you click the 'TEST' button EazyCNC will
perform a test movement on that motor/axis
and report the accuracy if you have a reference
switch installed. 

If you don't have a reference switch then you will
need to use a Dial Test Indicator or some such
to measure the accuracy.

\subsection{Pre-requisites}

\emph{Note that this test can damage your machine if not
performed carefully and as intended.}

The test run requires free travel of 25 mm or one inch plus
the distance is takes to accelerate from the start velocity
to the top velocity and back.

To use this test you must have the reference switch located
at least 30 mm away from the negative end of the axis movement.

If you don't have that 30 mm spare travel you run the risk of hitting the
the end of the axis movement range on the return leg of the
test run. It is acceptable to temporarily move the reference
switch to a suitable location for this test or use a temporary switch 
if you so desire.

If you don't have a reference switch then you need to ensure
that you start the test from a position on the axis from which
there is enough free travel on both sides of the starting position.


\subsection{The Test run}

When you click the the 'TEST button the test movement is performed as follows.

First EazyCNC 'homes' the  axis/motor ie it moves slowly 
towards the reference switch and then just out 
of it.

EazyCNC makes an internal note of this step position.

Next the axis/motor starts to move into the positive (or negative,
depending on which direction you have selected)
axis direction and accelerates at the given acceleration until
the motor reaches the given test speed. The movement continues
for 25 mm (about 1 inch) and then it
decelerates back to stand still. 

Then the axis is homed again and a note of the
step position at which the reference switch is
detected is noted down.

The difference between the two home positions noted down
is reported as the accuracy or repeatability of
the movement at the given acceleration and 
movement velocity.

A positive value indicates that steps were lost
on the way out i.e. during acceleration or high
speed movement. A negative value indicates that
steps were lost on the way back i.e. slow movement;
this should not really ever occur.

A small non zero value is acceptable or even expected 
as it is unlikely that  the system
can be absolutely accurate all things considered, but repeated tests at
given acceleration and velocity should show consistently
similar values.

To guarantee that the test is valid for machining
conditions the acceleration is performed step wise
at the current machine Update Period just as it
will when take place when executing G-codes. 

Therefore you need to remember that if you change 
the update period it is good to re-run the test
for each axis, especially if you are running close to the speed
and acceleration limits of your system.

%\subsubsection{Start Velocity -entry field}
%
%In this field enter the start velocity for the test, this should
%be a low speed at which the motor is more or less guaranteed
%to work without losing steps or stalling.

\subsubsection{Top Velocity -entry field}

This is the velocity you want to test for so keep
adjusting this and retesting 
until you have maxed out your system.

\subsubsection{Acceleration -entry field}

This is the acceleration you want to test for so keep
adjusting this and retesting 
until you have maxed out your system.


\subsubsection{Direction -popup menu}

This selects weather the test movement direction will be
along the positive direction or negative direction
from home/ref position.

This is the acceleration you want to test for so keep
adjusting this and retesting 
until you have maxed out your system.


\section{Setting the Kinematic limits}
\label{sec:setup-movement-limits}

The parameters in this screen Figure~\ref{fig:mach-setup-movement-view} 
tell EazyCNC how fast
it is safe to accelerate and run the motors, how often
the motor position and speed should be updated 
and how accurately you want EazyCNC to follow
the tool path described by the G-code file.



%---------------------------------------------------------------------------
\begin{figure}[htb]
\includegraphics[scale=0.35]{mach-setup-movement-view.png}
\caption{The movement setup screen}
\label{fig:mach-setup-movement-view}
\end{figure}
%---------------------------------------------------------------------------


Next to the individual axis/motor parameters these
are the most import parameters to carefully set as
if you set the speed and acceleration to too high values
the steppers will lose steps and the accuracy 
is ruined or the motors will completely stall.

On the other hand you will want to have the values
as high as reasonable so as not to waste time in
machining and with some cutters like plasma torches 
even the dimensions and quality of cut are dependent on
high enough speeds.

Note that these settings here set the maximum 
values, G-code programs specify the actual value
which can be lower or higher that what you specify
here. If the G-code specifies a lower value then that
applies but if the G-code specifies a higher value
then what you have set up here applies.

The only way to find out the maximum acceptable value
you can use is to try progressively higher
values and verify the accuracy of the 
motions for each trial.

Note that the values here are common to all axes. 
See section~\ref{subsec:test-panel}) on how to do this for
each axis. Once you have found out the maximum velocity
and acceleration for each axis, pick the minimum of those 
values and use that here as the maximum velocity and acceleration
for machining.


\subsection{Velocity -entry field}

Enter the minimum of the maximum velocities your motors
can handle.

Note that it is acceptable to use/have too high feed rate 
in G-code (the F-word) as EazyCNC will automatically
limit the feed rate to the maximum velocity you have
set here.

\subsection{Acceleration -entry field}

Enter the minimum of the maximum accelerations your motors
can handle.

\subsection{Path tolerance -entry field}
\label{sec:path-tolerance-entry-field}

This field tells EazyCNC how accurately during machining 
it should try to  follow the tool path described by the G-code 
program.

If you enter a value of zero here then the tool path is
accurately reproduced but this necessitates that the tool
comes to a complete stop between cutting movements (G-codes 
G1,G2 and G3).

This is because to have the tool
follow the path absolutely without stopping at corners would require infinite
acceleration and that strong motors are hard to find.

If you enter a non-zero value then EazyCNC will try to
follow the prescribed tool path to within the specified 
limit but using the leeway given by the tolerance to allow continuous movement
and not stopping between cuts. 

Rapid tool positioning (G0) always stops at the end of
the movement so this parameter does not apply.

If you use EazyCNC with a plasma torch  it is important to try to 
minimize the speed variations as the cut width depends
on the travel speed of the torch so use as large path
tolerance you can accept.

\subsection{Z-scaler -entry field}

Sometimes the Z-axis motor is different from the X and Y axis motors,
for example in a plasma cutting machine the X/Y movements need high acceleration
and velocities but the Z-movement is rather small and thus a smaller motor
that is not capable of such feats can be used. To prevent the system
from exceeding the Z-motor capabilities a Z-scaler value (smaller than one)
should be entered. This will effectively scale down the accelerations and
velocities for G-code movements that involve the Z-axis.

\subsection{Update Period -entry field}

Setting a suitable value for the Update Period is also
critical.

This value depends on the speed your computer. A faster
computer allows for a faster update period, however
there are limits on how fast it is acceptable to
update the speed and position into TOAD4.

USB limits the update period to a minimum value (maximum
update speed) of 1 msec, but typically you should aim
to a value of 10 to 20 msec on a modern PC hardware. Note
that this has nothing to do with step rate because
we are transferring position values to the TOAD4 and
the actual steps are generated on the TOAD4 board.

To understand how the update period affects things here
is brief description.

TOAD4 maintains a queue of movement commands so that
small pauses, interruptions or hiccups in the computer 
won't affect cutting movements. 

If TOAD4 runs out of movements commands ie the queues
run empty which happens if the computer does not
send new commands fast enough on average then the cutting 
movements will stop until more commands arrive.

At best this is not desirable as an interrupted cut can leave
a mark in the workpiece and  at worst the accuracy may be
lost if the movement was at such a high feed rate that
that the motor cannot be accurately stopped from such
speed.

The queue capacity is 16 commands, so an update period
of 20 msec means that there are commands for 16 x 20 msec
or 320 msec and the system can tolerate a pause, such
as Java garbage collection, for that length of time. So
longer period allows for longer pauses and hiccups in
the host computer system.

On the other hand TOAD4 can only change speed at the
interval of the update period so when the speed is
changing, like when the machine is cutting a circular
path, the speed is always partially 'wrong' for the duration of
the update period. 

So longer Update Period will result in an increased positional error,
which fortunately is not accumulative.

The upper theoretical
limit for such an error is  'feed rate x update period',
for example if you are cutting at 1200 mm/min which is 20 mm/sec
and your update period is 20 msec then the maximum error
caused by the update period is 20 mm/sec x 0.02 sec which
is 0.4 mm. This would probably be un-acceptable for milling
but such high feed rates when milling are rare and for plasma
cutting where such high feed rates are common this is in
the same ballpark as cutting accuracy anyway. Besides that is a worst
case error unlikely to happen in practice.

\section{Spindle setup}
\label{sec:spindle-setup}

%---------------------------------------------------------------------------
\begin{figure}[htb]
\includegraphics[scale=0.35]{mach-setup-spindle-view.png}
\caption{The spindle setup screen}
\label{fig:mach-setup-spindle-view}
\end{figure}
%---------------------------------------------------------------------------

Whenever the spindle has been turned ON with M3 or M4 -code TOAD4 will
output a voltage from the 'SPINDLE SPEED' output proportional to the S-word value.

The output voltage is relative to the voltage fed into the '+10 VREF IN' input
to the TOAD4 board, the intention is to grab that voltage from the VFD from
which it is typically readily available for just this purpose.

The output voltage is calculated as

\begin{equation}
Uout = \frac{Sword - MinSpeed}{MaxSpeed - MinSpeed}*Uref
\end{equation}

where 

\begin{equation*}
Sword = \textrm{The S-word value from the G-code program}
\end{equation*}

\begin{equation*}
Uout = \textrm{'SPINDLE SPEED' output voltage in TOAD4}
\end{equation*}

\begin{equation*}
Uref = \textrm{'+10 VREF IN' input voltage in TOAD4}
\end{equation*}

\begin{equation*}
MinSpeed = \textrm{Value entered into the 'Min Speed' entry field}
\end{equation*}

\begin{equation*}
MaxSpeed = \textrm{Value entered into the 'Max Speed' entry field}
\end{equation*}


if above should result in a value outside the range 0..100\% of Uref then
it is clamped to that range.

Typically a VFD has parameters that are used to set the minimum and
maximum spindle speed and it is precisely those same speed values that you 
should enter into EazyCNC here. Figure~\ref{fig:mach-setup-spindle-view}.

\subsection{Min Speed -entry field}

Enter into this field the rpm value which corresponds to the 0 Volt voltage at SPINDLE SPEED output and VFD speed input.
\subsection{Max Speed -entry field}

Enter into this field the rpm value which corresponds to the 10 Volt (100\%) voltage at SPINDLE SPEED output and VFD speed input.


\section{G-Code Options}
\label{sec:g-code-options}

G-code dates back to 1960s with the final revision RS274D approved in 1980. 
Over the years manufacturers of CNC system have added extension and variations to the standard.

EazyCNC tries to accommodate a common subset of the most popular systems
out there by allowing you to fine tune some details of the G-code interpretation to match your G-code program.


You can change them in the G-code screen, Figure~\ref{fig:mach-setup-g-code-view}.


%---------------------------------------------------------------------------
\begin{figure}[htb]
\includegraphics[scale=0.35]{mach-setup-g-code-view.png}
\caption{The G-code setup screen}
\label{fig:mach-setup-g-code-view}
\end{figure}
%---------------------------------------------------------------------------





For a complete description of G-codes supported by EazyCNC see Chapter~\ref{chap:g-code-chapter}.



\subsection{Incremental IJK -checkbox}
\label{sec:incremental-ijk-checkbox}

If this check box is ticked then the I,J and K words in the arc cutting G2 and G3 
commands are interpreted relative to the start point of the arc.

If you see a lot of large erroneous arcs in te tool path graphics panel when 
your are previewing your G-codes then you can be pretty confident that this
tick box is in the wrong state.

\subsection{Incremental XYZ -checkbox}

If this check box is ticked then the X,Y and Z words in the movement commands
G0,G1,G2 and G3 are interpreted relative to the previous X,Y or Z words/positions.

\subsection{G4 P in msec -checkbox}

If this check box is ticked then the P-word value in the G4 dwell command
is interpreted in milliseconds instead of seconds.

If the execution of G-codes seems to stop at G4 commands you can be pretty sure
that this check box is not ticked but should be.

\subsection{G41/G42 code options}

When tool compensation is turned on with either G41 or G42 G-codes the question arises 
how the tool should move in external corners.

Traditionally the tool moves arounds the corner in an arc.

The other option is for
the tool to move in a straigh line until it has 'cleared' the corner and the move
in a straight line along the next segment. If the (external) corner is very tight
then this would cause the tool to move very long away beyond the corner point, there
for a maximum length can be specified.

This later option maybe useful in plasma cutting as it takes the cutting flame further
away from the potentially sharp and narrow corner which tends to burn if the torch linger
around the corner too long.

\subsection{Use round join -radiobutton}

If this is selected then the traditional way of handling external corners is used, i.e.
the tool moves in an arc around the corner.

\subsection{Use miter join -radiobutton}

If this is selected then tool will move  in a straigh line until it has 'cleared' the corner and the move
in a straight line along the next segment. If, as a result of the corner geometry,
 the tool would move further from the corner than what
is specified in the 'bevel limi' -entry field, then the move is pruned as indicated.

\section{Shortcuts setup}
\label{sec:shortcut-setup}


EazyCNC is really designed to be used on a PC or tablet computer with a touch screen. 
However you can use most of the functions with convenient single key
keyboard shortcuts on a conventional keyboard or with a gamepad function keys
and joystick.

The key assignments are fully user definable so you can configure these as
you best please.

To examine or change the key assignments go to the Shortcuts setup screen,
Figure~\ref{fig:mach-setup-shortcuts-view}.

%---------------------------------------------------------------------------
\begin{figure}[htb]
\includegraphics[scale=0.35]{mach-setup-shortcuts-view.png}
\caption{The Shortcuts setup screen}
\label{fig:mach-setup-shortcuts-view}
\end{figure}
%---------------------------------------------------------------------------

On the left side column you see the keys and on the right side column the
corresponding assigned functionality.

To change the key, click on the left side column at the key you want to change; 
this will popup a dialog prompting you to press the new key you want to
assign for that functionality.

To change the functionality click on the right side column of the key
assignment you want to change and from the popup pick the functionality
you want to assign.

To create a totally new keyboard shortcut scroll to the bottom of
the list and click at the 'New Shortcut' button at the bottom of the
left column.

To delete a shortcut click on the left column on the shortcut you
want to delete and from the dialog that pops up select 'Delete'.

%---------------------------------------------------------------------------
\begin{figure}[htb]
\includegraphics[scale=0.35]{mach-setup-shortcuts-view-delete.png}
\caption{Deleting or re-defining a shortcut}
\label{fig:mach-setup-shortcuts-view-delete}
\end{figure}
%---------------------------------------------------------------------------





\section{Sys Info screen}

This screen displays bunch of system related information, 
Figure~\ref{fig:mach-setup-sys-info-view}.

Most of the information displayed is just to report back
to eazycnc@eazycnc.com if you are reporting a bug.

The two pieces of information that are useful for you
are the 'EazyCNC version' to identify which
version you are running (it is also shown in the window title
if that is visible) and 'EazyCNC parameter' which tells you
the location of the configuration file in case you have doubts
about which configuration file is being used.

%---------------------------------------------------------------------------
\begin{figure}[htb]
\includegraphics[scale=0.35]{mach-setup-sys-info-view.png}
\caption{The Diagnostics screen}
\label{fig:mach-setup-sys-info-view}
\end{figure}
%---------------------------------------------------------------------------

%\section{Diagnostics screen}

%In addition to displaying a bunch of system related information 
%the Diagnostics screen, Figure~\ref{fig:mach-setup-diagnostics-view}, can be
%used to get insight into how consistently EazyCNC is able to communicate
%with the TOAD4 controller.

%When the controller is connected (the 'MACH' button is active) the histogram
%shows in real time the distribution of actual update periods.

%The middle of the histogram represents the current 'Update Period' so
%the access times should consistently pile up slightly to the right
%of that. When EazyCNC is executing the G-code program the update speed
%is actually double so the pile should be concentrated on the left side
%of the histogram.

%If the update times show up all over the place then there is something
%in your computer setup disturbing EazyCNC, possibly an other program
%or process running in the background.

%As time goes by the histogram represents a longer and longer history
%and thus new information makes little difference to it, that is why
%you should clear the histogram by pressing the 'Reset' button to
%get fresh look at the distribution of update periods.

%---------------------------------------------------------------------------
%\begin{figure}[htb]
%\includegraphics[scale=0.35]{mach-setup-diagnostics-view.png}
%\caption{The Diagnostics screen}
%\label{fig:mach-setup-diagnostics-view}
%\end{figure}
%---------------------------------------------------------------------------
