\chapter{Introduction to CNC-Machining}



\section{Overview of a CNC Machining System}


This is probably familiar territory for you otherwise you would
not be here in  the first place but
this section is short introduction  to tell you where exactly EazyCNC
and TOAD4 fits in the big picture.

%Figure~\ref{fig:cnc-system-overview} tries to capture the 'big' picture of a CNC machining system.


%---------------------------------------------------------------------------
\begin{figure}[htb]
%\includegraphics[scale=0.4]{cnc-system-overview.png}
\caption{a CNC Machining System Overview}
\label{fig:cnc-system-overview}
\end{figure}
%---------------------------------------------------------------------------

The CNC machining process starts with a design of the part to be machined
which is turned into a sequence of instructions to the computer that
controls the motors, typically stepper motors, that move the cutting tool 
(or work piece) via series of gears, belts, pulleys and/or screws. 
These tool movements are typically called 'axes', for example X-axis, Y-axis etc.

The 'sequence of instructions' is called G-code and it is basically a text
file with coordinate points that define the path the cutting tool will make.

G-code can be hand written but is typically generated automatically from
a CAD (Computer Aided Design model) of the part using CAM (Computer Aided Manufacturing)  
software, either directly by the CAD/CAM program or by a program called 
post-processor.

The G-code file is read by a program that turns the coordinate information
and other commands in the G-code file into motor control pulses in real-time
observing programmed feed rates and machine parameters such as number of pulses
required to move a unit distance.

This is where EazyCNC/TOAD4 comes into picture because EazyCNC is the program that
does the G-code interpretation and TOAD4 is a micro controller that does the
real-time motor control.

\section{Understanding Real-Time}

In common language real time means roughly 'as it happens' but in 
computer jargon real-time has a specific and important meaning.

Real-time here means that the pulses that control the distance and speed
of movements need to be generated precisely at appropriate rate because
the motors and mechanisms that move the axis have physical limits beyond 
which they fail to move as required. 

EazyCNC runs an a personal computer  such as an IBM PC compatible 
or a Macintosh computer. These computers use an operating systems,
such as Windows, Linux or Mac OS X, that are not ideally suited to
real-time control. You can easily get a feel for this when you plug in 
a USB-device as often the mouse cursor stops for a second or two -- imaging
if the system paused like that when it should be turning a corner.

There are different ways out of this difficulty. A Windows program called
Mach3 uses a special 'driver' software for the real-time stuff and another
program called EMC2 uses a specially 'patched' version of the Linux 
operating system just to mention two.

EazyCNC takes a different approach.

EacyCNC delegates the most demanding real-time tasks to the TOAD4
micro controller which is better equipped and positioned to do the 
precise real-time generation of motor control pulses and such because
it does not have to deal with the endless variety of the PC hardware
and software and because it has been designed from ground up for 
the very task of performing real-time control.

EazyCNC reads and interprets the G-code and breaks it into bite size
chunks that the simple micro controller in the TOAD4 can process in real-time.

These bite size chunks are are transferred from the PC to TOAD4 over
the USB bus and put into a command queue in the TOAD4 micro controller.

EazyCNC attempts to keep the queue full at all times so that if EazyCNC
or the operating system it runs on needs to 'pause' for a second or so
there is still data for TOAD4 to process and machining can continue 
without missing a beat.

This is important because when stepper motors are run at high speed
a delayed pulse is equivalent to a sudden deceleration which may
cause the system to exceed the stepper motor's max torque in which case
the stepper motor will not be able to 'hold' its position and accuracy
is compromised. 


\section{Understanding Stepper Motors}

While EazyCNC and TOAD4 can be used with Servo Motors they really
are meant to be used with Stepper Motors.

Stepper Motor are motors with two or more stationary coils and
a rotating permanent magnet rotor. By energizing the coils
in sequence the rotor can be made to move.

Stepper Motors have some interesting and important properties.

First of all they do not 'run' if you just energize them, at most
they make a single small movement called step. This makes them
rather safe as a short circuited transistor in the drive system
cannot make the motor run wildly.

Secondly when you apply a controlled energizing sequence into
the coils the motor makes precise fractional rotational movements 
called steps, a typical stepper motor step is 1.8\degree or 
200 steps/revolution.

It is this second property that makes steppers very attractive
for controlling precise movements. 

If you 'step' a stepper motor ten times you can be pretty sure
that it actually moved ten steps. So there is no need to
measure position of the motor in any other way than counting
the pulses we send to it, no need for expensive encoders and 
or position scales. 

This makes stepper motors very cost effective
way to implement motion control.

But the lack of position feedback is also the downside of
stepper motors.

The positional control of a stepper motor based system
relies on an initial position and keeping track of the steps
and their direction.

The initial position can be either given manually or
found automatically by using a reference switch.

Manually means that you move the motor/axis to a 
know position, such as to the end of the movement range,
and tell the system that this is it.

If a reference switch is available then the system
can move the axis/motor through its range of movements
and make a note of the step number on which the reference
switch is reached and in this way calibrate its position.

Keeping track of the pulse and their direction
is done automatically and precisely by the software
but under certain circumstances the motor
cannot 'carry out the step' and the system looses
track of the real physical position, this is called
missing steps.

Missing steps can happen if the stepper is stepped too 
fast or the load exceeds what the motor can deliver. To
prevent that the correct maximum speed and acceleration 
parameters need to be programmed into the system.

Also note that manually forcing the axes to move, if
you manage to overpower the motors, will cause the
system to loose its position, so all manual movements
must be done via the 'jog' controls of the system.

It must also be mentioned that since we are controlling
the motor position, not the cutter position, any backlash 
or play in the mechanism is \emph{not} automatically compensated for.

Typically not of practical concern but good to know
is the fact that stepper motors are not very 'stiff'. Even
though the motor has specific torque it actually has very
little torque when the magnetic poles of the coil and
rotor are aligned i.e. at every full step. 

You can think about this as if the rotor and stator poles were
connected with springs; when the poles are aligned the
spring will not pull the rotor one way or the other,
they only exert force and torque when the rotor is moved
into misalignment and thus there is almost always a small
but measurable error between the physical and ideal step
position.

There is of course a lot more to know and understand
about Stepper Motors but above is the most import
thing to keep in mind when working with systems based on
them.






