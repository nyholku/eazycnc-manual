\chapter{Safety First!}


Machine tools are dangerous!

Always keep that in mind, both when designing and setting up your system and when
operating it on a daily bases.

CNC machine tools are heavy and strong machinery, moving sharp and hot cutting tools 
or extremely powerful plasma torches under computer control. Computers are complex systems
and it is \emph{impossible} to ensure 100\% error free and safe operation in
every situation. It is perfectly possible that a software design flaw, called bug,
cause the system to operate unexpectedly or even run away wild.

Therefore it is very important to take appropriate precautions for such an eventuality.

Every system needs to have an Emergency switch fitted. 

The emergency switch needs to be so wired that it will prevent any machine
movement and stops spindle or shuts down the torch arc when activated.

The emergency switch needs to be mounted to a place that is easily accessible when operating the machine.

The emergency switch has to be of the latching kind in other words: once
activated it must stay activated until manually de-activated.

Depending on the physical layout and power of your machine movements you need to 
consider if you should activate the emergency switch whenever you have your hands or limbs  
inside the working area of your machine.

With some machine configurations it may be preferable not to activate the
emergency switch if you need to pause the system in the middle of 
machining, for example to change the tool bit because the axes might
lose their positions and it may be acceptable to just ensure that the spindle
will not start on its own. 

For that purpose a kill switch to the spindle
motor controller maybe fitted that will prevent the spindle from running
no matter what the control systems does.

Above does not by any means endorse any particular way of ensuring safety and
no responsibility or liability is accepted by me. You need to do your own risk
and safety assessment and act accordingly.



