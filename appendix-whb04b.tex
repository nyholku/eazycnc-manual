
\section{XHC WHB04B Pendant / MPG} 
\begin{figure}[htb]
\includegraphics[scale=0.07]{whb04b.jpg}
\caption{WHB04B Pendant / MPG}
\label{fig:whb04b}
\end{figure}

EazyCNC supports a commercially available MPG named WHB04B 
(note the B at then end of the type code), 
see Figure~\ref{fig:whb04b}.  This actually comes in two varieties, 
WHB04B-4 for use with four axis and WHB04B-6 for use with six axis.
    
Out of the box this pendant (or rather EazyCNC) support a
much smaller set of functionality than for example the WHB04.
This is because the pendant has a smaller number of buttons
and even smaller number of usefully labeled buttons.

\subsection{WHB04B Controls}


The three main controls of the MPG are the axis selector knob,
step size selector knob and the large pulse wheel.

In general the axis selector selects which axis is affected
by the pulse wheel or buttons on the keypad. 
 
The step size selector change how fast the wheel moves the axis.

The keypad is used to activate EazyCNC functions.

\subsection{Display}
\begin{figure}[htb]
\includegraphics[scale=0.2]{whb04b-display.jpg}
\caption{WHB04B Display}
\label{fig:whb04b-display}
\end{figure}

Figure~\ref{fig:whb04b-display} shows a close up of the WHB04B display.

The display has DROs for three axes at a time. If the axis selector
is in X,Y or Z position then the corresponding axis values are 
displayed in the DROs. If the selector is in A position then axis positions
of the A, B and C axis are displayed.

An asterisk ('*') in front of an axis letter indicates that that axis has
is currently selected by the Axis Selector knob.
    
The DROs  display the same DRO values as on the EazyCNC computer screen.
    
Note that you can NOT see more decimals on the pendant DRO than on the computer screen 
eventhough the pendant always displays four decimal digits.
    
Note that if the Axis Selector is in the OFF position then EazyCNC
is not able to update the display at all.

On the top row alternatively eithe step selector percentage is displayed
or feed  ('F') or spindle ('S') is displayed. The percentage makes
little sense but can be used as a reminder of the selected wheel step size.

To display the feed or spindle speed you need to press the Feed +/- or 
Spindle +/- key. If the corrosponding feed or spindle speed is not 
currently displaying when you press the key then the display will switch
and the key is ignored, i.e. it will not change the feed or spindle speed.

When the pendant is powered up you may see the text 'RESET' on the top row.
To get rid of that you have to press any button on the key pad and rotate
the step selector knob. This is makes no sense but it is what it is
and EazyCNC cannot do anything about that.




\subsection{Keypad}
\begin{figure}[htb]
\includegraphics[scale=0.2]{whb04b-keypad.jpg}
\caption{WHB04B Keypad}
\label{fig:whb04b-keypad}
\end{figure}

Figure~\ref{fig:whb04b-keypad} shows a close up of the WHB04B keypad.

The keypad has 15 keys plus the orange 'Fn' key which alter the
functions of those keys that have an orange label on them. To 
activate the 'orange' function you need to hold the 'Fn' key down
while pressing the intended key.

Note that the functions associated with the keys are not fixed
and you can change them
in the Mach Setup / Shortcuts screen, see section \ref{sec:shortcut-setup},

For inspiration of functions you could assign to the keys I suggest
reading the WHB04 appendix which lists number of useful functions
that are available out-of-the-box with that pendant.



\subsection{Axis Selector}
\begin{figure}[htb]
\includegraphics[scale=0.2]{whb04b-axis-knob.jpg}
\caption{WHB04B Axis Selector }
\label{fig:whb04b-axis-knob}
\end{figure}



\subsection{WHB04 Wheel}

Every 'click' of the wheel moves the selected axis one step to the wheel direction.
The speed at which the mill table,head or plasma torch moves is relative to how fast you
turn the wheel. 
    
In principle the wheel allows absolute (well incremental really) control of 
the axis position and speed. In practice the wheel sometimes misses pulses 
so turning the wheel ten clicks may only result in nine steps, always
confirm by looking at the DROs.
    
Because it is possible to turn the wheel faster than the axis can move there
is a windup prevention that prevents the wheel position from advancing too 
much beyond where the axis has advanced.
    
Because of the long chain of hardware and software that connects the pulse wheel
to the axis stepper motor, the feel of control you get with the wheel is far from 
perfect. However it does allow you to control the axis position rather
swiftly and accurately.

The step size that is used when you turn the wheel depends on the Step Selector, 
see below.
    
\subsection{Step Selector}
\begin{figure}[htb]
\includegraphics[scale=0.2]{whb04b-step-knob.jpg}
\caption{WHB04B Step Selector}
\label{fig:whb04b-step-knob}
\end{figure}

The Step Selector has some confusing labeling, you should go by the
labels in white.

Depending on weather you are working with  inches or millimeters 
(see Mach Setup / Screen / Units)
the wheel click/step size varies as shown in the table below. 

\begin{tabular}{ r | l | l }
Selector position& Units = mm & Units = inch\\
\hline
0.001 / 2\% &  0.001 mm  & 0.001" (0.0254 mm)\\
0.01 / 5\% &  0.01 mm  & 0.001" (0.254 mm)\\
0.1 / 10\% &  0.1 mm  & 0.001" (2.54 mm)\\
1.0 / 30\% &  1.0 mm  & minimal\\
LEAD\% &  minimal  & minimal\\
\end{tabular}
    
In above 'minimal' indicates smallest possible step, for example
if in your Axis Setup / Step/mm you have 400 step/mm then 
the minimal step size is 1/400 mm i.e. 0.0025 mm.

Note that if theoretical step size as per above table results
in a step that is smaller than 'minimal' then the minimal step is 
selected.

For safety the "1.0" selector position does not result in one 
inch step size in inch mode. 

\seticons{whb04b-reset}
\subsection{Reset -Key}   
                                  
Pressing this has the same effect as pressing HOME button on the computer 
screen.

Use the axis selector knob to select the affected axis, or set the
axis selector knob to 'Off' to HOME ALL axis.

A long press will force HOME ALL action regardless of the axis selector knob position.


\seticons{whb04b-stop}
\subsection{Stop -Key}  


This key has the same effect as pressing the STOP button on the computer screen.
 
\seticons{whb04b-start-pause}
\subsection{Start / Run -key}   
 
This key has the same effect as pressing alternatively the RUN and HOLD buttons
on the computer screen.
   


\seticons{whb04b-feed+}
\subsection{Macro-1 [Feed+]-key} 

If the 'F' is not displayed on the pendant display then the key press 
is ignored and the display changes to display the feed speed.

When pressed together with the 'Fn' key this button has the same effect
as pressing the '+' key in the Feed/Override panel on the screen.

When pressed without  the 'Fn' key this key has no function unless you
assign one to it.
    

\seticons{whb04b-feed-}
\subsection{Macro-2 [Feed-]-key}   


If the 'F' is not displayed on the pendant display then the key press 
is ignored and the display changes to display the feed speed.

When pressed together with the 'Fn' key this button has the same effect
as pressing the '-' key in the Feed/Override panel on the screen.

When pressed without  the 'Fn' key this key has no function unless you
assign one to it.
  
\seticons{whb04b-spindle+}
\subsection{Macro-3 [Spindle+] -key}   


If the 'S' is not displayed on the pendant display then the key press 
is ignored and the display changes to display the spindle speed.

When pressed together with the 'Fn' key this button has the same effect
as pressing the '+' key in the Spindle panel on the screen.

When pressed without  the 'Fn' key this key has no function unless you
assign one to it.
  

\seticons{whb04b-spindle-}
\subsection{Macro-4 [Spindle-]-key}  


If the 'S' is not displayed on the pendant display then the key press 
is ignored and the display changes to display the spindle speed.

When pressed together with the 'Fn' key this button has the same effect
as pressing the '-' key in the Spindle panel on the screen.

When pressed without  the 'Fn' key this key has no function unless you
assign one to it.

\seticons{whb04b-m-home}
\subsection{Macro-5 [M-HOME]-key}   

This key has no function unless you assign one to it.

\seticons{whb04b-safe-z}
\subsection{Macro-6 [Safe-Z]-key}    
    
                                              
Pressing this key will cause the Z-axis to move to the  Safe Z value set in the Axis Setup / Axis Z.


\seticons{whb04b-w-home}
\subsection{Macro-7 [W-HOME]-key}

Pressing this key will cause the X and Y axis to jog to the zero
position of the DROs.

    
\seticons{whb04b-s-on-off}
\subsection{Macro-8 [S-ON/OFF]-key}  

Pressing this key has the same effect as pressing the SPINDLE button on 
the computer screen. 

In other words it toggles the spindle on and off.


\seticons{whb04b-fn}
\subsection{Fn -key}       
 
This key together with some other is used to evoke the secondary 
function of that key. To use that press and hold the 'Fn' key down
while pressing the other key.
    
\seticons{whb04b-probe-z}
\subsection{Macro-9 [Probe-Z]-key}   


Pressing this key has the same effect as 
pressing the Touch button in the Work Offsets / Set Z origin screen when the 
'Use PROBE to Touch' feature is enabled.
    
In other word this will perform a short probing move on the Z axis 
axis and as soon as the probe trips the movement stops and rectracts 
and the Z axis work offset is set to the Gage Height parameter in 
the Work Offsets screen.
    
Thus this is a handy way to perform 'zero' the Z-axis with the probe.
 
\seticons{whb04b-macro-10}
\subsection{Macro-10}   


This key has no function unless you assign one to it.

\seticons{whb04b-continuous}
\subsection{Continuous -key}  


This key has no function unless you assign one to it.

\seticons{whb04b-step}
\subsection{Step -key}       

This key has no function unless you assign one to it.
