\chapter{Operating Your CNC Machine}
\label{chap:operating}

This chapter tells you all how to operate EazyCNC to machine stuff with G-code.

G-code is covered in more detail is Chapter~\ref{chap:g-code-chapter}, but since the sole purpose
of EazyCNC is G-code interpretation most functionality is related to 
G-code in some way I cover that in this chapter.


\section{Using Keyboard and Joystick}
\label{sec:keyboard-and-joystick}

Most of the EazyCNC functions needed to operate your CNC machine can be activated
from the keyboard with single key from the computer's keyboard. 

A Joystick or Gamepad can also be used to active most of the functions.

To extend the number of functions accessible using the Joystick/Gamepad buttons
EazyCNC supports a software 'Shift Lock' -button, Figure~\ref{fig:shift-lock}.
If this button is activate then the Joystick/Keyboard generates an other set
of keys onto which functions can be attached.

See Section~\ref{sec:shortcut-setup} to learn how to view and change both keyboard 
and joystick shortcuts.

The default keyboard / shortcuts assignments are given in Table~\ref{tab:keyboard-short-cuts}.

\begin{table}[!ht] 
\begin{center} 
\begin{threeparttable} 
\caption{Keyboard and Joystick Shortcuts}
\label{tab:keyboard-short-cuts}
\centering
\begin{tabular}{l l}
\hline\hline
Key & Function\\ [0.5ex] 
\hline
R& RUN\\
H& HOLD\\
S& STOP\\
Space&Toggle RUN/HOLD\\
C&Toggle Coolant ON/OFF\\
T&Toggle Spindle ON/OFF\\
Cursor LEFT& Jog -X\\
Cursor RIGHT& Jog +X\\
Cursor DOWN& Jog -Y\\
Cursor UP& Jog +Y\\
Stick LEFT& Jog -X\\
Stick RIGHT& Jog +X\\
Stick DOWN& Jog -Y\\
Stick UP& Jog +Y\\
Button 1& RUN\\
Button 2& HOLD\\
Button 3& STOP\\
Alt-Cmd S& Save Configuration (Mac OS X)\\
Alt-Ctrl S& Save Configuration (Windows/Linux)\\
F12&Toggle Java console on/off\\
\hline
\end{tabular}
\end{threeparttable}
\end{center}
\end{table}



Joystick Figure~\ref{fig:shift-lock}.

%---------------------------------------------------------------------------
\begin{figure}[htb]
\includegraphics[scale=0.8]{shift-lock.png}
\caption{The Shift Lock -button}
\label{fig:shift-lock}
\end{figure}
%---------------------------------------------------------------------------




\section{Simulation versus Cutting Metal}

EazyCNC can be in one of two operating modes, simulation mode or machining mode.

The simulation mode is handy for training or experimentation and for 
G-code verification before you actually
machine anything. Because no TOAD4 motor controller is used or needed in the simulation
mode you can install and run the EazyCNC software in any computer and use it
in the comfort of your study or office.

The simulation mode attempts to be step-accurate i.e. simulate the exact movements
the stepper motors will make and it runs in real time (if your computer is fast enough)
meaning that simulation takes as long as the actual machining will take. As this
may take a long time you may speed things up when simulating by using a faster feedrate (the F-word
G-code) but this will affect the small details of the cutting paths so the path
generated at high speed is not 100\% same as the lower speed path.


You control the operating mode with SIMU and MACH buttons at the bottom of the screen,
Figure~\ref{fig:simu-mach}, the current operating mode is show with the
button lit.

To toggle a mode on or off click the corresponding button.

%---------------------------------------------------------------------------
\begin{figure}[htb]
\includegraphics[scale=0.8]{simu-mach.png}
\caption{The operating mode control and indicator buttons}
\label{fig:simu-mach}
\end{figure}
%---------------------------------------------------------------------------








\section{Status Display}
\label{status-display}

At top of the panel titled 'Machine Controls' at the bottom of the screen,
Figure~\ref{fig:status-display}, there is a status/error display.

%---------------------------------------------------------------------------
\begin{figure}[htb]
\includegraphics[scale=0.8]{status-display.png}
\caption{The status/error display}
\label{fig:status-display}
\end{figure}
%---------------------------------------------------------------------------

This display shows the general status of the system or a helpful error message in
case of an error.

Most errors are transient in nature because they are the results of an operator 
(that's you) mistake, such as pressing the right button at the wrong time or entering
an invalid value into an entry field, thus
the error messages as just shown briefly accompanied with a beep sound.

If you are not able to read the error/message quickly enough you can press and hold 
the '?' button down to re-call the last error message and read it at leisure.

During machining it is possible that the G-code contains messages to the operator,
via the '(MSG, operator message)' -mechanism,
and those messages are also displayed in this display.

\section{Interactive Execution of G-code}

The status display line doubles as an interactive G-code execution entry field. 
To execute G-code interactively click on the status line, type in G-code and
hit ENTER-key. 

This feature should be used with extreme caution as it very easy to enter
G-code commands that damage the machine or the workpiece. 

Even more care
and forethought needs to be exercised if the interactive execution is used
in the middle of machining when the system is in the HOLD-state, because the
the interactive commands may violate the expectations built into the G-code.

For example it is possible to turn off the spindle interactively with the 
'\texttt{M5}'  code in which case continuing with the machining is likely
to result in a broken milling cutter when the machining is continued
and the spindle is not running.

\section{G-code display}

On the main screen the top right hand quarter of the screen is occupied by the
G-code display and the associated controls, Figure~\ref{fig:g-code-display}.

%---------------------------------------------------------------------------
\begin{figure}[htb]
\includegraphics[scale=0.8]{g-code-display.png}
\caption{The G-code editor and display panel}
\label{fig:g-code-display}
\end{figure}
%---------------------------------------------------------------------------

The current G-code file is displayed there. Only one G-code program can
be open at any given time, but if it calls subroutines in an other
G-code file it is possible that multiple 'tabs' or files are displayed there.

The next G-code line to be executed is highlighted in blue. If there is
an error in the G-code then the line containing the error is highlighted
in red and the associated error message is displayed in the status/error
display.

The background of the G-code display turns green if the system is in the
RUN or HOLD state, indicating that the system is either machining or 
ready to continue machining.

\subsection{Loading G-code for execution}

In order to execute a G-code program you need to load it into the EazyCNC
first, to do that, click 'Open' button and select the G-code file to load.
G-code files are pure text (7-bit ASCII) and so often have a '.txt'
extension but that is not mandatory, some people and systems use '.nc' .

\subsection{The Goto -button}

G-code programs are meant to be executed from the beginning to the end. However
it is possible to start from somewhere else, for example if you
need to continue execution after a blackout, even if that is not recommended.

To start the machining from somewhere else than the current
line highlighted in blue, click on any line and then click the 'Goto' button,
this will move the highlight to that line and machine will enter the 'HOLD' -state.

The caveat is that because
you are skipping G-codes when you 'Goto' a specific line in the program
 you maybe breaking assumptions
the G-code generator, be it a person or a CAM software, made
when the code was written and this may have consequences. 

For example
a plasma cutter is typically programmed to start the cut with a piercing
burn outside the actual part outline and if you skip that the
arc may not be established and the metal not be pierced or the piercing may
leave its mark on the part.


\subsection{Editing G-code}

It is also possible to edit and even create G-code files directly in 
EazyCNC, but this is not recommended. EazyCNC is not a general purpose
text or G-code editor, anytime you edit the code EazyCNC will go through
the whole code and re-calculate the tool path and this can be slow,
especially if the G-code file is very long, as it very well can be.

To edit the code ensure that EazyCNC is in the stopped state (you 
cannot edit the code while machining)  and just click on line you
want to edit, most standard basic text editing facilities work as
you would expect.

To create a new G-code file from scratch click the 'New' button.

An asterisk, '*', after a file name indicates that the file contains
unsaved changes, to save the changes to the actual file click the 'Save'
button.



\section{Toolpath display}
\label{toolpath-display}

On the main screen the top lefthand quarter of the screen is occupied by the
toolpath display and the associated controls, Figure~\ref{fig:toolpath-display}.

%---------------------------------------------------------------------------
\begin{figure}[htb]
\includegraphics[scale=0.4]{toolpath-display.png}
\caption{The toolpath display panel}
\label{fig:toolpath-display}
\end{figure}
%---------------------------------------------------------------------------

In this panel the toolpath that the currently loaded G-code represents is illustrated in
a three dimensional view.

The movement limits of the axes are illustrated as grids and the corresponding positive axis directions
are displayed in corresponding colors. You can change the colors, see section~\ref{subsec:axis-color-settings}.

The cutting tool is illustrated as a purple cylinder and the actual tool path is displayed 
as series of connected red and green lines of different intensity and line width. The colors
can be changed, see section ~\ref{subsec:tool-color-settings}.

The display is updated in real time as the machining progresses. The toolpath machined or
cut so far is displayed in green and the yet to be cut toolpath is displayed in red. You
can change the colors, see section ~\ref{subsec:machined-path-color-settings}.

For the already machined toolpath, in green, the width of the line corresponds to the 
width of the tool used to cut that path if the spindle was turned 'on' during the
machining, otherwise the toolpath is shown as a thin line.The colors
can be changed, see section ~\ref{subsec:machined-path-color-settings}.

Weather the spindle is 'on' or 'off' for a given toolpath segment is indicated by
the intensity of the color of the line, spindle 'on' is illustrated with with
bright red or green and spindle 'off' is illustrated with a darker shade.



When the G-code program is executing or running you can see, if you look carefully,
that the green toolpath appears to precede the purple tool, this actually accurately
represents the fact that the movement commands are queued in the motor controller
waiting to be executed.

 

\subsection{Controlling the toolpath display}

If the 'Track' button is not active you can move around the three dimensional 
virtual space and inspect different parts of the toolpath by clicking and dragging 
in the toolpath display panel. If the 'Track' button is active the 
display is always automatically cantered at the tooltip and follows it
as the tool moves. You can toggle the tracking on and off by clicking the button
repeatedly.

If you hold down the 'ALT' key on your keyboard while you drag with
the mouse you can rotate the view. If one of the 'XY', 'YZ' or 'XZ' buttons is
active then the rotation is disabled and the view is forced to be towards
the corresponding coordinate plane along the 'plane normal' axis. For example if
you have the XY button selected then the view is as if you were looking
down from the top of the positive Z-axis towards the XY-plane. 

Note that you can de-select any of the 'Track', 'XY', 'YZ' or 'XZ' by
just re-clicking on it.

You can use the '+' and '-' buttons to zoom in and out to see more or
less of the toolpath or you can use the mouse scroll wheel.


\section{Coordinate displays aka DROs}

The coordinate displays, Figure~\ref{fig:dro-displays}, also known as
Digital Readouts or DROs, show the tool position in the currently
active coordinate system in the currently selected units.

The coordinate system can be quite complex but for most practical purpose 
you can just think that the DROs just show the position EazyCNC thinks 
the tool tip is at the moment. If the DROs says that the tool X-axis
is at position 100 and the G0 -code tells it to move to X150 then
the tool will travel 50 units to the right, no matter where it
physically actually was.


By clicking at the DRO and typing in a number you can change the
displayed value. This will not move the tool but will of course
change how subsequent G-code coordinates are interpreted. 

You can also zero the DRO by clicking at the 'ZERO' button.

If you press the 'HOME' button and you have a home/reference switch
installed for that axis EazyCNC will drive the axis to the home
switch position and then zero the DRO automatically. This establishes
a repeatable absolute  origin for that axis.







%---------------------------------------------------------------------------
\begin{figure}[htb]
\includegraphics[scale=0.8]{dro-displays.png}
\caption{The Digital Readouts}
\label{fig:dro-displays}
\end{figure}
%---------------------------------------------------------------------------

\section{Jogging}

The Jog buttons, Figure~\ref{fig:jog-controls} can be used to drive or move
the motors/axes 'manually'. If you need to move the axes you should use
the jog buttons and not any handles your machine possibly has for moving
the axes, as if you use the manual handles EazyCNC has no way of knowing that
the tool has moved and thus coordinate positions used by the system will be wrong.

The buttons to the right of the jog arrow buttons set the jogging mode i.e.
they control how the jogging is performed.

Never turn off the power to TOAD4 to move the axes manually, always use one of the
jog modes provided.


See  \ref{jogging-panel} for details how to setup the speeds used in jogging.


\subsection{FAST -button}

If the FAST-button is selected then pressing and holding a jog button will first move that axis slowly and then rapidly
accelerate to the full jog speed. Pressing a jog button briefly will
always move the axis a predetermined, minimal distance. 

The 'slow' speed is specified in the 'Crawl veloc.' entry field for that axis in
the Axis Setup -screen. The 'fast' speed is specified in the 'Jog veloc.' entry field and 
the acceleration is controlled by the 'Accel.' entry field.

\subsection{SLOW -button}

If the SLOW-button is selected then pressing a jog button will move that axis slowly a
as long as the button is pressed.

The 'slow' speed is specified in the 'Crawl veloc.' entry field for that axis in
the Axis Setup -screen.

\subsection{STEP -button}

If the STEP-button is selected then pressing a jog button will move that axis 
for by the amount specified in the 'Min. crawl' entry field for that axis in
the Axis Setup -screen.

\subsection{MANUAL -button}

If this button is selected then the motor drivers in TOAD4 are disable i.e.
the motor currents are turned off and the axes can be moved manually with
the hand wheels of the machine. 

This mode of jogging is intended to be used before or after machining to 
allow the operator to adjust the axes. This is necessary for example when using 
a feeler to touch a work piece before setting the coordinates.

When the machine axes are moved with the hand wheels EazyCNC
has no way of knowing how much the axis has moved and thus the coordinates
will be off/erroneous unless you reset them after moving the axes manually.

Note that when the motor drivers are turned off the axes are free to move
and thus even small forces, perhaps even internal tensions of the machine or
tool/workpiece contact may move the axes a small amount and thus ruin the
accuracy of the coordinates. So think twice before pressing this button.

Important, never ever turn off the TOAD4 to 'free' the axes if you need to move
them manually, always use this jog mode if you want to use the hand wheels
to move the axes.



%TODO "SPECIAL" button

%---------------------------------------------------------------------------
\begin{figure}[htb]
\includegraphics[scale=0.8]{jog-controls.png}
\caption{The Jog control buttons}
\label{fig:jog-controls}
\end{figure}
%---------------------------------------------------------------------------

\section{Finding your bearings i.e. coordinates}
\label{sec:finding-your-bearings}

Before you start machining you need to establish the correct relationship
between the virtual part described by the coordinates in your G-code 
file and the work piece you are about to cut.

It is important to remember that EazyCNC does not really know where
the tool is, it just keeps track of the position relaying on an
initial known position. Further even though EazyCNC can keep track of the
tool position it knows next to nothing about the workpiece location.

There are ways to calibrate all three (the EazyCNC coordinate system, the actual physical axis positions 
and the workpiece location) in relation to each other,
 but most of the time this would be an overkill.

The relationship between the physical axis position and the
coordinate system only really matter for two things: if you enable
the axis limits and if you need to continue machining after
a power down.

In the first instance, if EazyCNC does not have the correct relationship
between the physical axis position and it's internal coordinate
system then it will allow you to drive the axis bang against
one end of the movement and will unnecessarily limit your movements
at the other end.

In the second instance if you lose power, a fuse blows or the 
job takes so long that you can't do it all at one go, unless the
system can be returned to the same axis coordinate system relationship
you will not be able to continue machining accurately after 
the system has been powered off.

The relationship between the EazyCNC coordinate system and
the workpiece really only matters if you need to continue machining, see
above, or you need to remove and remount the workpiece during the
course of machining, work on existing part or the work piece
is very close in size to the finished part. 

\section{The easy and lazy way}

The easiest way to setup the coordinate systems is as follows. Just 
bolt down the workpiece. Jog the axes to ensure that you can
reach all the parts to be machined without hitting the
axes limits. Then jog the tool to an approximate location 
whose coordinates you know  and set the DROs to those coordinates.

If you create your G-codes so that the finished parts
smallest coordinates are at 0,0,0 i.e the part extends
to the right, forward and up of the origin, 
then all you have to do after mounting 
your workpiece is to drive the tool to the bottom left corner
on top of the workpiece, press 'ZERO' buttons to zero X and Y 
and type in the height of the work piece in the Z axis DRO 
and you are good to go!


For a simple system like a plasma cutter where the
motors are not strong enough to cause any damage even
if you hit ends of the movement above is very adequate, 
no home/reference or limit switches,no calibration, no nothing.

For milling you actual you want to drive the tool to the bottom
left corner of the workpiece so that the tool is well within the workpiece 
so that there is good margin for cutting. And you also want to
use set the Z-axis DRO to value slightly less than the height of
the workpiece so that the cutter will not foul the table.

\section{Going pro}

If you need or want to align EazyCNC's coordinate system
and the physical axes positions the easiest way is to
have the home/reference switches installed. With those
all you have to do after powering up the system is to
press the 'HOME' buttons and the system will drive the
axis to the switches and reset the coordinate positions.

For that to work accurately the switches need to perform
repeatably so optical gates are the way to go. On the other
hand optical switches need to be protected from chips
and dust.

As the axes will travel at slow speed during 'homing'
it is a good idea to have the reference switch in the
middle of the movement range instead of at the end
as this will cut down the homing time.

It is also possible to just carefully run the axes 
to end stops and zero the DROs or type in a know value
for that position.

Above takes care of the EazyCNC coordinate system
and physical machine axis relationship.

To 'calibrate' the workpiece to machine relationship
you can use what ever mechanically accurate method
you want. Simplest being using known sized gauge 
blocks. Or if you equip your system with a touch
probe you can use that to measure the workpiece location.



\section{Adjusting Feed Rate}
\label{sec:feed-override}

Feedrate is controlled by the \texttt{F}-word in your
G-code programs but can be override with the 
feed override controls illustrated in Figure~\ref{fig:feed-override}.

The idea is that while the G-code programmer designs at 
what feedrate the part should be cut and
programs that into the code it sometimes it is necessary to adjust or fine
tune that while machining.

You can turn the override on and off by clicking he 'ON' button
and you can adjust the override with the '+' and '-' buttons. 

When the override is off the display shows the current
feedrate as set by the last \texttt{F}-word executed,
when the override is off the display shows the overridden
value. 

The override can be expressed either as a percentage of the
\texttt{F}-word value or as units/minute. To change that
click on the button to the right of the display field.

You can adjust the override with the '+' and '-' keys
or you can type in a new value. Adjusting or setting
the override value will turn on the override.


%---------------------------------------------------------------------------
\begin{figure}[htb]
\includegraphics[scale=0.8]{feed-override.png}
\caption{The Feed Override controls}
\label{fig:feed-override}
\end{figure}
%---------------------------------------------------------------------------

\section{Controlling the Spindle}

Spindle is mainly controlled by your G-code program
using  \texttt{S}-word for setting the speed, \texttt{M3} to turn
the spindle 'on' clockwise, \texttt{M4} to turning
the spindle 'on' counter clockwise and \texttt{M5} to turning the spindle 'off'.

You can manually control or override the G-code program commands with
the spindle control illustrated in Figure~\ref{fig:spindle-speed}, but as 
soon the G-code execution reaches next spindle control command that
command will take over.


When the override is off the display shows the current
spindle speed as set by the last \texttt{S}-word executed,
when the override is off the display shows the overridden
value. 

You can adjust the override with the '+' and '-' keys
or you can type in a new value. Adjusting or setting
the override value will turn on the override.


%---------------------------------------------------------------------------
\begin{figure}[htb]
\includegraphics[scale=0.8]{spindle-speed.png}
\caption{The Spindle controls}
\label{fig:spindle-speed}
\end{figure}
%---------------------------------------------------------------------------







\section{Machining!}

\noindent
Finally!

Executing G-code or machining with G-code is basically very simple.

You set the operation mode, simulation or machining, with the 'SIMU' 
or 'MACH' button.

You load your G-code program with the 'Open' button.

Then you start the execution with the 'RUN' button,
Figure~\ref{fig:run-hold-stop}.

%---------------------------------------------------------------------------
\begin{figure}[htb]
\includegraphics[scale=0.8]{run-hold-stop.png}
\caption{The G-code execution control buttons}
\label{fig:run-hold-stop}
\end{figure}
%---------------------------------------------------------------------------



This will cause EazyCNC to interpret the G-codes line by line and 
control the motors and axis accordingly.

\subsection{Pausing the machining}

If you need to temporarily pause the execution, for example to clean up some swarf
from the work area, you can press the 'HOLD' button. 

Note that pausing the machining may slightly change the actual machined toolpath because 
pausing changes the speed of movements and this in turn may affect how corners are cut.

To continue machining you press the 'RUN' button.

You can't move the tool 'manually' with the jog controls while the G-code is executing, 
but you can move it when the system is in the paused or 'HOLD' state. If you move
the tool 'manually' EazyCNC will, when you hit the 'RUN button,  automatically return the tool the the position 
where it was before you jogged it.

In the hold state the spindle and coolant will keep running if they were 'on' when
you paused the machining. You can
turn them on/off manually with the respective control buttons, but they will NOT
be automatically restored to their original state when you resume machining.


EazyCNC may also enter the hold state when it encounters one of the pause 
codes \texttt{M0}, \texttt{M1} or \texttt{M60}. 

When EazyCNC encounters the \texttt{M1} code it only enters the hold state if
the M1-pause switch is activated, Figure~\ref{fig:m1-lock}.

%---------------------------------------------------------------------------
\begin{figure}[htb]
\includegraphics[scale=0.8]{m1-lock.png}
\caption{The M1-pause switch}
\label{fig:m1-lock}
\end{figure}
%---------------------------------------------------------------------------


\subsection{Stepping and Reversing}

Sometimes, especially when simulating and examining G-code, you may want to execute
the G-code one line at a time. To do that active the 'STEP' button, 
Figure~\ref{fig:step-reverse}.
%---------------------------------------------------------------------------
\begin{figure}[htb]
\includegraphics[scale=0.8]{step-reverse.png}
\caption{The step and reverse execution control buttons}
\label{fig:step-reverse}
\end{figure}
%---------------------------------------------------------------------------

When the 'STEP' button is active the execution of G-code will automatically
enter the pause or 'HOLD' state after executing every line of G-code.

An other use for the 'STEP' feature is in combination with the 'REVERSE'
button. In general you can't run G-code backwards because you can't un-machine
material back to the work piece or un-flow the coolant, but sometimes you
want re-run some cut or G-code line because for example the cutter broke
or you had a flameout in plasma cutting in the middle of the movement.

If something like that happens you press the 'HOLD' button to pause the 
machining. Then you active the 'STEP' and 'REVERSE' buttons and press the
'RUN' button to run backwards as required to get to a position before
where your the tool bit broke or the arc flamed out. Don't forget
to deactivate 'STEP' and 'REVERSE' afterwards.

Next you stop the spindle to change the cutter and restart the spindle.

Remember to install a kill switch for the spindle/torch 
and always use it before touching the spindle!

Once you have the cutter changed you can continue machining by hitting
the 'RUN' button.

\subsection{Stopping}

The machining will of course automatically stop when it reaches the
end of the G-code file or if it encounters a stop code, M2 or M30.

If you hit the 'STOP' button the machining will stop and spindle and coolant will
be turned off and the G-code is re-wound to the beginning, so that next time
you hit 'RUN' it will execute from the beginning. 


\section{Setting up and managing the coordinate systems}

This section discusses the coordinate systems in detail.

\subsection{Coordinate axes}

Figure~\ref{fig:coordinate-system} illustrates the coordinate axes. When standing
in front of the machine and looking into the machine X-axis runs from left
to right, Y axis from front to back and Z axis from bottom to top. This is a
well established convention in CNC world.

The origin of the coordinate system where ever your home/ref switches are
positioned or if you don't have a home/ref switch installed and enabled for
an axis then origin is where ever that axis motor is when you press the 'HOME'
switch. 

The physical origin of the coordinate axis is not really relevant
most of the time as you need to 'calibrate' or set your work coordinate 
system with respect to the the work piece anyway as explained in the next section.


%---------------------------------------------------------------------------
\begin{figure}[htb]
%\includegraphics[scale=0.8]{coordinate-system.png}
\caption{The coordinate systems}
\label{fig:coordinate-system}
\end{figure}
%---------------------------------------------------------------------------


\subsection{Work/Fixture Coordinate System/Offsets}

Work coordinate system, also know as fixture coordinate system
or offsets, is the main thing relates the axes physical positions to the
the coordinates displayed in the DROs or the G-codes coordinates.

EazyCNC supports 255 work coordinate systems, you are unlikely to
ever need more than a few, most people use only one.

Mathematically a work coordinate system is simply a number for each axis
that is added to the DRO or G-code value axis value to convert them to
the physical axis, so basically they off set your coordinates, hence the name.

Changing the coordinate system never moves the tool or axes but
it will of course change how much or to where commands move the
axes after the coordinate system is changed.

The purpose of the work/fixture coordinate systems is to
allow you to set up multiple workpieces and 'calibrate'
a separate coordinate system for each and then machine
them at one go instead of machining them one by one. Using
multiple coordinate system is hardly ever worth the effort
in a hobby installation, so I suggested you use coordinate
system number one and stick with it.

You manage the coordinate systems with the Work Offsets screen,
see Figure~\ref{fig:work-offsets-view}.



%---------------------------------------------------------------------------
\begin{figure}[htb]
\includegraphics[scale=0.4]{work-offsets-view.png}
\caption{The Work Offsets screen}
\label{fig:work-offsets-view}
\end{figure}
%---------------------------------------------------------------------------

\subsection{Selecting the active coordinate system}

At any given time one of the coordinate systems is in effect.
To switch from current coordinate system to an other click on
one the buttons labeled with '1 (G54)' etc or type in the
number of the coordinate system you want to use into the 
entry field labeled 'Active Offset'. The coordinate systems 
are numbered from 1 to 255.


You can also, of course, change the active coordinate system
in you G-code. To remind you of this and make the connection
in your mind between the coordinate systems and the G-codes
the corresponding G-code is shown on the button and to the
right of the 'Active Offset' entry field.

\subsection{Changing offsets/setting up the coordinates}

One way to set up the coordinate system was already
described in Section~\ref{sec:finding-your-bearings}, that is
you drive the axis to a know or desired position and 
set coordinates in the DROs either manually or automatically.

If you do that while you are in the 'Work Offsets' screen
you will see how the offsets in the 'X:','Y:'... entry fields
change. Or if you feel comfortable working with the offsets
directly you can just type them into the entry fields.

A third way to set offsets and  to specify the coordinate/axis
origin is by 'touching' the work piece with a probe or the cutter
in the tool in the tool holder. 

\subsection{Setting the XY-coordinate system origin via touching}

 This is
illustrated and done with the controls in the 'Set XY-origin' panel.
This feature assumes that you want to set XY origin to edge
of the work piece, typically the left/front edge of it.

Usually a special edge finder probe or wobbler tool is held
in the chuck while touching because it gives much better sensitivity
and accuracy to the touching and does not mark the work piece
and neither is likely to break like cutter would.

To use the touch feature simply Jog the axis to the work piece
edge, ensure that the probe diameter is correctly entered into
the 'Probe Diameter:' entry field and click on one of the
'Touch' buttons to indicate the edge you have the driven the
probe to. EazyCNC then calculates and sets the offsets so that
the coordinate origin for that axis is at the indicated edge
of the work piece.

\subsection{Setting the Z-coordinate system origin via touching}

 This is illustrated and done with the
controls in the 'Set Z-origin' panel. This feature assumes that you want to set Z origin to top
milling table.

Remember that physically changing the tool or cutter requires you to 
re-set the Z-axis via any of the methods described here, unless you
have a presettable toolholder system and keep the 
tool table up to date.


As setting the Z-coordinate origin via touching is dependent on the tool or cutter length 
mounted in the chuck you need to ensure that the correct tool is currently selected
and set up as described in the following Section~\ref{sec:tool-setup}. 

Touching in the Z-direction is usually done with the tool or
cutter in the chuck. To use the touch feature you will need a gauge piece about
10 mm or half an inch thick.

Start by jogging down the Z-axis close to
the milling table taking care not to actually hit 
the table with the cutter. Closer than the thickness
of the gauge piece is close enough.

Next jog \emph{up}, slowly and carefully, until your gauge 
piece just fits between the cutter and the table.

Now enter the thickness of your gauge piece into
the 'Gauge Thickness:' field and click 'Touch'.

EazyCNC then calculates and sets the Z offset so that
the coordinate origin for that axis is at the top of the
workpiece.

Note that it is much safer from the tool breakage point
of view to jog up i.e. first jog close to the milling
table without the gauge piece in place, then job up a little
bit at a time until you can just slide the gauge piece
between the table and the cutter. 

\section{Setting up and managing tool information}
\label{sec:tool-setup}

Certain features of EazyCNC rely on correct information of the
tool i.e. the cutter diameter and length. To help you to manage this
EazyCNC maintains a table of tool information for up to 256 tools. 

You manage the tool table with the Tool Setup  screen,
see Figure~\ref{fig:tool-setup-view}.



%---------------------------------------------------------------------------
\begin{figure}[htb]
\includegraphics[scale=0.4]{tool-setup-view.png}
\caption{The Tool Setup Screen}
\label{fig:tool-setup-view}
\end{figure}
%---------------------------------------------------------------------------

The tool dimension are used in movement limit calculations and
if they are not correct you may either move bang against the end
of an axis movement or EazyCNC may not allow you to move the
tool. This applies to both manual jogging and running G-code
programs.

The tool diameter is also used in the cutter compensation calculations,
and it can take advantage of the tool diameter information 
in the tool table or you can manage that in the G-code. 

Keeping the tool information in the tool table has the advantage
that the tool information can be kept up to date without touching the G-code files
as cutters wear or are changed, but for some types of machines like plasma cutters
where the width of the cut depends on the feed rate it maybe 
better managed in the G-code files themselves.

To effectively use the tool table you need a presettable toolholder 
system that allows you to repeatably change the tool without affecting
the cutter position i.e. a system that allows you to take a tool out
and put it back in later and which keeps the tool at the same height
in relation to the mill spindle. 

If you don't have a toolholder like that you can just set the cutter
manually, hopefully with a jig or something, to a correct correct 
depth in the chuck and update the tool length in the table
manually. With this method, as the length info is probably not accurate,  
you need to re-set the Z-axis coordinate system every time you change the tool. 

You may also decide that keeping the tool length information correct in 
the tool table is too much work in which case you can just set the tool length
to zero in the tool table, re-set the Z-coordinates as needed 
and disable the Z-axis limits, see Section~\ref{subsubsec:enable-limist}.

You also don't have to maintain the tool diameter as this 
can be handled in the G-code. If you find keeping the tool
diameter correct in the tool table too much work then just
set it zero and disable the XY-axis limits.

\subsection{Setting the current tool}

To set which tool in the tool table is in use, shown
and manipulated on the screen 
click on one the buttons labeled with '1 (T1)' etc or 
type in a number you want to use into the 
entry field labeled 'Current Tool'. The tools are numbered from
1 to 256.

You can also set the current tool in you G-code program
with the \texttt{T}' word. To remind you of this and make the connection
in your mind between the tool table entries and the G-codes
the corresponding T-word is shown on the button and to the
right of the 'Current Tool' entry field.


Note that EazyCNC relies on you to keep the physical 
world and the software in sync. Just changing
the tool number in you G-code program with the
'\texttt{T} is not enough, you
actually have to physically change the tool. To
be able to do that you need to stop the spindle
and pause the G-code execution by adding following
sequence to you program.

\begin{verbatim}
M5                     ; Spindle off
T 5                    ; Select tool 5
M0 (MSG Load tool 5)   ; Pause with a message to the operator
\end{verbatim}

When EazyCNC encounters above sequence it sets the current
tool to 5 and pauses i.e. enters the 'HOLD' state, allowing you
to change the tool after which you hit 'RUN'. 



Don't forget to wait for the spindle to stop and 
use the kill switch before touching the spindle!

If you change the tool length or diameter 
in the tool table when you change the tool  those updated length and 
diameter values are \emph{not} automatically used. Instead
you need to click one of the G40,G41,G42,G43 and G49 buttons
as needed if you update the tool table during machining.

\subsection{Managing the tool diameter and length}

To set or examine a tool table entry for a tool
first set it as the current tool and then
view or type in the tool diameter and length
in the 'Diameter:' and 'Length:' fields. 

You can also update the tooltable in your G-code program
with \texttt{G10 L1 P\emph {toolno} Z \emph radius X \emph{length}} command; you
could for example have a G-code program that
updates all of your tool table. 

\subsection{Setting the tool length via touching}


This is illustrated and done with the
controls in the 'Set Tool Length' panel.

This feature expects that you have set your Z-coordinate system
correctly, which may sound sound a bit odd as setting the
Z-coordinates system via touching requires that the tool
height is correct! The way around this problem is to first
set the tool length to zero, calibrate the Z-axis with no
tool in the holder and then proceed to set the tool heights
via touching.

Don't forget that all this is hardly worth the trouble if
you do not have a re-settable toolholder, otherwise you
can just always re-set the Z-axis and be done with it. 


To set the tool length via touching you need 
a gauge piece of known thickness. The thickness
is not important but about 10 mm or half an inch is good.

First enter
the correct tool number into the 'Current Tool'
entry field and then mount the corresponding cutter
into the spindle. Then jog down the Z-axis close to
the milling table taking care not to actually hit 
the table with the cutter. Closer than the thickness
of the gauge piece is close enough.

Next jog \emph{up}, slowly and carefully,  until your gauge 
piece just fits between the cutter and the table.

Now enter the thickness of your gauge piece into
the 'Gauge Thickness:' field and click 'Touch'.

EazyCNC then calculates and sets the Tool Length.





%You can work with both and even mix them but it is suggested that for sake of
%sanity you pick one or the other and stick with it.

%You can change the unit used in 'Mach Setup' screen, see section~\ref{sec:mach-setup-units}.

%If you want to change the unit from 'mm' to 'inch' or back you can do it at anytime
%and it won't change anything except the displayed values because internally
%EazyCNC maintains the correct physical values. 

%So for example if you change the unit from 'mm' to 'inch'
%and your X-axis 'steps/mm' value was \emph{1 steps/mm} it will be displayed as
%\emph{254 step/inch} and everything still works as they should. And similarly
%for movement limits etc they all retain their original physical meaning
%regardless of the unit, as they should.

%There are well establish conventions in CNC world for axes and units.

