\chapter{Installation}


\section{Getting Java}

Starting from version 0.0.0.20a EazyCNC embeds the Java
virtual machine so you do not need to install it, you
can skip this section altogether.

As a pre-requisite you need to have Java installed in
your system. You can get Java from

http://java.com

Mac OS version up to 10.6 (Snow Leopard) have Java pre-installed and you
should not install a newer version from Oracle.

In Linux you can also install Java with one of the
various package managers. 

Regardless how you get it, ensure that you have Oracle (or Apple) 
Java, not OpenJDK or anything else as they have not been 
tested to work with EazyCNC.

In the future Java will be built in so this will become
a moot point.


\section{Getting the Application}


Note that do not need to have actual hardware or any drivers
installed to run and play with the software.

EazyCNC is distributed via Internet so just download the file
appropriate for your operating system from the EazyCNC website
at:

http://www.eazycnc.com/downloads/downloads.php

The download size is about 25 MB.

Depending on your operating system EazyCNC is distributed as
a compressed file which you may have to un-compress with the
tools in you operating system.

\section{Installing the Application}

Remember that for versions of EazyCNC prior to 0.0.0.20a
you should have the Java Runtime Environment 
installed first. 

EazyCNC comes as a single executable file which does not
require an installer. Just copy the file to where ever you
want in your computers hard disk. 

If you prefer you can create a shortcut aka alias for the
application and put it on the Start Menu or Desktop or drag the icon into the Dock.



To launch the application just double click on the file. 
Note that after clicking op the icon ,it takes a while
before you see the EazyCNC main screen, especially on the
first launch. If you are impatient and launch the program
twice you will get programs running at the same time, which
is fine but probably not what you want.


If you have Java installed the program will start.

In Linux you may have to give the application execute
permissions first. You can you do that my right clicking
at the application file and selecting 'Properties' and
in the Dialog that appears you should be able to set
the file as executable.





